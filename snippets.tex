
\marginnote{
    \begin{rem}[\textbf{Top}]
    The category of topological spaces and continuous functions.
    \end{rem}
}

\begin{defn}[Powerset monad on \textbf{Top}]
Overloading the powerset symbol $\mathcal{P}$, we define the functor $\mathcal{P}: \mathbf{Top} \rightarrow \mathbf{Top}$ to send:
\begin{description}
\item[objects:] $X^\tau \mapsto \mathcal{P}X^{\mathcal{P}\tau}$, where $\mathcal{P}X$ is the powerset of $X$, and we define the new topology \[\mathcal{P}\tau := \{ S \subseteq \tau : \bigcup S \in \tau \}\]
\item[morphisms:] $f: X^\tau \rightarrow Y^\sigma$ to $\mathcal{P}f: \mathcal{P}X^{\mathcal{P}\tau} \rightarrow \mathcal{P}Y^{\mathcal{P}\sigma}$, where $\mathcal{P}f$ is the direct image map.
\[\mathcal{P}f(J \subseteq X) := f(J) \subseteq Y\]
\end{description}
This endofunctor $\mathcal{P}$ on \textbf{Top} is a monad, with unit and multiplication inherited from the powerset monad on \textbf{Set}. We present these explicitly, and verify their continuity:
\begin{description}
\item[unit:] $X^\tau \overset{\eta_{X^{\tau}}}{\longrightarrow} \mathcal{P}X^{\mathcal{P}\tau} := x \in X \mapsto \{x\} \in \mathcal{P}X$.
For continuity, if $S \subseteq \mathcal{P}X$ is open, by definition of $\mathcal{P}\tau$, $\bigcup S \subseteq X$ is open
\item[multiplication:] $\mathcal{P}\mathcal{P}X^{\mathcal{P}\mathcal{P}\tau} \overset{\mu_{X^{\tau}}}{\longrightarrow} \mathcal{P}X^{\mathcal{P}\tau}: S \subseteq \mathcal{P}\tau \mapsto \bigcup S \in \mathcal{P}\tau$
\end{description}
\end{defn}




\begin{figure}\label{space:line}
\tikzfig{testspaces/unitinterval}
\caption{The \textbf{unit interval} has the unit interval $[0,1]$ as the underlying set. Open sets are unions of open intervals $(x,y)$. Closed intervals and points are closed. We denote this space $\sim$.}
\end{figure}

\begin{defn}[Partial Functions]
A \textbf{continuous partial function} $X^\tau \rightarrow Y^\sigma$ is the intersection of a continuous function and a bowtie. 
\end{defn}

\marginnote{
\begin{intuition}\label{intuit:screen}
View the space as a screen display, with points as pixels. Then states are "shapes" of lit up pixels, and effects are "tests", which check for lit pixels in an open set. In a discrete topology, all shapes are distinguishable by testing. In an indiscrete topology, the only distinction is between "no shape" and "some shape".
\end{intuition}
}





\subsection{ Compact Hausdorff spaces are arachnid }

\marginnote{

\begin{rem}[Compactness]
An \textbf{open cover} of a space $X^\tau$ is a set of opens $\{U_i: i \in I\} \subset \tau$ such that:
\[ \bigcup\limits_{i \in I} U_i = X \]
A \textbf{subcover} of an open cover is a subset of the open cover that is still an open cover.
$X^\tau$ is \textbf{compact} when every open cover has a finite subcover.
\end{rem}

\begin{rem}[Hausdorff]
$X^\tau$ is \textbf{Hausdorff} when for any two distinct points $a,b \in X$ there are disjoint opens $U,V \in \tau$ such that $a \in U$ and $b \in V$. The mnemonic is that the points are "housed-off".
\end{rem}

}

\begin{defn}[Non-unital Spider]
A \textbf{non-unital spider} on $X^\tau$ is a tuple [[[placeholder]]] that satisfies the following relations:

\end{defn}

\begin{defn}[Arachnidity]
Call $X^\tau$ \textbf{arachnid} if it admits a family of non-unital spiders such that for any finite family $\mathbb{K}:=\{K_i \subset X : i \in I, \ |I| \in \mathbb{N}\}$ of subsets of $X$, there exists a non-unital spider on $X^\tau$ such that:

[[placeholder]]

\end{defn}

\begin{theorem}[Compact Hausdorff is Arachnidity]
$X^\tau$ is compact and Hausdorff iff $X^\tau$ is arachnid.
\end{theorem}

We prove this by first relating arachnidity to some conditions that are phrased in the "distinguishing shapes" setting of Intuition \ref{intuit:screen}, and then bridge to compact Hausdorff.

\subsection{Arachnidity, Distinguishing, Encoding, Recognition}

\newthought{When is the topology of a space "good"?} There are several possible criteria for "goodness", such as the ability to distinguish different shapes, encode those shapes in the results of testing, and reconstruct those shapes from an encoding.

\marginnote{
    \begin{remark}
        Where $\tau$ of all opens represents all possible tests, a test battery is a specific set of tests.
    \end{remark}
}
\begin{defn}[Test battery]
A \textbf{test battery} of $X^\tau$ is a subset of $\tau$.
\end{defn}
\marginnote{
    \begin{remark}
    A curriculum is a range of possible shapes.
    \end{remark}
}
\begin{defn}[Curriculum]
A \textbf{curriculum} of $X^\tau$ is a set of subsets of $X$.
\end{defn}

\marginnote{
\begin{remark}
If $X^\tau$ is not distinguishing, then there are two distinct shapes $J \neq K$ that no test can tell apart.
\end{remark}    
}
\begin{defn}[Distinguishing]\label{def:distinguish}
$X^\tau$ is \textbf{distinguishing} when
\[\forall J,K_{(\subseteq X)} (J \neq K \iff \exists U_{(\in \tau)}: placeholder)\]
\end{defn}

\marginnote{
    \begin{remark}
    An encoding is a binary spectrum; for each test in the battery, a bit indicating whether that test has failed or succeeded.
    \end{remark}
}
\begin{defn}[Encoding]
An \textbf{encoding} of $K \subset X$ with respect to a test battery $\mathbb{T}: \{U_i : i \in I, U_i \in \tau\}$ is a direct sum:
\[placeholder\]
\end{defn}

\marginnote{
    \begin{remark}
    A recogniser can take any finite curriculum and provide a finite test battery such that shapes can be distinguished via their finite encodings. Finiteness is a weak but necessary condition computability -- i.e. practical feasibility.
    \end{remark}
}
\begin{defn}[Recognition: computable distinguishing encodings]
$X^\tau$ is a \textbf{recogniser} if for any finite $\mathbb{K} := \{ K_i \subseteq X : i \in I, \ |I|\in \mathbb{N} \}$, there exists a finite test battery $\mathbb{U} := \{ U_j \in \tau: j \in J, \ |J|\in \mathbb{N} \} \subseteq \tau$ such that:
\[K_a \neq K_b \iff placeholder\]
\end{defn}

\marginnote{
    \begin{remark}
    Instead of providing a different test battery for each curriculum, it would be more convenient to extend the test battery whenever an unseen shape enters the curriculum; this is what a learner does.
    \end{remark}
}

\begin{defn}[Learning: extendable recognisers]
A recogniser is a \textbf{learner} when, for arbitrary finite curricula $\mathbb{K}' \supset \mathbb{K}$ and a finite test battery $\mathbb{U}$:
\[placeholder \iff \exists \mathbb{U}'_{(\mathbb{U} \subseteq \mathbb{U}' \subseteq \tau)}:( |\mathbb{U}'| \in \mathbb{N} \wedge placeholder)\]

\end{defn}

\marginnote{
    \begin{remark}
    Recall 
    \end{remark}
}
\begin{defn}[Recall: reconstruction from encodings]

\end{defn}

\marginnote{
    \begin{remark}

    \end{remark}
}

\begin{defn}

\end{defn}



\section{Enrichment structure}



Denote by $(X \times Y)^{(\tau \multimap \sigma)}$ the topological space of topological relations of type $X^\tau \rightarrow Y^\sigma$ as given above. We show that this topology is finer than the product topology.

\begin{proposition}
For any $X^\tau$ and $Y^\sigma$, $\tau \times \sigma \subseteq \tau \multimap \sigma$
\begin{proof}
Let $\mathfrak{b}_\tau, \mathfrak{b}_\sigma$ be bases for $\tau$ and $\sigma$ respectively, then $\tau \times \sigma$ has basis $\mathfrak{b}_\tau \times \mathfrak{b}_\sigma$. An arbitrarily element $(t \in \tau, s \in \sigma)$ of this product basis can be viewed as a topological relation $t \times s \subseteq X \times Y$. Every open of $\tau \times \sigma$ is a union of such basis elements, and topological relations are closed under arbitrary union, so we have the (evidently injective) correspondence:
\[ \tau \times \sigma \ni \bigcup\limits_{i \in I}(t_i \times s_i) \mapsto \bigcup\limits_{i \in I}(t_i \times s_i) \in \tau \multimap \sigma \]
\end{proof}
\end{proposition}

\marginnote{
\begin{example}[$\tau \multimap \sigma \nsubseteq \tau \times \sigma$]
Recalling Proposition \ref{prop:states}, let $\tau = \{\varnothing,\{\star\}\}$ on the singleton, and $\sigma$ be an arbitrary nondiscrete topology on base space Y. $(\{\star\} \times Y)^{(\tau \times \sigma)}$ is isomorphic to $Y^\sigma$, but $(\{\star\} \times Y)^{\tau \multimap \sigma)}$ is the isomorphic to the discrete topology $Y^\bullet$. For a more concrete example, consider the Sierpi\'{n}ski space $\mathcal{S}$ again, along with the topological relation $\{(0,0),(1,0),(1,1)\} \subset \mathcal{S} \times \mathcal{S}$; due to the presence of $(0,0)$, this topological relation cannot be formed by a union of basis elements of the product topology, which are:
\begin{description}
\item[$\{1\} \times \{1\}$ =] $\{(1,1)\}$
\item[$\{1\} \times \{0,1\}$ =] $\{ (1,0),(1,1) \}$
\item[$\{0,1\} \times \{1\}$ =] $\{ (1,1),(0,1) \}$
\item[$\{0,1\} \times \{0,1\}$ =] $\{ (0,0), (0,1), (1,0), (0,1) \}$
\end{description}

\end{example}
}

\begin{proposition}
$\tau \multimap \sigma = \tau \times \sigma \iff $
\begin{proof}

\end{proof}
\end{proposition}

\clearpage

\subsection{How \textbf{TopRel} relates to other well-known categories}

\[\begin{tikzcd}
    &&& {\mathbf{Loc}} \\
    \\
    \\
    {\mathbf{Top}} &&& {\mathbf{TopRel}} &&& {\mathbf{Rel}} \\
    \\
    \\
    &&& {\mathbf{Set}}
    \arrow[""{name=0, anchor=center, inner sep=0}, "{(-)_U}", curve={height=-12pt}, shorten <=8pt, shorten >=8pt, from=4-4, to=4-7]
    \arrow[""{name=1, anchor=center, inner sep=0}, "{(-)_D}", curve={height=-12pt}, shorten <=8pt, shorten >=8pt, from=4-7, to=4-4]
    \arrow[""{name=2, anchor=center, inner sep=0}, "{(-)_U}", curve={height=-12pt}, shorten <=10pt, shorten >=10pt, from=4-1, to=7-4]
    \arrow[""{name=3, anchor=center, inner sep=0}, "{(-)_D}", curve={height=-12pt}, shorten <=10pt, shorten >=10pt, from=7-4, to=4-1]
    \arrow[""{name=4, anchor=center, inner sep=0}, "{(-)_L}", curve={height=-12pt}, shorten <=10pt, shorten >=10pt, from=4-1, to=1-4]
    \arrow[""{name=5, anchor=center, inner sep=0}, "{(-)_P}", curve={height=-12pt}, shorten <=10pt, shorten >=10pt, from=1-4, to=4-1]
    \arrow[shorten <=5pt, shorten >=5pt, from=4-4, to=1-4]
    \arrow["i"{description}, shorten <=10pt, shorten >=10pt, hook, from=7-4, to=4-7]
    \arrow["i"{description}, shorten <=8pt, shorten >=8pt, hook, from=4-1, to=4-4]
    \arrow["\dashv"{anchor=center, rotate=45}, draw=none, from=3, to=2]
    \arrow["\dashv"{anchor=center, rotate=90}, draw=none, from=1, to=0]
    \arrow["\dashv"{anchor=center, rotate=-44}, draw=none, from=4, to=5]
\end{tikzcd}\]

Here are two ways to think about \textbf{TopRel} in relation to other categories. First, we can think of the analogy
\[\mathbf{TopRel}:\mathbf{Rel}::\mathbf{Top}:\mathbf{Set}\]
This analogy holds up to the free-forgetful adjunction between \textbf{Top} and \textbf{Set} embedding into a free-forgetful adjunction between \textbf{TopRel} and \textbf{Rel}.\marginnote{
    \begin{remark}[\textbf{TopRel} is not Kleisli in the obvious way]
    However, while \textbf{Rel} is the Kleisli category of \textbf{Set} with respect to the powerset monad, \textbf{TopRel} doesn't seem to be the Kleisli category of \textbf{Top} in an analogous way. The obvious notion of a powerset topology $\mathcal{P}\tau := \{S \subset \tau : \bigcup S \in \tau\}$ agrees with our na\"{i}ve definition of continuous relation, but the underlying unit and multiplication maps from the powerset monad on \textbf{Set} fail to be continuous with respect to this topology.
    \end{remark}
}



\begin{proposition}[\textbf{Rel} and \textbf{TopRel} free-forgetful adjunction]

\begin{proof}

\end{proof}
\end{proposition}


\begin{proposition}[(Outer square)]

\begin{proof}

\end{proof}
\end{proposition}

\begin{rem}[Locales, and the \textbf{Top}-\textbf{Loc} adjunction]

\end{rem}

\begin{proposition}[(Inner square)]

\begin{proof}

\end{proof}
\end{proposition}