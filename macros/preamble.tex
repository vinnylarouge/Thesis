\documentclass{tufte-book}
\setcounter{tocdepth}{4}
\setcounter{secnumdepth}{4}

%\usepackage{scrextend} 
%\changefontsizes[20pt]{12pt}

\geometry{a4paper,landscape,left=24.8mm,top=27.4mm,headsep=2\baselineskip,textwidth=164mm,marginparsep=8.2mm,marginparwidth=82mm,textheight=34\baselineskip,headheight=\baselineskip}

%\geometry{showframe}% for debugging purposes -- displays the margins

\usepackage{stmaryrd}

\usepackage{amsmath,amsthm,amssymb}
\theoremstyle{definition}
\newtheorem{theorem}{Theorem}[section]
\newtheorem*{theorem*}{Theorem}
\newtheorem{corollary}[theorem]{Corollary}
\newtheorem{lemma}[theorem]{Lemma} 
\newtheorem{proposition}[theorem]{Proposition}
\newtheorem{conj}[theorem]{Conjecture}
\newtheorem{defn}[theorem]{Definition}
\newtheorem{fact}[theorem]{Fact} 
\newtheorem{example}[theorem]{Example} 
\newtheorem{examples}[theorem]{Examples}
\newtheorem{example*}[theorem]{Example*}
\newtheorem{examples*}[theorem]{Examples*}
\newtheorem{remark}[theorem]{Remark}
\newtheorem{remark*}[theorem]{Remark*}
\newtheorem{question}[theorem]{Question}
\newtheorem{assumption}[theorem]{Assumption}
\newtheorem{conjecture}[theorem]{Conjecture}
\newtheorem{convention}[theorem]{Convention}
\newtheorem{justification}[theorem]{Justification} 
\newtheorem{construction}[theorem]{Construction}
\newtheorem{rem}[theorem]{Reminder}
\newtheorem{intuition}[theorem]{Intuition}
\newtheorem{term}[theorem]{Terminology}
\newtheorem{scholium}[theorem]{Scholium}
\newtheorem{requirement}[theorem]{Requirement}

\usepackage{tikz-cd}
\usepackage{macros/tikzfig}
\usepackage{macros/quiver}
\input{macros/thesis.tikzstyles}

% Set up the images/graphics package
\usepackage{graphicx}
\setkeys{Gin}{width=\linewidth,totalheight=\textheight,keepaspectratio}
\graphicspath{{graphics/}}

\title{String diagrams for text}
\author[V.W.]{Vincent Wang-Ma\'{s}cianica}
\date{\today}

% The following package makes prettier tables.  We're all about the bling!
\usepackage{booktabs}

% The units package provides nice, non-stacked fractions and better spacing
% for units.
\usepackage{units}

% The fancyvrb package lets us customize the formatting of verbatim
% environments.  We use a slightly smaller font.
\usepackage{fancyvrb}
\fvset{fontsize=\normalsize}

% Small sections of multiple columns
\usepackage{multicol}

% Squares
\usepackage{stix}

% Provides paragraphs of dummy text
\usepackage{lipsum}

% These commands are used to pretty-print LaTeX commands
\newcommand{\doccmd}[1]{\texttt{\textbackslash#1}}% command name -- adds backslash automatically
\newcommand{\docopt}[1]{\ensuremath{\langle}\textrm{\textit{#1}}\ensuremath{\rangle}}% optional command argument
\newcommand{\docarg}[1]{\textrm{\textit{#1}}}% (required) command argument
\newenvironment{docspec}{\begin{quote}\noindent}{\end{quote}}% command specification environment
\newcommand{\docenv}[1]{\textsf{#1}}% environment name
\newcommand{\docpkg}[1]{\texttt{#1}}% package name
\newcommand{\doccls}[1]{\texttt{#1}}% document class name
\newcommand{\docclsopt}[1]{\texttt{#1}}% document class option name

\usepackage{bussproofs}

\usepackage{xcolor}
\usepackage{xspace}
\def\bB{\begin{color}{blue}}
\def\bO{\begin{color}{orange}}
\def\bG{\begin{color}{green}}
\def\bM{\begin{color}{magenta}}
\def\e{\end{color}\xspace}

\usepackage{fontawesome}

%\usepackage{tocloft}
%\cftsetindents{section}{0em}{2em}
%\cftsetindents{subsection}{0em}{2em}
%\renewcommand\cfttoctitlefont{\hfill\Large\bfseries}
%\renewcommand\cftaftertoctitle{\hfill\mbox{}}