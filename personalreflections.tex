\section{Personal reflections}

\subsection{The quantum linguistics myth}
Now there are two implications of the quantum linguistics myth that I wish to dismiss here. The first that is often touted is that there is some kind of fruitful and practical synthesis to be won from the meeting of grammatical structure and vectorial semantics. This is the same kind of professionally-sensible claim that certain mathematicians will sometimes make in other domains too: that a deep consideration of structure will ultimately pay practical dividends. As far as I can tell in the case of DisCo and offshoots, this hasn't yet been demonstrated to be true. There's just no setting we know of, classicial or quantum, where deliberately introducing grammatical structure helps with any practical task, and shifting the goalposts to interpretability or whatever else has not worked either. So it isn't for lack of technical ability or imagination, it just appears to be that anything practical one could have done with "structure" or "composition" you could have also done without (given sufficient data and compute). This isn't to say that the structural view is totally without merit --- it is certainly more human-friendly and aids in Interpretability writ large as subsuming pedagogy and communication --- it's just impractical. There is a common ground optimistic stance on structure held by both linguists and machine learning theorists that lies in the intersection of "poverty of the stimulus" and geometric deep learning: that using structure to limit the inductive biases of a learner will allow them to learn whatever more effectively, an argument that admits that structure wouldn't let you do anything extra, but still posits a positive contribution. As far as the evidence from machine learning goes, I'm not sure that this is true, but I am on the lookout for evidence to the contrary. Anyway, this was surprising and disappointing for me, because of a more deep-seated belief in myself I have had to kill, that \emph{structure is magic}. If you are a computational linguist, I welcome you to try synthesising your structural formalisms with vectorial semantics across similar bridges as are built here, and I would be happy to be proved wrong about the importance and power of structure: the disenchanted view that I currently hold is that adding structure is, at best, a way to get computationally cheaper but worse answers.\\

There is a second, unspoken implication that is more seductive, and particular to the (essentially accidental) detour through "quantum": that there exists some deep and fundamental unity between quantum theory and natural language. This is a rather commonplace sin more generally, that in some field XYZ with relatively little mathematical sophistication someone will steal the valour of physics by squatting on "Quantum XYZ", or "Quantum-inspired XYZ", when all they really mean is e.g. the use of noncommuting operators or tensor products or density matrices or some other narrow mathematical facet of quantum theory. Such views by themselves may be harmless, but in conjunction with the vaguely held but common metaphysical assumption of mathematical realism among physicists, we find ourselves in trouble. The extent to which there are quantum theories of linguistics or cognitive science or anything else deep and multifaceted is that there are mathematical paintbrushes that have been used to illustrate quantum theory with which we can also better sketch and appreciate certain limited phenomena in other domains. However, even this honest appraisal is co-opted by a blameless form of motte-and-bailey, where a serious faction of practitioners will disavow such mysticism, but the field as a whole survives by luring na\"{i}ve researchers with the seductive implication. While this situation sucks, it's also fairly common\footnote{Just spitballing here, and not referring to any field in particular, I suspect that this sort of dysfunction is a natural consequence of the sociology of academic and industrial research. Insofar as academia generally (1-) does not have the resources to turn theory into practice and that (2-) there is pressure to publish among (3-) people who remain in academia, there is an understandable incentive for theorists to (-2) tell stories valued primarily against doctrinal measures of niceness, with (-1) no corresponding selection pressure for whether those stories actually translate to anything in the real world (-3) for a long time. As a result, the kinds of doctrines that manage to grow large enough organically to breach academic containment might on average be less practical than scrappy standalone ideas. But touching grass at this point is not enough to cure the ailment: it is actually a na\"{i}ve view that these doctrines crafted in ivory towers immediately perish in the sunlight of a free market economy. While these myths are endless sources of pain for the engineers who must bring them into fruition, they are crucial memetic pillars for kayfabe or some other epistemic asymmetry that serve the purpose of keeping morale high for the technically uninitiated, which includes investors. In this sense academics and capital are natural allies against embattled engineers and builders, an observation which might serve as a sufficient axiom to deduce and explain much of the workings of the academic-industrial complex. Sorry, but I had to get that off my chest. I have a kind of love-hate relationship with my work and I \emph{love} mansplaining. Thank you for indulging.}. It's really no different than becoming an architect or going to film school or trying to become a drug kingpin: come for the prestige, stay for the sunk-cost. Few positions on the top and most of the legwork is done by itinerant workers at the churning bottom. Many industries thrive on slowly consuming hopes and dreams in this way.\\

These objections would usually be fatal, but as it goes, the alternatives are no better. In fact, any defensive manoeuvre that works to justify formal linguistics as an academic activity or otherwise will provide cover for quantum linguistics too, with minor modifications such as swapping out "quantum" for some exotic dialect of logic or what-have-you.