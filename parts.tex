
\section{Shapes: cores, halos, and parthood}

\begin{defn}[Shape]
Shapes are a core and a halo.
\end{defn}

\begin{defn}[Core]
Cores are open sets.
\end{defn}

\begin{defn}[Halo]
\end{defn}

\begin{lemma}[Interactions between cores and halos in a sticky spider]\label{lem:corehalo} The following hold for all sticky spiders:\\
\begin{enumerate}
\item{Cores never overlap}
\item{Each core overlaps with exactly one halo}
\end{enumerate}
\end{lemma}

\begin{example}[Discrete Quantum Venn Diagram Paradox]

\end{example}

\begin{defn}[Parts]
When a core of a shape is equal to its halo, we call the shape a \textbf{part}. By Lemma \ref{lem:corehalo}, parts are open sets that do not overlap with any other core or halo.
\end{defn}

\begin{defn}[Mereology]
 When all the cores of a sticky spider are equal to their parts, we call the spider a \textbf{mereology}: a theory of parthood relations. A mereology on a topology $X^\tau$ is a collection of mutually disjoint open sets of $X^\tau$. Conversely, every such collection of disjoint open sets yields a mereology; the idempotent splits through the set of disjoint opens equipped with the discrete topology.
\end{defn}

\begin{proposition}[Mereologies, graphically]
A sticky spider is a mereology -- every core of the sticky spider is equal to its corresponding halo -- if and only if the following equation holds:\marginnote{\textbf{TopRel} does not have a dagger functor, so this equation is an approximation of the concept of a dagger-special-frobenius-algebra.}
\[\scalebox{1}{\tikzfig{idemproof/mereology}}\]
\end{proposition}

\begin{example}[Parts as places on a map]

\end{example}