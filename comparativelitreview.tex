
\section{The other literature review}

\subsection{Introductory remarks}

This section was written after I passed my viva, and its purpose is to situate this work with respect to the relevant literature in linguistics, for the benefit of linguists. It was formally optional for me to do so, but I felt I ought to out of a sense of scholarly conscientiousness, and because my examiners told me I should. These emotional forces have only mildly won out against the alternative of doing nothing, so I will write casually, as I would for myself. At any rate, having now read some linguistics, I suspect that linguists do not mind reading and writing wordy things, and only linguists would stick around to read this, so there will be little harm in me writing indulgently.\\

One way to summarise things is that the technical contents of this thesis point towards a nearby counterfactual history where all the linguists in the "Garden of Eden period" [Partee?] knew some of the modern structuralist mathematics that their programme obviously would have profited from. The fractiousness of linguists notwithstanding, it is my opinion that the interdisciplinarity of this thesis is accidental: from the perspective of the counterfactual history, the division of formal approaches to syntax and semantics (and some of the subdivisions thereof) would appear contrived. I am aware that this could come across as pretty arrogant and patronising talk, which is not my intention this time, because the point I am ultimately trying to make is that nothing in this thesis is particularly special, and could have been reinvented by anyone. Text circuits, and the contents of this thesis, owe no substantial intellectual debt to linguistics outside of perhaps Lambek and Firth, themselves far from the mainstream. This is because I prefer thinking about things myself as opposed to reading about what others have thought, and the relevant developments were due to collaborations with the similarly ignorant. Consequently, all of the parallels I am about to point out between the current stream of development and the main body of literature are independent rediscoveries. I take these parallels as indications of the "naturalness" of ideas that could have come about anytime and independently of specific individuals, and accordingly as evidence for the "nearbyness" of the counterfactual history I am trying to gesture at.

\subsection{Pregroups to Text Circuits vs. Transformational Grammar to Formal Semantics}

The cartoon version of the literature review is folkloric, autochtonal, and maybe a little jingoistic: it is a caricature or myth of the field that we in it tell ourselves. I think that is sufficient for most readers, but if you are here, I owe you a critical and comparative retelling. In my view, the development of DisCoCat is only two minor counterfactuals removed from the lineage of mainstream formal semantics from Montague onto Heim \& Kratzer and onwards. Moreover, both of these counterfactuals rest only on the difference of when they began relative to the ambient development of mathematical formalisms and available computing.\\

\subsubsection{Counterfactual 1: Truth-conditional vs. Vectorial Semantics}

Montague semantics may be essentially characterised as the meeting of two ideas [CITE]: structure-preserving maps from syntax, and taking truth-conditions to be the essential data of semantics. On some accounts, only the former aspect of compositionality of semantics according to syntax is essential [CITE]. Accordingly, the first counterfactual is just the swapping of truth-conditional for vectorial semantics. Today there are several good reasons to prefer the latter over the former. First, the view that truth-conditions alone are the \emph{sine qua non} of natural language meanings has been incompatible with correspondence theories of truth at least since Barr fixed Putnam's permutation argument [CITE, CITE]. Second, vectors as lists-of-numbers are more expressive and computationally practical, so much so in its current form that the very need for a formal account of "semantics" is put to question. Third, with a rare few exceptions, the truth-conditional programme and its descendents are bankrupt, and worse, have terrible mathematical taste. A lot of mathematical structure has been marshalled [CITE,CITE,CITE] to salvage the programme by trying to force intensions and pragmatics and everything-in-the-world into the propositional mould, and it is unclear what all of this mathematics buys us except for more of the same. In practical terms, the increasing extent to which statistical language models adequately handle semantics exactly matches the decreasing extent to which a complicated mathematical account of the same is warranted, and in theoretical terms it would be definitionally preposterous to seriously assert that the study of the mathematical models themselves lends insight into the phenomena they are intended to be surrogates for.

But all this criticism can only be said with the benefit of hindsight, and to give credit, it all must have seemed like a very good idea at the time. A model-theoretic, truth-conditional account of semantic data was the natural choice for a concrete target for the structure-preserving map, I speculate, for several reasons: Montague himself was a logician, and truth-conditions were at once flexible enough to capture (to a logician's satisfaction) some semantic phenomena of interest, while being amenable to computation \emph{by hand}, as was necessarily the case at the time owing to the lack of computers. Certainly there was adequate sophistication manipulating vectors by that time as well, but the vectorial view would have been more difficult to calculate with, even restricted to a setting without nonlinearities. It could be argued that, in any case, vectorial semantics in the form of word-embeddings requires a degree of data-storage and computing faculties that would not have been available until fairly recently. Moreover, even the theoretical soil was arguably unready: category theory was insufficiently spread and understood, and hence a broadly accessible mathematical understanding of structure and structure-preservation outside of particular concrete instances was unavailable.

\subsubsection{Counterfactual 2: Generative vs. Typelogical views of syntax}

- the usual mathematical conception of syntax is combinatoric --- see formal language theory.
-- now "generative" in some circles is just synonymous with "formal", but there is the original sense in which mathematical machinery generates correct sentences.
- the typelogical alternative is the proof-theoretic counterpart to a combinatoric conception, where parsing rather than generation takes precedence.
- parsing is the more practical thing for computers; compare to generation, where it becomes very complicated to make generation dependent on a semantic basis.

\subsubsection{Criticisms of Quantum Linguistics}

In light of these counterfactuals, there are two implications of the quantum linguistics myth that I wish to dismiss here. The first that is often touted is that there is some kind of fruitful and practical synthesis to be won from the meeting of grammatical structure and vectorial semantics. This is the same kind of professionally-sensible claim that certain mathematicians will sometimes make in other domains too: that a deep consideration of structure will ultimately pay practical dividends. As far as I can tell in the case of DisCo and offshoots, this hasn't yet been demonstrated to be true. There's just no setting we know of, classicial or quantum, where deliberately introducing grammatical structure helps with any practical task, and shifting the goalposts to interpretability or whatever else has not worked either. So it isn't for lack of technical ability or imagination, it just appears to be that anything you could have done with "structure" or "composition" you could have also done without. This isn't to say that the structural view is totally without merit; it is certainly more human-friendly and aids in Interpretability writ large as subsuming pedagogy and communication, it's just impractical. This was surprising and disappointing for me, because of a more deep-seated belief in myself I have had to kill, that \emph{structure is magic}. If you are a computational linguist, I welcome you to try synthesising your structural formalisms with vectorial semantics across similar bridges as are built here, and I would be happy to be proved wrong about the importance and power of structure: the disenchanted view that I currently hold is that adding structure is in general a way to get computationally cheaper but worse answers.

There is a second, unspoken implication that is more seductive; that there exists some deep and fundamental unity between quantum theory and natural language. This is a rather commonplace sin more generally, that in some field XYZ with relatively little mathematical sophistication someone will steal the valour of physics by squatting on "Quantum XYZ", or "Quantum-inspired XYZ", when all they really mean is e.g. the use of noncommuting operators or tensor products or density matrices or some other narrow mathematical facet of quantum theory. Such views by themselves may be harmless, but in conjunction with the vaguely held but common assumption of mathematical realism [CITE], we find ourselves in trouble. The extent to which there are quantum theories of linguistics or cognitive science or anything else deep and multifaceted is that there are mathematical paintbrushes that have been used to illustrate quantum theory with which we can also better sketch and appreciate certain limited phenomena in other domains. However, even this honest appraisal is co-opted by a blameless form of motte-and-bailey, where a serious faction will disavow mysticism, but the field as a whole survives by luring na\"{i}ve researchers with the seductive implication.

These objections would usually be fatal, but as it goes, the alternatives are no better; sometimes worse. Any defensive manoeuvre that works to justify formal linguistics as an academic activity or otherwise will provide cover for quantum linguistics too, with little modification.

\subsection{On Deep Structure, the Universal Base Hypothesis, and the "Lexical Objection"}

\subsection{On communication, and the mathematical infeasibility of the Autonomy of Syntax}

\subsection{On frameworks for rewriting systems}

\subsection{On formality in cognitive semantics}

