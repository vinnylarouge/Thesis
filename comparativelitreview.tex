
\section{The other literature review}

\subsection{Introductory remarks}

This section was written after I passed my viva, and its purpose is to situate this work with respect to the relevant literature in linguistics, for the benefit of linguists. It was formally optional for me to do so, but I felt I ought to out of a sense of scholarly conscientiousness, and because my examiners told me I should. These emotional forces have only mildly won out against the alternative of doing nothing, so I will write casually, as I would for myself. At any rate, having now read some linguistics, I suspect that linguists do not mind reading and writing wordy things, and only linguists would stick around to read this, so there will be little harm in me writing indulgently.\\

One way to summarise things is that the technical contents of this thesis point towards a nearby counterfactual history where all the linguists in the "Garden of Eden period" [Partee?] knew some of the modern structuralist mathematics that their programme obviously would have profited from. The fractiousness of linguists notwithstanding, it is my opinion that the interdisciplinarity of this thesis is accidental: from the perspective of the counterfactual history, the division of formal approaches to syntax and semantics (and some of the subdivisions thereof) would appear contrived. I am aware that this could come across as pretty arrogant and patronising talk, which is not my intention this time, because the point I am ultimately trying to make is that nothing in this thesis is particularly special, and could have been reinvented by anyone. Text circuits, and the contents of this thesis, owe no substantial intellectual debt to linguistics outside of perhaps Lambek and Firth, themselves far from the mainstream. This is because I prefer thinking about things myself as opposed to reading about what others have thought, and the relevant developments were due to collaborations with the similarly ignorant. Consequently, all of the parallels I am about to point out between the current stream of development and the main body of literature are independent rediscoveries. I take these parallels as indications of the "naturalness" of ideas that could have come about anytime and independently of specific individuals, and accordingly as evidence for the "nearbyness" of the counterfactual history I am trying to gesture at.

\subsection{Pregroups to Text Circuits vs. Transformational Grammar to Formal Semantics}

The cartoon version of the literature review is folkloric, autochtonal, and maybe a little jingoistic: it is a caricature or myth of the field that people of the field tell about themselves. I think that is sufficient for most readers, but if you are here, I owe you a comparative retelling.

\subsection{On communication, and the mathematical infeasibility of the Autonomy of Syntax}

\subsection{On rewriting systems}

\subsection{On cognitive semantics}

\subsection{On meaning and meaninglessness}

The version of the thesis my examiners passed began with a remark to the effect that I share an interest with (certain) linguists in language as a (potentially) formal object, but I would not hold myself hostage to their literature, nor their methods. My examiners correctly deduced that this dismissive attitude was in part a cover to avoid engaging with the literature in linguistics altogether, which led to a brief sidebar on whether there are kinds of mathematicisation that qualify as intellectual colonialism. The consensus was yes, and on that note, I do share some methods in common with the linguists at MIT between \emph{Syntactic Structures} and \emph{Aspects of the Theory of Syntax}, as well as the stereotypical physicist encountering new fields with fresh eyes [XKCD] -- namely, armchair introspection and its sidekick, synoptic and reductive mathematical "toy-modelling" in search of simplicity. I gave two other reasons to my examiners for the lack of engagement, but my arguments that I shouldn't be obliged to cite unreadable and useless material because I view myself as a kind of above-it-all self-loathing artist-cum-diagram-arms-dealer was dismissed by my examiners, who pointed out my obligations as a scholar. Now that I have satisfied my scholarly obligations, I would like to go over my reasons.\\

Firstly, I don't respect the methods of formal linguistics. The extent of my prior exposure to formal linguistics is reading Heim \& Kratzer. I have a copy signed by Angelika, with an inscription "I hope it wasn't pure torture." It was unreadable because rather good and simple ideas were obfuscated in the mathematical equivalent of machine code. The first half of the book up to quantifiers can be summarised as a cartesian closed functor [Masters], and here is a fresh and morally lossless categorical exegesis of the technical contents of the whole book: Define \emph{"generative grammar"} to be a coloured operad, define semantic data to be a coloured operad along with an algebra valued in the homsets of cartesian closed category generated by types $e$ and $t$, and define \emph{"semantics for generative grammar"} to be the obvious operad map. I am even being generous, by already generalising in that description beyond truth-conditional semantics for natural language, which have been untenable since Putnam's permutation argument was fixed by Barr [Putnam, Barr]. Since I had it on good authority that Heim \& Kratzer was a gold standard, I made the informed decision to engage with all formal linguistics as little as possible to avoid mental pollution. I believe that I have adequately illustrated the possibilities of using diagrammatic mathematics in formal linguistics, and that my disdain for the extant formal alternatives is valid.\\

Secondly, I think the field as a whole is doomed anyway, at least spiritually. I think it is a deep kind of irony that the field that nominally concerns itself with meanings is suffering from meaninglessness, and I'm fairly certain it isn't just projection on my part. Insofar as:
\begin{enumerate}
\item large language models are \emph{the} hard technology for language
\item hard sciences are so by virtue of hard technologies
\item formal linguistics was destined to be a hard science
\item large language models have nothing to do with formal linguistics
\end{enumerate}
The whole thing has been a depressing failure of existential proportions. As far as I can tell, nobody contests point 4; in fact, the linguists I have spoken with have taken pains to stress this point [????]. Point 3 is just too embarrassing to admit out loud, but it is difficult to articulate why that is so without the whole complex. Point 2 is an unspoken and fairly modern cultural convention; for the vast majority of human history, craft has been ahead of science [Architecture]. Hence, the strategy is to deny point 1, which is why any self-respecting computational linguist will hasten to point out the small handful of pet examples where LLMs fail, if they aren't busy working with LLMs already.

As soon as mathematics became involved, the aim became procedural knowledge of language, and the grail was a language computer. It is impossible to take any other stance seriously: who would dare admit that the true aim was to establish an eternal monastic tradition of artisinal Markerese? As I have also indicated in the introduction, the relationship of theory to empirical capture is complicated today, because while the gradual accommodation of empirical observation is how a theory grows, the observed conclusion-at-scale of star-unstar is a theory-free big-data situation [BigData]. 



The only value system in which I see the endeavour of formalising the structure of language as a worthwhile activity is some kind of authentic post-postmodern neoromanticism that promotes a sort of aesthetic, productive, and communicative diversity of formalisms and methods of inquiry rather than canonical or universal structures, but on this view I take the vast majority of activity within any given scholarly institution as seriously as, say, Mario 64 speedrunning [A-button]. 