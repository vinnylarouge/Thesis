\section{Interpreting text circuits in \textbf{ContRel}}\label{subsec:toptext}

In this sketch I will tie up some remaining loose ends, by outlining how the iconic semantics interpretation of text circuits in \textbf{ContRel} is "natural", in the sense that we may relate the interpretation to some linguistic phenomena without much effort. To recap, we have seen already in previous sketches how sticky spiders yield static arrangements of labelled shapes in space, how homotopies may be used in conjunction with temporal algebras to reason about evolving configurations, how to deal with modal verbs and sentential complementation and how configuration spaces of nice spiders let us reason about all possible arrangements of the same shapes.

\subsection{Open sets: concepts}

Apart from enabling us to paint pictures with words, \textbf{ContRel} is worth the trouble because the opens of topological spaces crudely model how we talk about concepts, and the points of a topological space crudely model instances of concepts. We consider these open-set tests to correspond to "concepts", such as redness or quickness of motion. Figure \ref{fig:pointing} generalises to a sketch argument that insofar as we conceive of concepts in (possibly abstractly) spatial terms, the meanings of words are modellable as shared strategies for spatial deixis; absolute precision is communicatively impossible, and the next best thing mathematically requires topology.

\begin{figure}[h!]\label{fig:pointing}
\[\resizebox{\textwidth}{!}{\tikzfig{topology/pointingfinger}}\]
\caption{Points in space are a useful mathematical fiction. Suppose we have a point on a unit interval. Consider how we might tell someone else about where this point is. We could point at it with a pudgy appendage, or the tip of a pencil, or give some finite decimal approximation. But in each case we are only speaking of a vicinity, a neighbourhood, an \emph{open set in the borel basis of the reals} that contains the point. Identifying a true point on a real line requires an infinite intersection of open balls of decreasing radius; an infinite process of pointing again and again, which nobody has the time to do. In the same way, most language outside of mathematics is only capable of offering successively finer, finite approximations.}
\end{figure}

Maybe this explains the asymmetry of why tests are open sets, but why are states allowed to be arbitrary subsets? One could argue that states in this model represent what is conceived or perceived. Suppose we have an analog photograph whether in hand or in mind, and we want to remark on a particular shade of red in some uniform patch of the photograph. As in the case of pointing out a point on the real interval, we have successively finer approximations with a vocabulary of concepts: "red", "burgundy", "hex code \#800021"... but never the point in colourspace itself. If someone takes our linguistic description of the colour and tries to reproduce it, they will be off in a manner that we can in principle detect, cognize, and correct: "make it a little darker" or "add a little blue to it". That is to say, there are in principle differences in mind that we cannot distinguish linguistically in a finite manner; we would have to continue the process of "even darker" and "add a bit less blue than last time" forever. All this is just the mathematical formulation of a very common observation: sometimes you cannot do an experience justice with words, and you eventually give up with "I guess you just had to be there". Yet the experience is there and we can perform linguistic operations on it, and the states accommodate this.

\subsection{Copy: stative verbs and adjectives}

\emph{Stative} verbs are those that posit an unchanging state of affairs, such as \texttt{Bob \underline{likes} drinking}. Insofar as stative verbs are restrictions of all possible configurations to a permissible subset, they are conceptually similar to adjectives, such as \texttt{\underline{red} car}, which restricts permissible representations in colourspace. When we interpret concepts as open-set tests, \textbf{ContRel} conspires in our favour by giving us free copy maps on every wire. This allows us to define a family of processes that behave like stative restrictions of possibilities.

\begin{figure}[h!]
\[\resizebox{\textwidth}{!}{\tikzfig{topology/copygatecommute}}\]
\caption{
\begin{example}[Adjectives by analysis of configuration spaces]
The desirable property we obtain is that in the absence of \emph{dynamic} verbs that posit a change in the state of affairs, stative constructions commute in text: if I'm just telling you static properties of the way things are, it doesn't matter in what order I tell you the facts because restrictions commute. Recall that gates of the following form are intersections with respect to open sets, and they commute. These intersections model conjunctive specifications of properties.
\end{example}
}
\end{figure}

\begin{figure}[h!]
\[\resizebox{0.8\textwidth}{!}{\tikzfig{topology/oxfordex}}\]
\caption{
Consider the configuration space of a sticky spider on the unit square with three labelled shapes, which has 6 connected components, depicted. \texttt{Oxford contains Catz.} restricts away configurations where \texttt{Catz} is not enclosed in \texttt{Oxford}. Adding on \texttt{England contains Oxford.} further restricts away incongruent configurations, leaving us only with a single connected component, which contains all spatial configurations that satisfy the text. A similar story holds for abstract conceptual spaces, in which \texttt{fast red car}, \texttt{fast car that is red}, \texttt{car is (red and fast)} all mean the same thing.
}
\end{figure}

\subsection{Coclosure: adverbs and adpositions}

\begin{figure}[h!]
\[\scalebox{0.9}{\tikzfig{topology/adverbex1}}\]
\caption{
Recall that \textbf{ContRel} is coclosed (Proposition \ref{prop:coclosure}), which means that every dynamic verb may be expressed as the composite of a coevaluator and an open set on the space of homotopies. For instance, \texttt{move} is an intransitive dynamic verb, which corresponds to a concept in the space of all movements.
}
\end{figure}

\begin{figure}[h!]
\[\scalebox{0.9}{\tikzfig{topology/adverbex2}}\]
\caption{
Adverb-boxes may be modelled as static restrictions in movement-space. For instance, \texttt{straight} may restrict movements to just those that satisfy some notion of path-length minimality: e.g., given a metric in movement-space on path-lengths, we may construct an open ball (Definition \ref{def:openball}) around the geodesic to model the adverb \texttt{straight}.
}
\end{figure}

\begin{figure}[h!]
\[\scalebox{0.9}{\tikzfig{topology/adverbex3}}\]
\caption{
Similarly, adposition-boxes may be modelled as static restrictions on the product of the spaces of nouns and verbs. For instance, \texttt{towards} may be modelled as an open set that pairs potential positions of the thing-being-moved-towards with movements in movement-space that indeed move towards the target.
}
\end{figure}

\subsection{Homotopies: dynamic verbs and weak interchange}

\begin{figure}[h!]
\[\resizebox{\textwidth}{!}{\tikzfig{topology/collisionparticular}}\]
\caption{
\emph{Dynamic} verbs are those that posit a change in state, such as \texttt{Bob \underline{goes} home}. We want to model these verbs by homotopies, where the unit interval parameter models time. A nice and diagrammatically immediate property is that dynamic verbs are obstacles to the commutation of stative words.
\begin{example}[Dynamic verbs block stative commutations]
States of affairs can change after motion. A simple example is the case of \emph{collision}, where two shapes start off not touching, and then they move rigidly towards one another to end up touching.
\end{example}
}
\end{figure}

\begin{figure}[h!]
\[\resizebox{\textwidth}{!}{\tikzfig{topology/collideinterior}}\]
\caption{
Recalling that homotopies between relations are the unions of homotopies between maps, we have a homotopy that is the union of all collision trajectories, which we mark $\textcolor{orange}{\forall}$. We my define the interior $i(\textcolor{orange}{\forall})$ as the concept of collision; the expressible collection of all particular collisions. But this is not just an open set on the potential configuration of shapes, it is a collection of open sets parameterised by homotopy.
}
\end{figure}

\begin{figure}[h!]
\[\resizebox{\textwidth}{!}{\tikzfig{topology/collisionex}}\]
\caption{\texttt{Collision} blocks the commutation of the statives \texttt{touching} and \texttt{not touching}. In prose, the text \texttt{Alice and Bob are not touching. Alice and Bob collide. Alice and Bob are touching.} is not equal to \texttt{Alice and Bob are (touching and not touching). Alice and Bob collide.}; the latter is an incongruent state of affairs, which is reflected by an empty set of potential models (when \texttt{touching} and \texttt{not touching} intersect.)}
\end{figure}

\begin{figure}
\[\resizebox{\textwidth}{!}{\tikzfig{topology/seqparhom}}\]
\caption{
We can compose multiple motions in parallel by copying the unit interval, allowing it to parameterise multiple gates simultaneously, or compose them sequentially. The sequential composition of dynamic verbs in time explains the weak-interchange subtlety of text diagrams; there is now a diagrammatic distinction between events happening sequentially and in parallel. So now we have noncommuting gates that model \emph{actions}, or verbs.
}
\end{figure}
