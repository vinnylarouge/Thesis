\section{Composition of dynamic verbs via temporal anaphora}
\marginnote{\textcolor{blue}{Postscript: These sketches are mostly a restructuring of content that otherwise dangled from the previous chapter. Dynamic verbs and modals are two new sketches I had in mind while initially writing the thesis but didn't make it to the submitted version. There will probably be technical errors, but the sketches are not intended to be rigorous. None of these sketches (and nothing else in this thesis for that matter) should be taken as canonical once-and-for-all solutions to the conceptual problems they are meant to tackle; they are more meant to provoke as first-pass attempts, and they are meant to demonstrate how to play around and have fun in \textbf{ContRel} with string diagrams. I'll also note here that everything in \textbf{ContRel} is a kind of truth-conditional possible worlds semantics (up to some arbitrary but fixed choice of what particular ensembles of shapes and movements the modeller supplies up front), so there are no guarantees about how any of this material would fare if one tried to take the diagrams and interpret them in terms of neural networks, and I make no claims about whether the mathematics reflects actual cognition. However, I will claim that these mathematical sketches reflect at least the phenomenology of how \emph{I} think about language, which should come as no surprise because my methodology was armchair introspection.}}
Dynamic verbs in iconic semantics may be modelled by homotopies, but non-parallel composition of homotopies is only defined up to parameters with indications of how the two separate homotopies begin and end relative to one another; i.e. temporal data.

\begin{example}[Gluing homotopies sequentially at a time $\gamma \in (0,1)$] Given two homotopies $f: X \times [0,1] \rightarrow Y$ and $g: Y \times [0,1] \rightarrow Z$, we may define their composite along $Y$ with respect to $\gamma \in (0,1)$ by compressing $f$ to occur within $[0,\gamma]$ while holding $g$ fixed at time $0$, followed by compressing $g$ to occur within time $[\gamma,1]$ while holding $f$ fixed at time $1$.
\[f ;_{\gamma} g (x,t) := \begin{cases}
g(f(x,\frac{t}{\gamma}),0) & \text{if } t < \gamma\\
g(f(x,1),0)& \text{if } t = \gamma\\
g(f(x,1),\frac{t}{\gamma}-1)& \text{if } t > \gamma\\
\end{cases} : X \times [0,1] \rightarrow Z\]
In this case, composition asks for one free parameter $\gamma$, but it is easy to see that we may ask for more, corresponding to the free parameters of gaps, overlaps, and so on.
\end{example}

The technical difficulty I'd like to sketch a solution for is that while these parameters must be given as real numbers in the interval $[0,1]$, temporal natural language underspecifies: e.g. in the utterance \texttt{Bob drank, and then he slept} he could have drank in the morning and then slept in the afternoon, or both in the evening, and so on. The easy solution is to have absolute temporal anchors, but we seem to get by with less, which appears to necessitate a possible-worlds approach. Arguably the theoretical minimum we require is a kind of algebra for temporal aspects as in the Yucatec Maya language \citep{bohnemeyerTemporalAnaphoraTenseless2009a}, so here I sketch an algebra for temporal anaphora in \textbf{ContRel} that only requires copy-delete along with the standard topology on $\mathbb{R}$ obtained by the encoding of intervals as the open set $<: [0,1] \times [0,1]$. Then I'll show how this temporal data can be used to supply the information required for homotopy composition, which should indicate that \textbf{ContRel} is in-principle sufficiently expressive for dynamic iconic semantics for natural language, i.e. the interpretation of text as little moving cartoons.

\begin{defn}[A sketch text-circuit algebra for temporal anaphora]
We consider three kinds of events. The first is episodic, which corresponds to some interval on $[0,1]$ with endpoints $t_{\texttt{EV}}^0$ and $t_{\texttt{EV}}^1$. We model these as bipartite states with the initial constraint that $t_{\texttt{EV}}^0 < t_{\texttt{EV}}^1$. The second is habitual, which could in principle be an arbitrary subset of $[0,1]$, but there are pathologies we would like to rule out as a matter of common sense (e.g. we don't really talk about events that occur in time according cantor set), so we treat habituals as open sets (unions of intervals) to be later constructed or supplied as constraints; when we are finished specifying the algebra, equipping it with unions as a kind of formal sum will approximate those open sets that are constructible by finite amounts of talking about times. The third is a hybrid of the first two, where we consider some open set with distinguished endpoints, modelled as a restriction/intersection of an interval with some other open set.
\[\resizebox{\textwidth}{!}{\tikzfig{time/episodic}}\]
Now we model temporal aspects as circuit components --- what appears to distinguish aspects from tenses is that aspects are always relative to the temporal data of two events, whereas tenses may be "intransitive" on events --- so all of our aspectual data will involve constraining pairs of events (one of which is a \texttt{TOPIC}). The first kind of aspect we consider is \emph{perfective}, which constraints an event time to be within topic time; we model this as imposing a constraint that the endpoints of the event must lie within the interval specified by the endpoints of the topic. In discourse, introducing a perfective constraint corresponds to adding a gate.
\[\tikzfig{time/perf}\]
The \emph{terminative} aspect constrains an event to occur entirely before the beginning of the topic time. Terminative composition of verbs may be glossed as \texttt{(event) and-then (topic)}, and this kind of composition yields the view of text circuits as implicitly encoding the temporal order in which gate-as-events occur, where now the sequential ordering of gates matters. This failure of interchange interprets text circuits in something like a premonoidal setting \citep{jeffreyPremonoidalCategoriesFlow1998,romanStringDiagramsPremonoidal2024}.
\[\tikzfig{time/terminative}\]
The \emph{imperfective} aspect we consider as constraining an episodic topic time to lie within some ongoing habitual event, where the habitual event is represented as a free coparameter. In discourse, introducing an imperfective constraint corresponds to splicing in such a constraint, which we gloss as a gate that restricts the endpoints of the topic interval to lie within the open set representing the habitual event time as a coparameter. We skip over the subtly distinct \emph{progressive} aspect here as we won't need it for our later example, but it should be clear that an approach along these lines will also suffice.
\[\resizebox{\textwidth}{!}{\tikzfig{time/imperf}}\]
\end{defn}

\begin{example}
So here is an example of Yucatan Maya taken from\citep{nativlangMayaMayaHow2019}, which is an excerpt of an interview with a speaker fleeing a cyclone. I have split the excerpt into numbered single-verb clauses, accompanied by glosses in English with aspect-markers and the corresponding evolution of a text-circuit by the discourse rewrites we have defined. The first event introduced into discourse is the arrival of the refugees in the village, which is marked as perfective.
\[(1) \quad \left[ \begin{array}{c}\texttt{\emph{Kaajk'ucho'on} t\'{u}un way tek\`{a}ajil x Jaxleyile,} \\ \hline \texttt{When we }\underbrace{\texttt{\emph{arrived}}}_{\textsc{\makebox[0pt]{PERF.}}}\texttt{,}\end{array} \right] \vcenter{\scalebox{0.75}{\tikzfig{time/ex0}}}\]
The second event is what the refugees saw, implicitly concurrent with event (1), which we opt to treat with a prepended copy of endpoints. \texttt{arrive \& see} then form an atomic topic for events (3) and (4), which we deal with by constraining both (1) and (2) in the same way. Note that there is a single variable open set $t_\texttt{say}$ that is repeated 4 times in the diagram.
\[(2) \quad \left[ \begin{array}{c} \texttt{\emph{kilike'} tul\'{a}akal m\'{a}ake',} \\ \hline \texttt{we } \underbrace{\texttt{\emph{saw}}}_{\textsc{\makebox[0pt]{PERF.}}}\texttt{,} \end{array} \right] \quad \vcenter{\scalebox{0.75}{\tikzfig{time/ex1}}}\]
The third event refers to the villagers saying something, in the imperfective aspect with respect to events (1) and (2), so we constrain those topics accordingly. In gloss, it was an ongoing event that the villagers were saying something when the refugees arrived.
\[(3) \quad \left[ \begin{array}{c} \texttt{\emph{t\'{a}an uya'aliko'obe'} jach} \\ \hline \texttt{everyone }\underbrace{\texttt{\emph{was saying}}}_{\textsc{\makebox[0pt]{IMPF. wrt. (1,2)}}}\end{array} \right] \quad \vcenter{\scalebox{0.75}{\tikzfig{time/ex2}}}\]
The fourth event refers to what the villagers had heard, in the terminative aspect with respect to (1) and (2). In gloss, the villagers were saying (reporting) the episodic event of them hearing something on the radio, and this hearing-event had completed before the refugees' arrival.
\[(4) \quad \left[ \begin{array}{c} \texttt{\emph{ts'uyu'ubiko'ob ti'} r\`{a}adyoe'} \\ \hline \texttt{(they) }\underbrace{\texttt{\emph{had heard }}}_{\textsc{\makebox[0pt]{TERM. wrt. (1,2)}}}\texttt{on the radio} \end{array} \right] \quad \vcenter{\scalebox{0.6}{\tikzfig{time/ex3}}}\]
The fifth event refers the coming of the cyclone, which was ongoing at the time of the villagers hearing the radio report. This introduces a new habitual event as the variable open set $t_\texttt{come}$, repeated twice in the diagram as constraints.
\[(5) \quad \left[ \begin{array}{c} \texttt{\emph{t\'{u}un t\`{a}al} le sikl\`{o}ono'.} \\ \hline \texttt{the cyclone }\underbrace{\texttt{\emph{was coming.}}}_{\textsc{\makebox[0pt]{(4) IMPF. wrt. (5)}}} \end{array} \right] \quad \vcenter{\scalebox{0.6}{\tikzfig{time/ex4}}}\]
Altogether, the final diagram represents a map from two open sets on $[0,1]$ (representing the potentially habitual events \texttt{say} and \texttt{come} encoded as variable open sets $t_\texttt{say}$ and $t_\texttt{come}$) to return a state in \textbf{ContRel} that encodes the set of possible endpoints for the episodic events \texttt{arrive}, \texttt{see} and \texttt{hear}: $\{(t_\texttt{arrive}^0,t_\texttt{arrive}^1,t_\texttt{see}^0,t_\texttt{see}^1,t_\texttt{hear}^0,t_\texttt{hear}^1)\}$. Moreover, we have set up the algebra to allow us to leverage compositional discourse structure in such a way that sampling any of the elements of the resultant set returns a choice of endpoints consistent with the temporal constraints of the excerpt.
\end{example}