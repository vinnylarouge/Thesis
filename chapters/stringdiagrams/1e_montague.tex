\section{A modern mathematician's companion to Montague's "Universal Grammar"}

\newthought{Summary:} To do "Montague semantics" means taking structure-respecting homomorphisms from grammar to meaning \bR CITE \e. Montague considered grammars to be coloured operads; Montague's "algebras" are (multi-sorted) clones, which are in bijection with (multi-sorted) Lawvere Theories, which are equivalently coloured operads, which are special cases of PROPs for symmetric monoidal categories. Hence text circuits share a mathematical lineage with many other mathematical conceptions of grammar, while also enjoying Montague semantics.

\subsection{What did Montague consider grammar to be?}\label{sec:monty}

\newthought{Montague semantics/grammar as Montague envisioned it is largely contained in two papers} -- \emph{Universal Grammar} \citep{montague1970universal}, and \emph{The Proper Treatment of Quantifiers in English} \citep{montague1973proper} -- both written shortly before his murder in 1971. The methods employed were not mathematically novel -- the lambda calculus had been around since \bR DATE \e, and Tarski and Carnap had been developing intensional higher-order logics since \bR DATE \e -- but for linguists who, by-and-large, only knew first order predicate logic, these methods were a tour-de-force that solved longstanding problems in formal semantics.

\newthought{There is a natural division of Montague's approach into two structural components.} According to Partee \bR CITE \e, a formal semanticist, advocate, and torch-bearer for Montague, the chief interest of Montague's approach (as far as his contemporary linguists were concerned) lay in the following ideas:

\begin{enumerate}
\item{Take truth conditions to be the essential data of semantics.}
\item{
\begin{enumerate}
\item{Use lambdas to emulate the structure of syntax...}
\item{...in a typed system of intensional predicate logic, such that composition is function application.}
\end{enumerate}}
\end{enumerate}

More precisely, Montague devised a higher-order intensional logic for the first point, and the notion of a structure-preserving map from syntax to semantics for the second. The truth-conditional perspective was important at the time for enabling semantic computation, but within formal semantics there arose other perspectives on the nature of formal semantics, such as inquisitive \bR CITE \e and update semantics \bR CITE \e. Today, the empirical evidence we have from vector-based methods in computational linguistics is that none of those conceptions of semantics are intrinsically interesting or canonical: certainly none are procedurally necessary for a broad conception of practicality. So let's nevermind points 1 and 2b.\\

I have split the second point to highlight the role of lambdas. This element was the crux of the Montagovian revolution: according to Janssen in a personal communication with Partee from 1994, lambdas were "...the feature that made compositionality possible at all." Using lambdas to make the semantic domain compositional then gave a target for the structure-preserving homomorphism from the syntactic domain. Today, we have more refined ways to grant structure to semantic domains using category-theoretic tools. So let's redact "lambdas" from 2a.\\

What remains that is of interest is the question of what Montague considered the structure of syntax to be. This is worth understanding, since we claim text circuits are a "structure of syntax", and that functorial interpretation of text circuits in symmetric monoidal categories is Montagovian semantics in spirit if not in letter. So let's begin. In Section 1 of \emph{Universal Grammar}, Montague's first paragraph establishes common notions of relation and function -- the latter he calls \emph{operation}, to distinguish the $n$-ary case from the unary case which he calls \emph{function}. This is all done with ordinals indexing lists of elements of an arbitrary but fixed set $A$, which leads later on to nested indices and redundancy by repeated mention of $A$. We will try to avoid these issues going forward by eliding some data where there is no confusion, following common modern practice.\\

Next, Montague introduces his notion of \emph{algebra} and \emph{homomorphism}. He separates the data of the carrier set and the generators from the \emph{polynomial operations} that generate the term algebra.

\begin{defn}[Generating data of an Algebra]\label{algdata} 
Let $A$ be the carrier set, and $F_\gamma$ be a set of functions $A^k \rightarrow A$ for some $k \in \mathbb{N}$, indexed by $\gamma \in \Gamma$. Denoted $\langle A, F_\gamma \rangle_{\gamma \in \Gamma}$
\end{defn}

The following three data are taken to be common among Montague's algebras:

\begin{defn}[Identities]\label{ids} 
A family of operations populated, for all $n, m \in \mathbb{N}$, $n \leq m$, by an $m$-ary operation $I_{n,m}$, defined on all $m$-tuples as

$$I_{n,m}(a) = a_n$$

where $a_n$ is the $n^{\text{th}}$ entry of the $m$-tuple $a$.
\end{defn}


\begin{defn}[Constants]\label{constants}
For all elements of the carrier $x \in A$, and all $m \in \mathbb{N}$, a constant operation $C_{x,m}$ defined on all $m$-tuples $a$ as:
$$C_{x,m}(a) = x$$
\end{defn}

\begin{defn}[Composition]\label{comp}
Given an $n$-ary operation $G$, and $n$ instances of $m$-ary operations $H_{1 \leq i \leq n}$, define the composite $G(H_i)_{1 \leq i \leq n}$ to act on $m$-tuples $a$ by:

$$G(H_i)_{1 \leq i \leq n}(a) = G(H_i(a))_{1 \leq i \leq n}$$

\emph{N.B.} the $m$-tuple $a$ is copied $n$ times by the composition. Writing out the right hand side more explicitly:

$$G\bigg( \ \big( \ H_1(a) \ , \ H_2(a) \ , \ \ldots \ , \ H_n(a) \ \big) \  \bigg)$$
\end{defn}

\begin{defn}[Polynomial Operations]\label{polyop}
The polynomial operations over an algebra $\langle A, F_\gamma \rangle_{\gamma \in \Gamma}$ are defined to be smallest class $K$ containing all $F_{\gamma \in \Gamma}$, identities, constants, closed under composition.
\end{defn}

\begin{defn}[Homomorphism of Algebras]\label{homo}
$h$ is a homomorphism from $\langle A, F_\gamma \rangle_{\gamma \in \Gamma}$ \emph{into} $\langle B, G_\gamma \rangle_{\gamma \in \Delta}$ iff
\begin{enumerate}
    \item{$\Gamma = \Delta$ and $\forall \gamma : $}
    \item{\bR FILL! \e}
\end{enumerate}
\end{defn}

Section 2 seeks to define a broad conception of `syntax', which he terms a \emph{disambiguated language}. This is a free clone with carrier set $A$, generating operations $F_\gamma$ indexed by $\gamma \in \Gamma$, along with extra decorating information:

\begin{enumerate}
\item{$(\delta \in) \Delta$ is an (indexing) set of syntactic categories (e.g.~\texttt{NP}, \texttt{V}, etc.). Montague calls this the \emph{set of category indices}. $X_\delta \subseteq A$ form the \emph{basic expressions} of type $\delta$ in the language.}
\item{a set $S$ assigns types among $\delta \in \Delta$ to the inputs and output of -- not necessarily all -- $F_\gamma$.}
\item{a special $\delta_0 \in \Delta$ is taken to be the type of declarative sentences.}
\end{enumerate}

This definition is already considerably progressive. Because there is no condition of disjointness upon the $X_\delta$ -- a view that permits consideration of the same word playing different syntactic roles -- (1) permits the same basic expression $x \in A$ to participate in multiple types $X_\delta \subseteq A$ ($\star$). (2) misses being a normal typing system on several counts. There is no condition requiring all $F_\gamma$ to be typed by $S$, and no condition restricting each $F_\gamma$ to appear at most once: this raises the possibilities that ($\dag$) some operations $F$ go untyped, or that ($\ddag$) some are typed multiply.\\

Taking a disambiguated language $\mathfrak{U}$ on a carrier set $A$, Montague defines a \emph{language} to be a pair $L := <\mathfrak{U}, R>$, where $R$ is a relation from a subset of the carrier $A$ to a set $\texttt{PE}_L$, the set of \emph{proper expressions} of the language $L$. An admirable purpose of $R$ appears to be to permit the modelling of \emph{syntactic ambiguity}, where multiple elements of the term algebra $\mathfrak{U}$ (corresponding to syntactic derivations) are related to the same "proper language expression".\\

However, we see aspects where Montague was born ahead of his time mathematically: it appears that his intent was to impose a system of types to constrain composition of operations, but the tools were not available for him. Montague addresses ($\dag$) obliquely, by defining $\texttt{ME}_L$ to be the image in $\texttt{PE}_L$ of $R$ of just those expressions among $A$ that are typed. Nothing appears to guard against ($\ddag$), which causes problems as Montague expresses structural constraints (in the modern view) in terms of constraints on the codomain of an interpreting functor (cf. Montague's notion of \emph{generating} syntactic categories). One consquence, in conjunction with $(\star)$, is that every multiply typed operation $F$ induces a boolean algebra where the typings are the generators and the operations are elementwise in the inputs and output. Worse problems occur, as Montague's clone definition include all projectors, and when defined separately from the typing structure, these projectors may be typed in a way that permits operations that arbitrarily change types, which appears to defeat the purpose. We doubt these artefacts are intentional, so we will interpret Montague assuming his intent was a type-system with categorical semantics as we would use today.

\subsection{On the historical inevitability of text diagrams}

Definition \ref{ids} is equivalent to asking for all projections. Definitions \ref{ids} and \ref{comp} together characterise Montagovian algebras as (concrete) clones \bR CITE \e, which then generalised to (abstract) clones \bR CITE \e, which were then discovered to be in bijection with Lawvere theories \bR CITE \e. By an evident extension of [Prop 3.51] to the typed case, a \emph{disambiguated language} is a multi-sorted Lawvere theory without relations, where the sorts are generated from products of a pointed set $(\Delta, \delta_0 : 1 \rightarrow \Delta)$.\\

Lawvere theories are themselves a special case of operadic composition \bR CITE \e. Operadic composition naturally as that of coloured trees \bR CITE \e, which is equivalently depicted as nesting expressions according to the tree-structure \bR CITE \e. Interpreting colours as types, operadic composition subsumes whatever one might wish to do with a typed gentzen-style sequent system where rules are multi-input single-output. While typelogical grammars stop there, PROPs generalise operadic composition to multi-input multi-output \bR CITE \e, and as combinatorial specifications for string diagrams, weak $n$-categories generalise PROPs, and we have shown weak $n$-categories to subsume a distinct evolutionary line of formal syntax from CFGs to TAGs.\\

In summary, text circuits share a common conceptual and mathematical ancestry with many approaches to formal syntax, hence one ought to expect deep compatibility.