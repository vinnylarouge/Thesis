\section{Populating space with shapes using sticky spiders}

\begin{figure}\label{fig:spiderbicate}
\scalebox{0.7}{\tikzfig{bestiary/spiderbicat}}
\caption{The generators (in dashed boxes) and relations that make a spider. When the spider satisfies in addition the three inequalities b1-3, we call it a \textbf{relation-spider}.}
\end{figure}

\subsection{When does an object have a spider (or something close to one)?}

\marginnote{
\begin{rem}[copy-compare spiders of $\mathbf{Rel}$]
For a set $X$, the \emph{copy} map $X \rightarrow X \times X$ is defined:
\[\{(x,(x,x)) : x \in X \}\]
the \emph{compare} map $X \times X \rightarrow X$ is defined:
\[\{((x,x),x) : x \in X \}\]
These two maps are the (co)multiplications of special frobenius algebras. The (co)units are \emph{delete}:
\[\{(x,\star) : x \in X\}\]
and \emph{everything}:
\[\{(\star,x) : x \in X\}\]
\end{rem}
}

\begin{example}[The copy-compare spiders of $\mathbf{Rel}$ are not always continuous]\label{ex:compnotspider}
The compare map for the Sierpi\'{n}ski space is not continuous, because the preimage of $\{0,1\}$ is $\{(0,0),(1,1)\}$, which is not open in the product space of $\mathcal{S}$ with itself.
\end{example}

\begin{proposition}\label{prop:copydiscrete}
The copy map is a spider iff the topology is discrete.
\begin{proof}
Discrete topologies inherit the usual copy-compare spiders from \textbf{Rel}, so we have to show that when the copy map is a spider, the underlying wire must have a discrete topology. Suppose that some wire has a spider, and construct the following open set using an arbitrary point $p$:
\[\scalebox{0.5}{\tikzfig{structure/copyspiderproof/openpoint}}\]
It will suffice to show that this open set is the singleton $\{p\}$ -- when all singletons are open, the topology is discrete. As a lemma, using frobenius rules and the property of zero morphisms, we can show that comparing distinct points $p \neq q$ yields the $\varnothing$ state.
\[\scalebox{0.5}{\tikzfig{structure/copyspiderproof/openpointproof}}\]
The following case analysis shows that our open set only contains the point $p$.
\[\scalebox{0.5}{\tikzfig{structure/copyspiderproof/openpointcases}}\]
\end{proof}
\end{proposition}

\marginnote{
    \begin{rem}[Split idempotents]
    An \textbf{idempotent} in a category is a map $e: A \rightarrow A$ such that \[A \overset{e}{\rightarrow} A \overset{e}{\rightarrow} A = A \overset{e}{\rightarrow} A\]
    A \textbf{split idempotent} is an idempotent $e: A \rightarrow A$ along with a \textbf{retract} $r: A \rightarrow B$ and a \textbf{section} $s: B \rightarrow A$ such that:
    \[A \overset{e}{\rightarrow} A = A \overset{r}{\rightarrow} B \overset{s}{\rightarrow} A\]
    \[B \overset{s}{\rightarrow} A \overset{r}{\rightarrow} B = B \overset{\mathop{id}}{\rightarrow} B\]
    \end{rem}
}

\begin{fullwidth}

\newthought{We can use split idempotents to transform copy-spiders from discrete topologies to almost-spiders on other spaces.} We can graphically express the behaviour of a split idempotent $e$ as follows, where the semicircles for the section and retract $r,s$ form a visual pun.

\[\scalebox{1}{\tikzfig{structure/idemspider/splitidem}}\]

\newthought{An idempotent on $Y^\sigma$ that splits through a discrete topology $X^\star$ does these things:}
\begin{description}
\item[The section:]{picks a subset of $s(x) \subseteq Y$ for each point $x \in X$. In order to be a split idempotent, $s(x)$ must be distinct for distinct points $x$.}
\item[The retract:]{(reading backwards) picks an open set $r(x) \in \sigma$ for each point $x \in X$. In order to be a split idempotent, $r(x)$ only overlaps $s(x)$ out of all other selected shapes: \[r(x) \cap s(x') = 1 \iff x = x'\]}
\end{description}
The combined effect is that the shapes $s(x)$ become copiable elements of the sticky spider, just as the points are the copiable elements of a regular spider.

\begin{defn}[Sticky spiders]
A \textbf{sticky spider} (or just an $e$-spider, if we know that $e$ is a split idempotent), is a spider \emph{except} every identity wire on any side of an equation in Figure \ref{fig:spiderbicate} is replaced by the idempotent $e$.
\[\scalebox{1}{\tikzfig{structure/idemspider/espiderproperties}}\]
\[\scalebox{1}{\tikzfig{structure/idemspider/espiderproperties2}}\]

The desired graphical behaviour of a sticky spider is that one can still coalesce all connected spider-bodies together, but the 1-1 spider "sticks around" rather than disappearing as the identity. This is achieved by the following rules that cohere the idempotent $e$ with the (co)unit and (co)multiplications.

\[placeholder\]
\end{defn}

\begin{construction}[Sticky spiders from split idempotents]
Given an idempotent $e: Y^\sigma \rightarrow Y^\sigma$ that splits through a discrete topology $X^\star$, we construct a new (co)multiplication as follows:
\[\scalebox{1}{\tikzfig{structure/idemspider/idemspiderv2}}\]
\end{construction}

\begin{proposition}[Every idempotent that splits through a discrete topology gives a sticky spider]\label{prop:splitmeanssticky}
\[\scalebox{1}{\tikzfig{structure/idemspider/espiderstatement}}\]
\begin{proof}
We can check that our construction satisfies the frobenius rules as follows. We only present one equality; the rest follow the same idea.
\[\scalebox{1}{\tikzfig{structure/idemspider/espiderproofv2}}\]
To verify the sticky spider rules, we first observe that since \[X^\star \overset{s}{\rightarrow} Y^\sigma \overset{r}{\rightarrow} X^\star = X^\star \overset{\mathop{id}}{\rightarrow} X^\star\]
$r$ must have all of $X^\star$ in its image, and $s$ must have all of $X^\star$ in its preimage, so we have the following:
\[\scalebox{1}{\tikzfig{structure/idemspider/splitonto}}\]
Now we show that e-unitality holds:
\[\scalebox{1}{\tikzfig{structure/idemspider/espiderproof2}}\]
The proofs of e-counitality, and e-speciality follow similarly.
\end{proof}
\end{proposition}

\newthought{We can prove a partial converse of Proposition \ref{prop:splitmeanssticky}:} we can identify two diagrammatic equations that tell us precisely when a sticky spider has an idempotent that splits though some discrete topology.

\begin{theorem}\label{thm:stickygraphical}
A sticky spider has an idempotent that splits through a discrete topology if and only if in addition to the sticky spider equalities, the following relations are also satisfied.
\[\scalebox{1}{\tikzfig{idemproof/unit-everything}} \quad\quad\quad\quad \scalebox{1}{\tikzfig{idemproof/comult-copy}}\]
\end{theorem}

\[\scalebox{0.75}{\tikzfig{idemproof/claimmap2}}\]

\begin{proposition}[comult/copy implies counit/delete]\label{prop:counitdelete}
\[\scalebox{1}{\tikzfig{idemproof/ecopy2delclaim}}\]
\begin{proof}
\[\scalebox{1}{\tikzfig{idemproof/ecopy2del}}\]
\end{proof}
\end{proposition}

\newpage
\begin{lemma}[All-or-Nothing]\label{lem:allornothing}
Consider the set $e(\{x\})$ obtained by applying the idempotent $e$ to a singleton $\{x\}$, and take an arbitrary element $y \in e(x)$ of this set. Then $e(\{y\}) = \varnothing$ or $e(\{x\}) = e(\{y\})$. Diagrammatically: \[\scalebox{0.75}{\tikzfig{idemproof/allornothingclaim}}\]
\begin{proof}
\[\scalebox{0.75}{\tikzfig{idemproof/allornothing2a}}\]
\[\scalebox{0.75}{\tikzfig{idemproof/allornothing2b}}\]
\end{proof}
\end{lemma}

\begin{proposition}[$e$ of any point is $e$-copiable]\label{prop:epointcopy}
\[\scalebox{0.75}{\tikzfig{idemproof/pointidemcopiable}}\]
\begin{proof}
\[\scalebox{0.75}{\tikzfig{idemproof/pointidemcopiableproof}}\]
\end{proof}
\end{proposition}

\begin{proposition}[The unit is the union of all $e$-copiables]\label{prop:copiablebasis}
\[\scalebox{1}{\tikzfig{idemproof/copiablebasisclaim}}\]
\begin{proof}
\[\scalebox{1}{\tikzfig{idemproof/copiablebasis}}\]
\end{proof}
\end{proposition}

\begin{proposition}[$e$-copiable decomposition of $e$]\label{prop:decompidem}
\[\scalebox{1}{\tikzfig{idemproof/decompidemclaim}}\]
\begin{proof}
\[\scalebox{1}{\tikzfig{idemproof/decompidem}}\]
\end{proof}
\end{proposition}

\begin{proposition}[$e$-copiable decomposition of counit]\label{prop:decompcounit}
\[\scalebox{1}{\tikzfig{idemproof/decompcounitclaim}}\]
\begin{proof}
\[\scalebox{1}{\tikzfig{idemproof/decompcounit}}\]
\end{proof}
\end{proposition}

\newthought{The $e$-copiable states really do behave like an orthonormal basis, as the following Lemmas show.}

\begin{lemma}[$e$-copiables are orthogonal under multiplication]\label{lem:match}
\[\scalebox{1}{\tikzfig{idemproof/matchclaim}}\]
\begin{proof}
\[\scalebox{0.9}{\tikzfig{idemproof/match}}\]
\end{proof}
\end{lemma}

\begin{convention}[Shorthand for the open set associated with an $e$-copiable]
We introduce the following diagrammatic shorthand.
\[\scalebox{1}{\tikzfig{idemproof/openshorthand}}\]
Including the coloured dot is justified, because these open sets are co-copiable with respect to the multiplication of the sticky spider.
\[\scalebox{1}{\tikzfig{idemproof/shorthandjustification}}\]
\end{convention}

\begin{lemma}[Co-match]\label{lem:comatch}
\[\scalebox{1}{\tikzfig{idemproof/comatchclaim}}\]
\begin{proof}
\[\scalebox{0.9}{\tikzfig{idemproof/comatch}}\]
The claim then follows by applying Lemma \ref{lem:match} to the final diagram.
\end{proof}
\end{lemma}

\begin{lemma}[e-copiables are e-fixpoints]\label{lem:ecopyfixpoint}
\[\scalebox{1}{\tikzfig{idemproof/dotcopyableclaim}}\]
\begin{proof}
\[\scalebox{1}{\tikzfig{idemproof/dotcopyable}}\]
Observe that the final equation of the proof also holds when the initial e-copiable is the empty set.
\end{proof}
\end{lemma}

\begin{lemma}[$e$-copiables are normal]\label{lem:ecopynormal}
\[\scalebox{1}{\tikzfig{idemproof/copynormalclaim}}\]
\begin{proof}
\[\scalebox{1}{\tikzfig{idemproof/copynormal}}\]
\end{proof}
\end{lemma}
{}
\begin{proposition}[$e$-copiable decomposition of multiplication]\label{prop:decompmult}
\[\scalebox{1}{\tikzfig{idemproof/decompmultclaim}}\]
\begin{proof}
\[\scalebox{1}{\tikzfig{idemproof/decompmult}}\]
\end{proof}
\end{proposition}

\begin{proposition}[$e$-copiable decomposition of comultiplication]\label{prop:decompcomult}
\[\scalebox{1}{\tikzfig{idemproof/decompcomultclaim}}\]
\begin{proof}
\[\scalebox{1}{\tikzfig{idemproof/decompcomult}}\]
\end{proof}
\end{proposition}

\newthought{Now we can prove Theorem \ref{thm:stickygraphical}.}
\begin{proof}
The key observation is that the $e$-copiable decomposition of the idempotent given by Proposition \ref{prop:decompidem} is equivalent to a split idempotent though the set of $e$-copiables equipped with discrete topology.
\[placeholder\]

\end{proof}

\end{fullwidth}

\section{Sticky spiders are collections of shapes in space}

...

\section{Displacing shapes with symmetries}

\begin{defn}[Topological Group]

\end{defn}

\begin{defn}[Sticky spiders equivalent up to symmetric displacement]

\end{defn}

\section{Moving shapes continuously}

\section{Configuration space of a sticky spider}

\begin{defn}[The configuration space of a sticky spider]

\end{defn}

\begin{defn}[The "interaction" relation]

\end{defn}

\begin{defn}[Movement equivalence classes]

\end{defn}

\section{Topological models for a selection of concepts}

\begin{construction}[Putting-in and getting-out things from containers]

\end{construction}

\begin{construction}[Filling holes, fullness, and fitting]

\end{construction}

\begin{construction}[Sending by throwing and conduits]

\end{construction}

\begin{construction}[Conduit Metaphor]

\end{construction}

\begin{fullwidth}