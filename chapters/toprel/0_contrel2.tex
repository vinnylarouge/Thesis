\section{Continuous Relations}

\newthought{To the best of my knowledge, the study of \textbf{ContRel} is a novel contribution. I venture two potential reasons.}

\newthought{First, it is because and not despite of the na\"{i}vity of the construction.} Usually, the relationship between \textbf{Rel} and \textbf{Set} is often understood in sophisticated general methods which are inappropriate in different ways. I have tried applying Kliesli machinery which generalises to "relationification" of arbitrary categories via appropriate analogs of the powerset monad to relate \textbf{Top} and \textbf{ContRel}, but it is not evident to me whether there is such a monad. The view of relations as spans of maps in the base category should work, since \textbf{Top} has pullbacks, but this makes calculation difficult and especially cumbersome when monoidal structure is involved. The na\"{i}ve approach I take is to observe that the preimages of functions are precisely relational converses when functions are viewed as relations, so the preimage-preserves-opens condition that defines continuous functions directly translates to the relational case.

\newthought{Second, the relational nature of \textbf{ContRel} means that the category has poor exactness properties.} Even if the sophisticated machinery mentioned in the first reason do manage to work, relational variants of \textbf{Top} are poor candidates for any kind of serious mathematics because they lack many limits and colimits. Since we take an entirely "monoidal" approach -- a relative newcomer in terms of mathematical technique -- we are able to find and make use of the rich structure of \textbf{ContRel} with a different toolkit.\\

In the end, we want to formalise doodles, so perhaps there is some virtue in proceeding by elementary means.

\marginnote{
\begin{rem}[Topological Space]
A \emph{topological space} is a pair $(X,\tau)$, where $X$ is a set, and $\tau \subset \mathcal{P}(X)$ are the \emph{open sets} of $X$, such that:
\begin{description}
    \item["nothing" and "everything" are open]  \[\varnothing,X \in \tau\]
    \item[Arbitrary unions of opens are open] \[\{ U_i : i \in I \} \subseteq \tau \Rightarrow \bigcup\limits_{i \in I} U_i \in \tau \]
    \item[Finite intersections of opens are open] $n \in \mathbb{N}$: \[U_1,\cdots, U_n \in \tau \Rightarrow \bigcap\limits_{1\cdots, i , \cdots n} U_i \in \tau\]
\end{description}
\end{rem}
}

\marginnote{
\begin{rem}[Relational Converse]
Recall that a relation $R: S \rightarrow T$ is a subset $R \subseteq S \times T$. \[R^\dag : T \rightarrow S := \{ (t,s) : (s,t) \in R \}\]
\end{rem}
}

\marginnote{
\begin{rem}[Continuous function]
A function between sets $f: X \rightarrow Y$ is a continuous function between topologies $f: (X,\tau) \rightarrow (Y,\sigma)$ if \[U \in \sigma \Rightarrow f^{-1}(U) \in \tau\] where $f^{-1}$ denotes the inverse image.
\end{rem}
}

Recall that functions are relations, and the inverse image used in the definition of continuous maps is equivalent to the relational converse when functions are viewed as relations. So we can na\"{i}vely extend the notion of continuous maps to continuous relations between topological spaces.

\begin{defn}[Continuous Relation]\label{defn:Contrelation}
A continuous relation $R: (X,\tau) \rightarrow (Y,\sigma)$ is a relation $R: X \rightarrow Y$ such that \[U \in \sigma \Rightarrow R^{\dag}(U) \in \tau\] where $\dag$ denotes the relational converse.
\end{defn}

\begin{notation}
For shorthand, we denote the topology $(X,\tau)$ as $X^{\tau}$. As special cases, we denote the discrete topology on $X$ as $X^{\star}$, and the indiscrete topology $X^{\circ}$.
\end{notation}

The symmetric monoidal structure is that of product topologies on objects, and products of relations on morphisms.

\marginnote{
\begin{rem}[Product Topology]
We denote the product topology of $X^\tau$ and $Y^\sigma$ as $(X \times Y)^{(\tau \times \sigma)}$. $\tau \times \sigma$ is the topology on $X \times Y$ generated by the basis $\{t \times s : t \in \mathfrak{b}_\tau, s \in \mathfrak{b}_\sigma\}$, where $\mathfrak{b}_\tau$ and $\mathfrak{b}_\sigma$ are bases for $\tau$ and $\sigma$ respectively.
\end{rem}
}

\marginnote{
    \begin{rem}[Product of relations]
    For relations between sets $R: X \rightarrow Y, S: A \rightarrow B$, the product relation $R \times S: X \times A \rightarrow Y \times B$ is defined to be \[ \{ ((x,a),(y,b)) : (x,y) \in R, (a,b) \in S \} \]
    \end{rem}
}

\subsection{The category \textbf{ContRel}}

\begin{proposition}[\textbf{ContRel} is a category]
continuous relations form a category $\mathbf{ContRel}$.
\begin{proof}
\newthought{Identities:} Identity relations, which are always continuous since the preimage of an open $U$ is itself.

\newthought{Composition:} The normal composition of relations. We verify that the composite $X^\tau \overset{R}{\rightarrow} Y^\sigma \overset{S}{\rightarrow} Z^\theta$ of continuous relations is again continuous as follows:
\[U \in \theta \implies S^\dag(U) \in \sigma \implies R^\dag \circ S^\dag(U) = (S \circ R)^\dag \in \tau\]

\newthought{Associativity of composition:} Inherited from \textbf{Rel}.
\end{proof}
\end{proposition}

\subsection{Symmetric Monoidal structure}

\begin{proposition}
$(\mathbf{ContRel},\bullet,X^\tau \otimes Y^\sigma := (X \times Y)^{(\tau \times \sigma)})$ is a symmetric monoidal closed category.
\begin{proof}

\newthought{Tensor Unit:} The one-point space $\bullet$. Explicitly, $\{\star\}$ with topology $\{\varnothing,\{\star\}\}$.

\newthought{Tensor Product:} For objects, $X^\tau \otimes Y^\sigma$ has base set $X \times Y$ equipped with the product topology $\tau \times \sigma$. For morphisms, $R \otimes S$ the product of relations. We show that the tensor of continuous relations is again a continuous relation. Take continuous relations $R: X^\tau \rightarrow Y^\sigma$, $S: A^\alpha \rightarrow B^\beta$, and let $U$ be open in the product topology $(\sigma \times \beta)$. We need to prove that $(R \times S)^\dag(U) \in (\tau \times \alpha)$. We may express $U$ as $\bigcup\limits_{i \in I} y_i \times b_i$, where the $y_i$ and $b_i$ are in the bases $\mathfrak{b}_\sigma$ and $\mathfrak{b}_\beta$ respectively. Since for any relations we have that $R(A \cup B) = R(A) \cup R(B)$ and $(R \times S)^\dag = R^\dag \times S^\dag$:
\begin{align*}
&(R \times S)^\dag(\bigcup\limits_{i \in I} y_i \times b_i)\\
 &= \bigcup\limits_{i \in I}(R \times S)^\dag(y_i \times b_i)\\
 &= \bigcup\limits_{i \in I}(R^\dag \times S^\dag)(y_i \times b_i)
 \end{align*}
Since each $y_i$ is open and $R$ is continuous, $R^\dag(y_i) \in \tau$. Symmetrically, $S^\dag(b_i) \in \alpha$. So each $(R^\dag \times S^\dag)(y_i \times b_i) \in (\tau \times \alpha)$. Topologies are closed under arbitrary union, so we are done.

\newthought{The natural isomorphisms are inherited from \textbf{Rel}}. We will be explicit with the unitor, but for the rest, we will only check that the usual isomorphisms from \textbf{Rel} are continuous in \textbf{ContRel}. To avoid bracket-glut, we will vertically stack some tensored expressions.

\newthought{Unitors:} The left unitors are defined as the relations $\lambda_{X^{\tau}}: \bullet \times X^\tau \rightarrow X^\tau := \{(\begin{pmatrix}\star \\x \end{pmatrix}, x) \ | \ x \in X\}$, and we reverse the pairs to obtain the inverse $\lambda^{-1}_{X^{\tau}}$. These relations are continuous since the product topology of $\tau$ with the singleton is homeomorphic to $\tau$: $U \in \tau \iff (\bullet,U) \in (\bullet \times \tau)$. These relations are evidently inverses that compose to the identity. The construction is symmetric for the right unitors $\rho_{X^{\tau}}$.

\newthought{Associators:}
The associators $\alpha_{X^{\tau}Y^{\sigma}Z^{\rho}} : ((X \times Y) \times Z)^{((\tau \times \sigma) \times \rho)} \rightarrow (X \times (Y \times Z))^{(\tau \times (\sigma \times \rho))}$ are inherited from \textbf{Rel}. They are:
\[\alpha_{X^{\tau}Y^{\sigma}Z^{\rho}} := \{\big( \ (\begin{pmatrix} x \\ y \end{pmatrix} , z) \ , \ (x, \begin{pmatrix} y \\ z \end{pmatrix}) \big) \quad | \quad x \in X \ , \ y \in Y \ ,\ z \in Z \}\]
To check the continuity of the associator, observe that product topologies are isomorphic in \textbf{Top} up to bracketing, and these isomorphisms are inherited by \textbf{ContRel}. The inverse of the associator has the pairs of the relation reversed and is evidently an inverse that composes to the identity.

\newthought{Braids:}
The braidings $\theta_{X^{\tau}Y^{\sigma}} : (X \times Y)^{\tau \times \sigma} \rightarrow (Y \times X)^{\sigma \times \tau}$ are defined:
\[\{(\begin{pmatrix} x \\ y \end{pmatrix} \ , \ \begin{pmatrix} y \\ x \end{pmatrix}) \quad | \quad x \in X \ , \ y \in Y  \}\]
The braidings inherit continuity from the isomorphisms between $X^\tau \times Y^\sigma$ and $Y^\sigma \times X^\tau$ in \textbf{Top}. They inherit everything else from \textbf{Rel}

\newthought{Coherences:}
Since we have verified all of the natural isomorphisms are continuous, it suffices to say that the coherences [] are inherited from the symmetric monoidal structure of \textbf{Rel} up to marking objects with topologies.
\end{proof}

\newthought{Monoidal Closure:}
Here is the evaluator.
\[placeholder\]
\end{proposition}

\subsection{Relations that are always continuous}

\newthought{Here are five continuous relations for any $X^\tau$:}

\[\scalebox{0.75}{\tikzfig{bestiary/generators}}\]

\newthought{Copy and delete obey the following equalities:}

\[\scalebox{0.75}{\tikzfig{bestiary/basicrelations}}\]

\newthought{The copy map can also be used to distinguish the deterministic maps -- points and functions -- which we notate with an extra dot.}

\[\scalebox{0.75}{\tikzfig{structure/determinism}}\]

\newthought{Everything, delete, nothing-states and nothing-tests combine to give two numbers, one and zero.} There are extra expressions in grey squares above: they anticipate the tape-diagrams we will later use to graphically express another monoidal product of \textbf{ContRel}, the direct sum $\oplus$.

\[\scalebox{0.75}{\tikzfig{bestiary/scalarrelations}}\]

\newthought{Zero scalars turn entire diagrams into zero morphisms.} There is a zero-morphism for every input-output pair of objects in \textbf{ContRel}. 

\[\scalebox{0.75}{\tikzfig{bestiary/zerorelations}}\]

\marginnote{
\begin{defn}[Biproducts and zero objects]
A \emph{biproduct} is simultaneously a categorical product and coproduct. A \emph{zero object} is both an initial and a terminal object. \textbf{Rel} has biproducts (the coproduct of sets equipped with reversible injections) and a zero object (the empty set).
\end{defn}

\begin{proposition}
$\mathbf{ContRel}$ has a zero object.
\begin{proof}
As in \textbf{Rel}, there is a unique relation from every object to and from the empty set with the empty topology.
\end{proof}
\end{proposition}

\begin{proposition}
$\mathbf{ContRel}$ has biproducts.
\begin{proof}
The biproduct of topologies $X^\tau$ and $Y^\sigma$ is their direct sum topology $(X \sqcup Y)^{(\tau + \sigma)}$ -- the coarsest topology that contains the disjoint union $\tau \sqcup \sigma$. As in \textbf{Rel}, the (in/pro)jections are partial identities, which are continuous by construction. To verify that it is a coproduct, given continuous relations $R: X^\tau \rightarrow Z^\rho$ and $S: Y^\sigma \rightarrow Z^\rho$, where the disjoint union $X \sqcup Y$ of sets is $\{x_1 \ | \ x \in X\} \cup \{y_2 \ | \ y \in Y\}$, we observe that $R + S := \{ (x_1,z) \ | \ (x,z) \in R \} \cup \{ (y_2,z) \ | \ y \in S \}$ is continuous and commutes with the injections as required. The argument that it is a product is symmetric.
\end{proof}
\end{proposition}

\begin{remark}
Biproducts yield another symmetric monoidal structure which the $\times$ monoidal product distributes over appropriately to yield a rig category. Throughout the chapter we have been using $\cup$, but we could have also "diagrammatised" $\cup$ by treating it as a monoid internal to \textbf{ContRel} viewed as a symmetric monoidal category with respect to the biproduct. There are two diagrammatic formalisms for rig categories that we could have used, [] and []. Neither case is perfectly suitable due to the fact that we sometimes took unions over arbitrary indexing sets, which is alright in topology but not depictable as a finite diagram in the $\oplus$-structure. A neat fact that follows is that a topological space is compact precisely when any arbitrarily indexed $\cup$ of tests in the $\times$-structure is \emph{depictable} in the $\oplus$-structure of either diagrammatic calculus for rig categories. \textbf{FdHilb} also has a monoidal product notated $\otimes$ that distributes over the monoidal structure given by biproducts $\oplus$. In contrast, we have used $\times$ -- the cartesian product notation -- for the monoidal product of \textbf{ContRel} since that is closer to what is familiar for sets.
\end{remark}
}