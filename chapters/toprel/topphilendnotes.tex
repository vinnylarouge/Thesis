\section{Philosopher's endnotes}

\newthought{Why topology?} Because the opens of topological spaces crudely model how we talk about concepts, and the points of a topological space crudely model instances of concepts. Why this is so is best demonstrated by an illustrated example.

\newthought{Points in space are a mathematical fiction.} Useful, but a fiction. Suppose we have a point on a unit interval. Consider how we might tell someone else about where this point is. We could point at it with a pudgy appendage, or the tip of a pencil, or give some finite decimal approximation.

\[\tikzfig{topology/pointingfinger}\]

But in each case we are only speaking of a vicinity, a neighbourhood, an \emph{open set in the borel basis of the reals} that contains the point. Identifying a true point on a real line requires an infinite intersection of open balls of decreasing radius; an infinite process of pointing again and again, which nobody has the time to do. Language is only capable of offering successively finer, finite approximations to whatever it is that occurs in the mind or in reality.

\newthought{Maybe that explains the asymmetry of why tests are open sets, but why are states allowed to be arbitrary subsets?} Because states in this model represent what is conceived or perceived. Suppose we have an analog photograph whether in hand or in mind, and we want to remark on a particular shade of red in some uniform patch of the photograph. As in the case of pointing out a point on the real interval, we have successively finer approximations with a vocabulary of concepts: "red", "burgundy", "hex code #800021"... but never the point in colourspace itself. If someone takes our linguistic description of the colour and tries to reproduce it, they will be off in a manner that we can in principle detect, cognize, and correct: "make it a little darker" or "add a little blue to it". That is to say, there are, in principle, differences in mind that we cannot distinguish by boundedly finite language; we would have to continue the process of "even darker" and "add a bit less blue than last time" forever. All this is just the mathematical formulation of a very common observation: sometimes you cannot do an experience justice with words, and you eventually give up with "I guess you just had to be there". Yet the experience is there and we can perform linguistic operations on it, and the states accommodate this.

\newthought{Why the \textbf{Rel} fetish? Why not just use sets and functions?} They are mathematically equivalently expressive in a sense, but the diagrams are nicer in \textbf{Rel}. To each their own.