\section{Basic topological concepts in flatland via \textbf{ContRel}}\label{sec:concepts}

The goal now is to demonstrate the use of the sticky spiders we have obtained in the previous section as formal semantics for the kinds of schematic doodles or cartoons we would like to draw. Throughout we consider sticky spiders on $\mathbb{R}^2$. In service of defining \emph{configuration spaces} of shapes up to rigid displacement, we diagrammatically characterise the topological subgroup of isometries of $\mathbb{R}^2$ by building up in the diagrammatic presentations of the unit interval, metrics, and topological groups. To further isolate rigid displacements that arise from continuous sliding motion of shapes in the plane (thus excluding displacements that result in mirror-images), we diagrammatically characterise an analogue of homotopy in the relational setting. Finally, in we build up a stock of topological concepts and study by examples how implementing these concepts within text circuits explains some idiosyncrasies of the theory: namely why noun wires are labelled by their noun, why adjective gates ought to commute, and why verb gates do not.

\newthought{Itinerary of upcoming examples and definitions:} I would like to say it's obvious how to make shapes move and tell when they are touching and so on, but you may not believe me. So, I am going to recover a fair chunk of elementary topology diagrammatically in \textbf{ContRel}.

\newthought{Shapes and places} We can already do interesting things with just sticky spiders. In Example \ref{ex:chessboard}, we show how we can use sticky spiders to tell how pieces are on a chessboard. How is it that pieces on a chessboard in \emph{continuous} space are translated into \emph{discrete} data, such as \texttt{pawn on e4}?

\newthought{The reals}
To begin modelling more complex concepts, we first need to extend our topological tools. The first stop is the reals. There are many spaces homeomorphic to the real line. How do we know when we have one of them? The following theorem provides an answer:
\begin{theorem}[Friedman]\label{thm:Friedman}
Let $\big((X,\tau), < \big)$ be a topological space with a total order. If there exists a continuous map $f: X \times X \rightarrow X$ such that $\forall a,b_{\in X} : a < f(a,b) < b$, then $X$ is homeomorphic to $\mathbb{R}$. \citep{friedman_fom_2005}
\end{theorem}
We diagrammatically characterise total orders in Definition \ref{defn:lessthan}. Here Trichotomy requires us to appeal to the rig structure so that we can access unions of processes explicitly. Going forward we will also introduce quantifiers into process-theoretic equations, essentially treating process-equations as we would any other symbolic algebraic specification. Then we express Friedman's theorem using diagrammatic equations in Definition \ref{defn:friedfunct}.

\newthought{The unit interval}
If we have the unit interval, we can begin to define what it would mean for spaces to be connected (by drawing lines between points in those spaces), and we can also move towards defining motion as movement along a line. To obtain the unit interval (up to homeomorphism) we need to add endpoints to $\mathbb{R}$. We can introduce endpoints for open intervals directly by asking for the space $X$ to have points that are less than or greater than all other points. Another method, which we will use here for primarily aesthetic reasons, is to use endocombinators to define suprema. For a motivating example of the ubiquity of endocombinators, consider the case when we have a locally indiscrete topology:
\begin{defn}[Locally indiscrete topology]
$(X,\tau)$ is \emph{locally indiscrete} when every open set is also closed.
\end{defn}
If we know that a topology is locally indiscrete and we are given an open $U$, we would like to notate the complement $X/U$ -- which we know to be open -- as any of the following, which only differ up to notation.
\[\resizebox{0.5\textwidth}{!}{\tikzfig{topology/negbubble}}\]
Unfortunately, the complementation operation $X/-$ is not in general a continuous relation, hence in the lattermost expression above we resort to using bubbles -- endocombinators -- as syntactic sugar. Endocombinators are like functional expressions applied to diagrams: the semantics and notation for which we borrow and modify from \citep{haydon_compositional_2020}.
\begin{defn}[Partial endocombinator]
In a category $\mathcal{C}$, a \emph{partial endocombinator} on a homset $\mathcal(C)(A,B)$ is a function $\mathcal(C)(A,B) \rightarrow \mathcal(C)(A,B)$
\end{defn}

\newthought{Simply connected shapes and spaces}

Once we have a unit interval, we can define the usual topological notion of a simply connected space (Definition \ref{def:simpconn}): one where any two points can be connected by a continuous line without leaving the space.

\newthought{Displacing shapes}

Static shapes in space are nice, but moving them around would be nicer. So we have to define a stock of concepts to express rigid motion. Rigidity however is a difficult concept to express in topological spaces up to homeomorphism -- everyone is well aware of the popular gloss of topology in terms of coffee cups being homeomorphic to donuts. To obtain rigid transformations as we have in Euclidean space, we need to define metrics (Definition \ref{def:metric}), and in order to do that, we need addition (Definition \ref{def:addition}).

\newthought{Rigid displacements}

Then we return to sticky spiders and we restrict our consideration to those on the open unit square, so that we can speak of shapes on a canvas. We want to rigidly displace the shapes of a sticky spider, which is obtained via the notions of isometries (Definition \ref{def:isometry}) and topological groups (Definition \ref{def:topgrp}).

\begin{remark}
When we draw on a finite canvas representing all of euclidean space, properly there should be a fishbowl effect that relatively magnifies shapes close to the origin and shrinks those at the periphery, but that is only an artefact of representing all of euclidean space on a finite canvas. Since all the usual metrics are still really there, going forward we will ignore this fishbowl effect and just doodle shapes as we see fit.
\[\tikzfig{topology/fisheye}\]
\end{remark}

\newthought{Motion of rigid bodies}

To obtain continuous transformations in the plane from the configuration of shapes in one spider to end at the configuration of shapes in another, we define a conservative generalisation of \emph{homotopy}: the continuous deformation of one map to another (Definition \ref{defn:homotopy}). A side benefit we obtain is that when we have homotopies, we can also define contractible shapes, which are connected and have no holes. Then we define configuration spaces of shapes on the plane so we can talk about all permissible positions shapes may be in, and at last we can define motions of rigid bodies.

\newthought{Modelling linguistic topological concepts}

By "linguistic", I mean to refer to the kinds of concepts we use in everyday language, and by "topological" I \emph{don't} mean spatial relationships that are invariant under homeomorphism (as recall, we deal primarily with rigid and finitely deformable objects in our day-to-day). I mean to refer to non-metric and non-directional spatial relationships such as parthood, touching, within/surrounds, containment/entrapment, interlocking, and mutually constrained motions. These are concepts that even very young children have an intuitive grasp of \citep{jean_childs_1967}, but their formal definitions are difficult to pin down. One such relation modelled here -- touching -- is in fact a \emph{semantic prime} \citep{wierzbicka_semantics_1996}: a word that is present in essentially all natural languages that is conceptually primitive, in the sense that it resists definition in simpler terms. It is among the ranks of concepts like \emph{wanting} or \emph{living}, words that are understood by the experience of being human, rather than by school. As such, I make no claim that these definitions are canonical, just that they are good enough to reason with. Then I will introduce some dynamic relationships such as collision, along with some remarks on how vignettes may be composed out of homotopies and a general recipe for obtaining homotopies that -- via deformation -- relate any pair of sticky spiders with the same number of shapes. This analysis extends to talk of rigid and deforming bodies and the manner, order, and coordination of their movement and interaction in any space.

\newthought{Thus, I consider any linguistic semantics that can be grounded in mechanical or tabletop models, to be formal.} By Example \ref{ex:chessboard}, we have enough technology to speak of locations in space, so we have access to "tabletop semantics": anything that in principle can be represented by counters and meeples in a boardgame, with for instance reserved spaces on the board for health and hunger and whatever else is necessary. The analysis of rigid bodies is hopefully sufficient to demonstrate that, in principle, one may linguistically characterise mechanical models, and there are a lot of mechanical models, including: mechanically realised clocks [duh.], analogues of electric circuits \citep{noauthor_spintronics_nodate}, computers \citep{richard_ridel_mechanical_2015}, and human-like automata \citep{wikipedia_authors_jaquet-droz_2022}. Of course in reality mechanical motions are reversible among rigid objects, and directional behaviour is provided by a source of energy, such as gravitational potential, or wound springs. But we may in principle replace these sources of energy by a belt that we choose to spin in one direction -- our own arrow of time.

\newthought{\textbf{Objection:} That is way outside the scope of formal semantics.} Insofar as semantics is sensemaking, we certainly are capable of making sense of things in terms of mechanical models and games by means of metaphor, the mathematical treatment of which is concern of Section \ref{sec:metaphor}. Part of the reason I went through the trouble of deriving linguistic topological relations from nothing is so that I can claim that by any sensible definition, I am doing formal semantics for natural language. Whether or not I'm exceeding the scope of what a linguist might consider formal semantics is ultimately irrelevant, as I am not concerned with the modal human mechanism, only with process-theoretic characterisations of things, because they tend to be worthwhile practically. There is maybe also a prejudice that formal semantics must necessarily resolve in some symbolic logic, to which I might charitably respond that I'm working with an algebraic system, just not a one-dimensional one. Less charitably, I don't care what you think linguistics ought to be.

\newpage

\begin{myboxB}
\begin{example}[Where is a piece on a chessboard?]\label{ex:chessboard}
How is it that we quotient away the continuous structure of positions on a chessboard to locate pieces among a discrete set of squares? Evidently shifting a piece a little off the centre of a square doesn't change the state of the game, and this resistance to small perturbations suggests that a topological model is appropriate. We construct two spiders, one for pieces, and one for places on the chessboard. For the spider that represents the position of pieces, we open balls of some radius $r$, and we consider the places spider to consist of square halos (which tile the chessboard), containing a core inset by the same radius $r$; in this way, any piece can only overlap at most one square. As a technical aside, to keep the core of the tiles open, we can choose an arbitrarily sharp curvature $\epsilon$ at the corners.
\[\scalebox{0.70}{\tikzfig{topology/chessboard}}\]
Now we observe that the calculation of positions corresponds to composing sticky spiders. We take the initial state to be the sticky spider that assigns a ball of radius $r$ on the board for each piece. We can then obtain the set of positions of each piece by composing with the places spider. The composite (pieces;places)
will send the king to a2, the bishop to b4, and the knight to d1, i.e. $\bra{K} \mapsto \bra{a2}$, $\bra{B} \mapsto \bra{b4}$ and $\bra{N} \mapsto \bra{d1}$. In other words, we have obtained a process that models how we pass from continuous states-of-affairs on a physical chessboard to an abstract and discrete game-state.
\[\resizebox{0.70\textwidth}{!}{\tikzfig{topology/chessboard2a}}\]
\end{example}
\end{myboxB}

\begin{myboxR}
\begin{defn}[Less than]\label{defn:lessthan}
We define a total ordering relation $<$ as an open set on $X \times X$ that obeys the usual axiomatic rules:
\[\resizebox{\textwidth}{!}{\tikzfig{topology/lessthan}}\]
\[\tikzfig{topology/lessthantotal}\]
\end{defn}
\end{myboxR}

\begin{myboxB}
\begin{defn}[Friedman's function]\label{defn:friedfunct}
Just as a wire in \textbf{ContRel} has the discrete topology if it possesses spider structure (Proposition \ref{prop:copydiscrete}), a wire is homeomorphic to the real line by Theorem \ref{thm:Friedman} if it possesses an open that behaves as Definition \ref{defn:lessthan}, and a map that satisfies:
\[\tikzfig{topology/betweenfunct}\]
\end{defn}
\end{myboxB}

\begin{myboxR}
\begin{defn}[Upper and lower bounds via endocombinators]\label{defn:bounds}
Upper bounds are endocombinators that send states to points. This is an equational condition quantified over all states.
\[\tikzfig{topology/upperbound}\]
\end{defn}
We can add in further equations governing the upper bound endocombinator to turn it into a supremum, where the lower endpoint is obtained as the supremum of the empty set, and the upper endpoint is the supremum of the whole set.
\begin{defn}[Suprema]\label{defn:sup}
\[\tikzfig{topology/sup}\]
\end{defn}
\begin{defn}[Endpoints]\label{defn:endpoints}
Now we can define endpoints purely graphically. The lower endpoint is the supremum of the empty state, and the upper the supremum of everything.
\[\tikzfig{topology/endpoints}\]
\end{defn}
Going forward, we will denote the unit interval using a thick dotted wire.

\end{myboxR}

\begin{myboxB}
\begin{defn}[Simple connectivity]\label{def:simpconn}
Recall that we notate points and functions with the same small black dot for copying and deleting, as points are precisely the states that are copy-delete cohomomorphisms. In prose, simple connectivity states that for any pair of points that are within the open $V$, there exists some continuous function from the unit interval into the space that starts at one of the points and ends at the other. The left pair of conditions state that the points $\textcolor{blue}{x}$ and $\textcolor{red}{y}$ are within $V$. The right triple of conditions require the the image of the homotopy $\textcolor{orange}{f}$ is contained in $V$, and that its endpoints are $\textcolor{blue}{x}$ and $\textcolor{red}{y}$.
\[\tikzfig{topology/simplyconnected}\]
Simple connectivity is a useful enough concept that we will notate simply connected open sets as follows, where the hole is a reminder that simply connected spaces might still have holes in them.
\[\tikzfig{topology/simpconnotation}\]
\end{defn}
\end{myboxB}

\begin{myboxR}
\begin{defn}[Addition]\label{def:addition}
In order to define metrics, we must have additive structure, which we encode as an additive monoid that is a function. All we need to know is that the lower endpoint of the unit interval stands in for "zero distance" -- as the unit of the monoid -- and that adding positive distances together will deterministically give you a larger positive distance.
\[\resizebox{\textwidth}{!}{\tikzfig{topology/addition}}\]
\end{defn}
\end{myboxR}

\begin{myboxB}
\begin{defn}[Metric]\label{def:metric}
A metric on a space is a continuous map $X \rightarrow \mathbb{R}^+$ to the positive reals that satisfies these axioms. We depict metrics as trapezoids.
\end{defn}
\[\resizebox{\textwidth}{!}{\tikzfig{topology/metric}}\]
\end{myboxB}

\begin{myboxR}
\begin{defn}[Open balls]\label{def:openball}
Once we have metrics, we can define the usual topological notion of open balls. With respect to a metric, an $\varepsilon$-open ball at $\textcolor{blue}{x}$ is the open set (effect) of all points that are $\varepsilon$-close to $\textcolor{blue}{x}$ by the chosen metric.
\[\tikzfig{topology/openball2}\]
Open balls will come in handy later, and a side-effect which we note but do not explore is that open balls form a basis for any metric space, so in the future whenever we construct spaces that come with natural metrics, we can speak of their topology without any further work.
\end{defn}
\end{myboxR}

\begin{myboxB}
With metrics in hand, we can proceed to define isometries as topological groups that keep distances invariant.
\begin{defn}[Topological groups]\label{def:topgrp} 
It is no trouble to depict collections of invertible transformations of spaces $X \rightarrow X$.
\[\tikzfig{topology/topologicalgroup}\]
A consequence of invertibility and the requirement that the identity transform is a group element forces all transformations in a topological group to be functions.
\end{defn}
\begin{defn}[Isometry]\label{def:isometry}
An isometry is a distance preserving transformation: applying the metric pointwise before and after a topological group action gives the same value.
\[\tikzfig{topology/isometry}\]
\end{defn}
\end{myboxB}

\begin{myboxB}
\begin{defn}[Planar isometries]\label{defn:planarisometry}
We know the planar isometries of Euclidean space can be expressed as a translation, rotation, and an addition bit of data to indicate the chirality of the shape -- as mirror reflections are also an isometry. In flatland, we do not expect shapes to suddenly flip over. We would like to express just those rigid transformations that leave the chirality of the shape intact, because really we want to only be able to slide the shapes around the canvas, not leave the canvas to flip over.
\[\tikzfig{topology/planeisometries}\]
With this in mind, we have the following condition relating different spiders on the unit square, telling us when one is the same as the other up to rigidly displacing shapes.
\[\tikzfig{topology/displacerigid}\]
\end{defn}
\end{myboxB}

\begin{myboxR}
\begin{remark}\label{remark:homotopywrinkle}
Usually, when we are restricted to speaking of topological spaces and continuous functions, a homotopy is defined:
\end{remark}

\begin{defn}[Homotopy in \textbf{Top}]
where $f$ and $g$ are continuous maps $A \rightarrow B$, a \emph{homotopy} $\eta : f \Rightarrow g$ is a continuous function $\eta : [0,1] \times A \rightarrow B$ such that $\eta(0,-) = f(-)$ and $\eta(1,-) = g(1,-)$.
\end{defn}

In other words, a homotopy is like a short film where at the beginning there is an $f$, which continuously deforms to end the film being a $g$. Directly replacing "function" with "relation" in the above definition does not quite do what we want, because we would be able to define the following "homotopy" between open sets.

\[\tikzfig{topology/homotopyctrex1}\]

What is happening in the above film is that we have our starting open set, which stays constant for a while. Then suddenly the ending open set appears, the starting open disappears, and we are left with our ending; while \emph{technically} there was no discontinuous jump, this isn't the notion of sliding we want. The exemplified issue is that we can patch together (by union of continuous relations) vignettes of continuous relations that are not individually total on $[0,1]$.
\end{myboxR}

\begin{myboxB}
\begin{defn}[Relational Homotopy]\label{defn:homotopy}
We can patch the issue in Remark \ref{remark:homotopywrinkle} by asking for homotopies in \textbf{ContRel} to satisfy the additional condition that they are expressible as a union of continuous partial maps that are total on the unit interval.
\[\tikzfig{topology/homotopy}\]
\end{defn}
\end{myboxB}

\begin{myboxR}
\begin{remark}
Observe that the second condition asking for decomposition in terms of partial comes for free by Proposition \ref{prop:hombasis}; the constraint of the definition is provided by the first condition, which is a stronger condition than just asking that the original continuous relation be total on $I$:
\[\tikzfig{topology/homotopyctrex1}\]
Definition \ref{defn:homotopy} is "natural" in light of Proposition \ref{prop:hombasis}, that the partial continuous functions $A \rightarrow B$ form a basis for $\mathbf{ContRel}(A,B)$: we are just asking that homotopies between partial continuous functions -- which can be viewed as regular homotopies with domain restricted to the subspace topology induced by an open set -- form a basis for homotopies between continuous relations.
\[\tikzfig{topology/homotopyctrex}\]
\end{remark}
\end{myboxR}

\begin{myboxB}
\begin{defn}[Contractibility]\label{defn:contractible}
With homotopies in hand, we can define a stronger notion of connected shapes with no holes, which are usually called \emph{contractible}. The reason for the terminology reflects the method by which we can guarantee a shape in flatland has no holes: when any loop in the shape is \emph{contractible} to a point.
\[\resizebox{\textwidth}{!}{\tikzfig{topology/contractible}}\]
Contractible open sets are worth their own notation too; a solid black effect, this time with no hole.
\[\tikzfig{topology/contractnotation}\]
\end{defn}
\end{myboxB}

\begin{myboxR}
\begin{defn}[Touching]\label{defn:touching}
Let's distinguish touching from overlap. Two shapes are "touching" intuitively when they are as close as they can be to each other, somewhere; any closer and they would overlap. Let's assume that we can restrict our attention to the parts of the shape that are touching, and that we can fill in the holes of these parts. At the point of touching, there is an infinitesimal gap -- just as when we touch things in meatspace, there is a very small gap between us and the object due to the repulsive electromagnetic force between atoms. To deal with infinitesimals we borrow the $\epsilon-\delta$ trick from mathematical analysis; for any arbitrarily small $\delta$, we can pick an even smaller ball of radius $\epsilon$ such that if we stick the ball in the gap, the ball forms a bridge that overlaps the two filled-in shapes, which allows us to draw a continuous line between them. Diagrammatically, this is: 
\[\resizebox{\textwidth}{!}{\tikzfig{topology/touching}}\]
\end{defn}
\end{myboxR}

\begin{myboxB}
\begin{defn}[Within and surrounds]\label{defn:within}
If $U$ surrounds $V$, or equivalently, if $V$ is within $U$, then we are saying that leaving $V$ in almost any direction, we will see some of $U$ before we go off to infinity. We can once again use open balls for this purpose, which correspond to possible places you can get to from a starting point $\mathbf{x}$ within a distance $\epsilon$. In prose, we are asking that any open ball that contains all of $U$ must also contain all of $V$.
\[\tikzfig{topology/within}\]
\end{defn}
\end{myboxB}

\begin{myboxR}
\begin{defn}[Containment and enclosure]\label{defn:containment}
There is a strong version of within-ness, which we will call enclosure. As in when we are in elevators and the door is shut, nothing gets in or out of the container. Intuitively, there is a hole in the container surrounded on all sides, and the contained shape lives within the hole. To give a real-world example, honey lives within a honeycomb cell in a beehive, but whether the honey is enclosed in the cell depends on whether it is sealed off from air with beeswax. So in prose we are asking that any way we fill in the holes of the container with a blob, that blob must cover the contained shape. Diagrammatically, this amounts to levelling up from open balls in our previous definition to contractible sets:
\[\tikzfig{topology/enclose}\]
\end{defn}
\end{myboxR}


\begin{myboxB}
\begin{defn}[Trapping]\label{defn:trapped}
There is an intermediate notion between within-ness and enclosure; for instance, standing in the stonehenge you are surrounded by the pillars, but you can always walk away, whereas if the pillars are very close, such as the bars of a jail cell, a human would not be able to leave the trap while still being able to see the outside. The difficulty here is that relative sizes come into play: small animals would still consider it a case of mere within-ness, because they can still walk away between the bars. So we would like to say that no matter how the pair of objects move rigidly, being trapped means that the trapped $V$ stays within $U$. In other words, that in configuration space, if we forget about all other shapes, we can partition our space of configurations by two concepts, whether $V$ is within $U$ or not, and moreover that these two components is disjoint -- i.e. not simply connected -- so there is no rigid motion that can allow $V$ to escape from being within $U$ if $V$ starts off trapped inside in $U$.
\[\resizebox{\textwidth}{!}{\tikzfig{topology/trapped}}\]
\end{defn}
\end{myboxB}

\begin{myboxR}
\begin{defn}[Interlock]\label{defn:interlocked}
Two shapes might be tightly interlocked without being inside one another. Some potentially familiar examples are plastic models of molecular structure that we encounter in school, metal lids in cold weather that are too tightly hugging the glass jar, or stubborn Lego pieces that refuse to come apart. The commonality of all these cases is that the two shapes must move together as one, unless deformed or broken. In other words, when two shapes are interlocked, knowing the position in space of one shape determines the position of the other, and this determination is a fixed isometry of space. So we only need to specify a range of positions $S$ for the entire subconfiguration of interlocked shapes $U$ and $V$, and we may obtain their respective positions by a fixed rigid motion $\rho$. Since objects may interlock in multiple ways, we may have a sum of these expressions. We additionally observe that interlocking shapes should also be touching, which translates to containment inside the touching concept. Finally, we observe that as in the case of entrapment and enclosure, rigid motions are interlocking-invariant, which translates diagrammatically to the constraint that each $S,\rho$ expression is an entire connected component in configuration space.
\[\resizebox{\textwidth}{!}{\tikzfig{topology/interlock}}\]
\end{defn}
\end{myboxR}

\begin{myboxB}
\begin{defn}[Mutual constraint]\label{defn:constrained}
A weaker notion of interlocking is when shapes only imperfectly determine each other's potential displacements, by specifying an allowed range. It would be an understatement to say there is some interest in studying how shapes mutually constrain each other's movements in this way.
\[\tikzfig{topology/constrained}\]
\end{defn}
\end{myboxB}

\begin{comment}
\begin{myboxB}
\begin{construction}[Morphing sticky spiders with homotopies]\label{cons:morph}
We aim to construct homotopies relating (almost) arbitrary sticky spiders. For now we focus on just changing one shape into another arbitrary one. The idea is as follows. First, we need a cover of open balls $\cup\mathcal{J} = T^0$ and $\cup\mathcal{K} = T^1$ of the start and end cores $T^0$ and $T^1$ such that each $k \in T^1$ is expressible as a rigid isometry of some core $j \in \mathcal{J}$; this is so we can slide and rearrange open balls comprising $T^0$ and reconstruct them as $T^1$. As an intermediate step to eliminate holes and unify connected components, we gather all of the balls at a meeting point $m$ (to be determined shortly.) Intuitively we can illustrate this process as follows:
\[\tikzfig{topology/slideconstruction}\]
Second, in order to perform the sliding of open balls, we observe that, given a basepoint to act as origin (which we assume is provided by the data of the split idempotent of configuration space) we can express the group action of rigid isometries $\mathbf{Iso}(\mathbf{R}^2)$ on $\mathbf{R}^2$ as a continuous function:
\[\tikzfig{topology/groupaction}\]
\end{construction}
\end{myboxB}

\begin{myboxB}
Third, before we begin sliding the open balls, we must ensure that the halo of the shape cooperates. We observe that a given shape $i$ in a sticky spider may be expressed as the union of a family of constant continuous partial functions in the following way. Given an open cover $\mathcal{J}$ such that $\cup\mathcal{J} = T_i$, where $T_i$ is the core of the shape $i$, each function is a constant map from some $T_j \in \mathcal{J}$ to some point $x \in S_i$, where $S_i$ is the halo of the shape $i$. For each $T_j \in \mathcal{J}$ and every point $x \in S_i$, the constant partial function that maps $T_j$ to $x$ is in the family.
\[\tikzfig{topology/spiderbreakdown}\]
By definition of sticky spiders, there must exist some point $m$ that is in both the core and the halo: we pick such a point as the rendezvous for the open balls. For each partial map in the family, we provide a homotopy that varies only the image point $x$ continuously in the space to finish at $m$. Now we can slide the open balls to the rendezvous $m$. Since homotopies are reversible by the continuous map $t \mapsto (1-t)$ on the interval, we can perform the above steps for shapes $T^0$ and $T^1$ to finish at the same open ball, reversing the process for $T^1$ and composing sequentially to obtain a finished transformation. The final wrinkle to address is when dealing with multiple shapes. Recalling our exclusion conditions (Propositions \ref{prop:core-core-exclusion, prop:core-halo-exclusion}) for shapes, it may be that parts of one shape are enclosed in another, so the processes must be coordinated so that there are no overlaps. For example, the enclosing shape must be first opened, so that the enclosed shape may leave. I will keep it an article of faith that such coordinations exist. I struggle to come up with a proof that all sticky spiders $\mathbf{R}^2$ are mutually transformable by homotopy in this (or any other) way, so that will remain a conjecture. But it is clear that a great deal of spiders are mutually transformable; almost certainly any we would care to draw. So this will just be a construction for now.
\end{myboxB}
\end{comment}

\clearpage
\newpage

