\section{Mathematician's endnotes}

\subsection{The category \textbf{TopRel}}

\begin{proposition}[\textbf{TopRel} is a category]
continuous relations form a category $\mathbf{TopRel}$.
\begin{proof}
\newthought{Identities:} Identity relations, which are always topological.
\newthought{Composition:} The normal composition of relations. We verify that the composite $X^\tau \overset{R}{\rightarrow} Y^\sigma \overset{S}{\rightarrow} Z^\theta$ of continuous relations is again continuous as follows:
\[U \in \theta \implies S^\dag(U) \in \sigma \implies R^\dag \circ S^\dag(U) = (S \circ R)^\dag \in \tau\]
\newthought{Associativity of composition:} Inherited from \textbf{Rel}.
\end{proof}
\end{proposition}

\subsection{Symmetric Monoidal structure}

\marginnote{
\begin{rem}[Product Topology]
We denote the product topology of $X^\tau$ and $Y^\sigma$ as $(X \times Y)^{(\tau \times \sigma)}$. $\tau \times \sigma$ is the topology on $X \times Y$ generated by the basis $\{t \times s : t \in \mathfrak{b}_\tau, s \in \mathfrak{b}_\sigma\}$, where $\mathfrak{b}_\tau$ and $\mathfrak{b}_\sigma$ are bases for $\tau$ and $\sigma$ respectively.
\end{rem}
}

\begin{proposition}
$(\mathbf{TopRel},\{\star\}^{\bullet},X^\tau \otimes Y^\sigma := (X \times Y)^{(\tau \times \sigma)})$ is a symmetric monoidal closed category.
\begin{proof}

\newthought{Tensor Unit:} The one-point space $\bullet$. Explicitly, $\{\star\}$ with topology $\{\varnothing,\{\star\}\}$.

\marginnote{
    \begin{rem}[Product of relations]
    For relations between sets $R: X \rightarrow Y, S: A \rightarrow B$, the product relation $R \times S: X \times A \rightarrow Y \times B$ is defined to be \[ \{ ((x,a),(y,b)) : (x,y) \in R, (a,b) \in S \} \]
    \end{rem}
}

\newthought{Tensor Product:} For objects, $X^\tau \otimes Y^\sigma$ has base set $X \times Y$ equipped with the product topology $\tau \times \sigma$. For morphisms, $R \otimes S$ the product of relations. We show that the tensor of continuous relations is again a continuous relation. Take continuous relations $R: X^\tau \rightarrow Y^\sigma$, $S: A^\alpha \rightarrow B^\beta$, and let $U$ be open in the product topology $(\sigma \times \beta)$. We need to prove that $(R \times S)^\dag(U) \in (\tau \times \alpha)$. We may express $U$ as $\bigcup\limits_{i \in I} y_i \times b_i$, where the $y_i$ and $b_i$ are in the bases $\mathfrak{b}_\sigma$ and $\mathfrak{b}_\beta$ respectively. Since for any relations we have that $R(A \cup B) = R(A) \cup R(B)$ and $(R \times S)^\dag = R^\dag \times S^\dag$:

\begin{align*}
&(R \times S)^\dag(\bigcup\limits_{i \in I} y_i \times b_i)\\
 &= \bigcup\limits_{i \in I}(R \times S)^\dag(y_i \times b_i)\\
 &= \bigcup\limits_{i \in I}(R^\dag \times S^\dag)(y_i \times b_i)
 \end{align*}

Since each $y_i$ is open and $R$ is continuous, $R^\dag(y_i) \in \tau$. Symmetrically, $S^\dag(b_i) \in \alpha$. So each $(R^\dag \times S^\dag)(y_i \times b_i) \in (\tau \times \alpha)$. Topologies are closed under arbitrary union, so we are done.

\newthought{Unitors:} The left unitors $\lambda_{X^{\tau}}: \bullet \times X^\tau \rightarrow X^\tau$ maps $(\star,x) \mapsto x$, and we reverse the direction of the mapping to obtain the inverse $\lambda^{-1}_{X^{\tau}}$. The construction is symmetric for the right unitors $\rho_{X^{\tau}}$. 

\newthought{Associators:}

The associators $\alpha_{X^{\tau},Y^{\sigma},Z^{\rho}}$

\newthought{Braids:}

...

\newthought{Coherences:}

...

\end{proof}
\end{proposition}

\subsection{Rig category structure}

\begin{defn}[Biproducts and zero objects]
A \emph{biproduct} is simultaneously a categorical product and coproduct. A \emph{zero object} is both an initial and a terminal object. \textbf{Rel} has biproducts (the coproduct of sets equipped with reversible injections) and a zero object (the empty set).
\end{defn}

\begin{proposition}
$\mathbf{TopRel}$ has a zero object and biproducts.
\begin{proof}
\end{proof}
\end{proposition}

\begin{notation}[Notations for rig structure]
\textbf{FdHilb} also has a monoidal product notated $\otimes$ that distributes over the monoidal structure given by biproducts $\oplus$. In contrast, we will use $\times$ -- the cartesian product notation -- for the monoidal product of \textbf{TopRel}, and $\bigcup$ for the biproduct structure, since that is closer to what is familiar with sets.
\end{notation}

\subsection{Relation to \textbf{Rel} and \textbf{Loc}}

We provide free-forgetful adjunctions relating \textbf{TopRel} to \textbf{Rel} and \textbf{Loc}, one of which "forgets topology" while the other "forgets points". Together these adjunctions demonstrate that \textbf{TopRel} really does behave as if we have equipped relations with topology and vice versa. So perhaps the clearest motivation for why this is desirable is the ease with which the tensor structure can be defined when there is an underlying space of points -- it is not as easy to define a noncartesian tensor product structure on \textbf{Loc}; I do not know of one, and if there is, I'll wager a pint that \textbf{TopRel} is easier to calculate with.

\newthought{We exhibit a free-forgetful adjunction between \textbf{Rel} and \textbf{TopRel}.}

\begin{lemma}[Any relation $R$ between discrete topologies is continuous]\label{lem:disccont}
\begin{proof}
All subsets in a discrete topologies are open.
\end{proof}
\end{lemma}

\begin{defn}[F: $\mathbf{Rel} \rightarrow \mathbf{TopRel}$] We define the action of the functor $F$:
\begin{description}
\item[On objects] $F(X) := X^\star$, ($X$ with the discrete topology)
\item[On morphisms] $F(X \overset{R}{\rightarrow} Y) := X^\star \overset{R}{\rightarrow} Y^\star$, the existence of which in \textbf{TopRel} is provided by Lemma \ref{lem:disccont}.
\end{description}
Evidently identities and associativity of composition are preserved.
\end{defn}

\begin{defn}[U: $\mathbf{TopRel} \rightarrow \mathbf{Rel}$]
\begin{description} We define the action of the functor $U$ as forgetting the topological structure.
\item[On objects] $U(X^\tau) := X$
\item[On morphisms] $U(X^\tau \overset{R}{\rightarrow} Y^\sigma) := X \overset{R}{\rightarrow} Y$
\end{description}
Evidently identities and associativity of composition are preserved.
\end{defn}


\begin{lemma}[$FU = 1_{\textbf{Rel}}$]\label{lem:idadj}
The composite $FU$ is precisely equal to the identity functor on $\mathbf{Rel}$.
\begin{proof}
On objects, $FU(X) = F(X^\star) = X$. On morphisms, $FU(X \overset{R}{\rightarrow} Y) = F(X^\star \overset{R}{\rightarrow} Y^\star) = X \overset{R}{\rightarrow} Y$
\end{proof}
\end{lemma}

\begin{rem}[Coarser and finer]
Given a set of points $X$ with two topologies $X^\tau$ and $X^\sigma$, if $\tau \subset \sigma$, we say that $\tau$ \emph{is coarser than} $\sigma$, or $\sigma$ \emph{is finer than} $\tau$.
\end{rem}

\begin{lemma}[Coarsening is a continuous relation]\label{lem:coarse}
Let $X^\sigma$ be coarser than $X^\tau$. The identity relation on underlying points $X^\tau \overset{1_X}{\rightarrow} X^\sigma$ is then a continuous relation.
\begin{proof}
The preimage of the identity of any open set $U \in \sigma, U \subseteq X$ is again $U$. By definition of coarseness, $U \in \tau$.
\end{proof}
\end{lemma}

\begin{proposition}[$F \dashv U$]
\begin{proof}
We verify the triangular identities governing the unit and counit of the adjunction, which we first provide. By Lemma \ref{lem:idadj}, we take the natural transformation $1_\mathbf{Rel} \Rightarrow FU$ we take to be the identity morphism:

\[\eta_{X} := 1_{X}\]

The counit natural transformation $UF \Rightarrow 1_{\mathbf{TopRel}}$ we define to be a coarsening, the existence of which in \textbf{TopRel} is granted by Lemma \ref{lem:coarse}.

\[\epsilon_{X^\tau} : X^\star \rightarrow X^\tau := \{(x,x) : x \in X\}\]

First we evaluate $F \overset{F\eta}{\rightarrow} FUF \overset{\epsilon F}{\rightarrow} F$ at an arbitrary object (set) $X \in \textbf{Rel}$. $F(X) = X^\star = FUF(X)$, where the latter equality holds because $FU$ is precisely the identity functor on \textbf{Rel}. For the first leg from the left, $F(\eta_X) = F(1_X) = X^\star \overset{1_X}{\rightarrow} X^\star = 1_{X^\star}$. For the second, $\epsilon_{F(X)} = \epsilon_{X^\star} = X^\star \overset{1_X}{\rightarrow} X^\star = 1_{X^\star}$. So we have that $F\eta ; \epsilon F = F$ as required.\\

Now we evaluate $U \overset{\eta U}{\rightarrow} UFU \overset{U \epsilon}{\rightarrow} U$ at an arbitrary object (topological space) $X^\tau \in \textbf{TopRel}$. $U(X^\tau) = X = UFU(X^\tau)$, where the latter equality again holds because $FU = 1_\textbf{Rel}$. For the first leg from the left, $\eta_{U(X^\tau)} = \eta_X = 1_{X}$. For the second, $U(\epsilon_{X^\tau}) = U(X^\star \overset{1_X}{\rightarrow} X^\tau) = X \overset{1_X}{\rightarrow} X = 1_X$. So $\eta U ; G \epsilon = G$, as required.
\end{proof}
\end{proposition}

\newthought{We exhibit a free-forgetful adjunction between \textbf{Loc} and \textbf{TopRel}}

Just as the forgetful functor from \textbf{TopRel} to \textbf{Rel} "forgets topology while keeping the points", we might consider a forgetful functor to \textbf{Loc} that "forgets points while remembering topology". This subsection touches on point-free topology and takes some mathematical concepts for granted, and is unimportant enough to be skipped by the uninitiated or uninterested reader.

\begin{rem}[The category \textbf{Loc}][]
A \emph{frame} is a poset with all joins and finite meets satisfying the infinite distributive law:
\[x \wedge (\bigvee\limits_{i}y_i) = \bigvee\limits_{i}(x \wedge y_i)\]
A \emph{frame homomorphism} $\phi: A \rightarrow B$ is a function between frames that preserves finite meets and arbitrary joins, i.e.:
\[\phi(x \wedge_A y) = \phi(x) \wedge_B \phi(y), \phi(x \vee_A y) = \phi(x) \vee_B \phi(y)\]
The category \textbf{Frm} has frames as objects and frame homomorphisms as morphisms.\\
The category \textbf{Loc} is defined to be $\textbf{Frm}^\text{op}$.
\end{rem}

\begin{remark}
Here are informal intuitions to ease the definition. The lattice of open sets of a given topology ordered by inclusion forms a frame -- observe the analogy "arbitrary unions" : "all joins" :: "finite intersections" : "finite meets". Closure under arbitrary joins guarantees a maximal element corresponding to the open set that is the whole space. So frames are a setting to speak of topological structure alone, without referring to a set of underlying points, hence, pointless topology. Observe that in the definition of continuous functions, open sets in the \emph{codomain} must correspond (uniquely) to open sets in the \emph{domain} -- so every continuous function induces a frame homomorphism going in the \underline{opposite direction} that the function does between spaces, hence, to obtain the category \textbf{Loc} such that directions align, we reverse the arrows of \textbf{Frm}. Observe that continuous relations induce frame homomorphisms in the same way. These observations give us insight into how to construct the free and forgetful functors.
\end{remark}

\begin{defn}[$U: \textbf{TopRel} \rightarrow \textbf{Loc}$]
On objects, U sends a topology $X^\tau$ to the frame of opens in $\tau$, which we denote $\hat{\tau}$.\\
On morphisms $R: X^\tau \rightarrow Y^\sigma$, the corresponding partial frame morphism $\hat{\tau} \leftarrow \hat{\sigma}$ (notice the direction reversal for \textbf{Loc}), we define to be $\{(U_{\in \sigma},R^\dagger(U)_{\in \tau}) \ | \ U \in \sigma\}$. We ascertain that this is (1) a function that is (2) a frame homomorphism. For (1), since the relational converse picks out precisely one subset given any subset as input, these pairs do define a function. For (2), we observe that the relational converse (as all relations) preserve arbitrary unions and intersections, i.e. $R^\dagger(\bigcap\limits_i U_i) = \bigcap\limits_i R^\dagger(U_i)$ and $R^\dagger(\bigcup\limits_i U_i) = \bigcup\limits_i R^\dagger(U_i)$, so we do have a frame homomorphism.\\
Associativity follows easily.
\end{defn}

\begin{defn}[$F: \textbf{Loc} \rightarrow \textbf{TopRel}$]
On objects, $F$ sends a frame $\mathfrak{F}$ to the space
\end{defn}

\subsection{Why not Span(\textbf{Top})?}

One common generalisation of relations is to take spans of monics in the base category []. This actually produces a different category than the one we have defined. Below is an example of a span of monic continuous functions from \textbf{Top} that corresponds to a relation that doesn't live in \textbf{TopRel}. It is the span with the singleton as apex, with maps from the singleton to the "closed point" of two Sierpi\'{n}ski spaces.

\[\tikzfig{topology/spanctrex}\]

\subsection{Why not a Kliesli construction?}

Only because I am too stupid to think of the correct monad to use.