\section{The category \textbf{TopRel}}

\begin{proposition}
continuous relations form a category $\mathbf{TopRel}$.
\begin{proof}
\newthought{Identities:} Identity relations, which are always topological.

\newthought{Composition:} The normal composition of relations. We verify that the composite $X^\tau \overset{R}{\rightarrow} Y^\sigma \overset{S}{\rightarrow} Z^\theta$ of continuous relations is again continuous as follows:
\[U \in \theta \implies S^\dag(U) \in \sigma \implies R^\dag \circ S^\dag(U) = (S \circ R)^\dag \in \tau\]

\newthought{Associativity of composition:} Inherited from \textbf{Rel}.
\end{proof}
\end{proposition}

\section{Biproducts}

We exhibit a free-forgetful adjunction between \textbf{Rel} and \textbf{TopRel}.

\begin{defn}[F: $\mathbf{Rel} \rightarrow \mathbf{TopRel}$] We define the action of the functor $F$:
\begin{description}
\item[On objects] $F(X) := X^\star$, ($X$ with the discrete topology)
\item[On morphisms] $F(X \overset{R}{\rightarrow} Y) := X^\star \overset{R}{\rightarrow} Y^\star$. Recall that any relation between sets is continuous with respect to the discrete topology.
\end{description}
Evidently identities and associativity of composition are preserved.
\end{defn}

\begin{defn}[U: $\mathbf{TopRel} \rightarrow \mathbf{Rel}$]
\begin{description} We define the action of the functor $U$:
\item[On objects] $U(X^\tau) := X$
\item[On morphisms] $U(X^\tau \overset{R}{\rightarrow} Y^\sigma) := X \overset{R}{\rightarrow} Y$
\end{description}
Evidently identities and associativity of composition are preserved.
\end{defn}

\begin{proposition}[$F \dashv U \dashv F$]
\begin{proof}

By triangular identities.

The composite $FU$ is precisely equal to the identity functor on $\mathbf{Rel}$. The unit natural transformation $1_\mathbf{Rel} \Rightarrow FU$ we take to be the identity morphisms.

\[\eta_{X} := \text{id}_{X}\]

The counit natural transformation $UF \Rightarrow 1_{\mathbf{TopRel}}$ we define:

\[\epsilon_{X^\tau} : X^\star \rightarrow X^\tau := \{(x,x) : x \in X\}\]

Now we verify the triangle identities...

\[placeholder\]

\end{proof}
\end{proposition}

\begin{corollary}
$\mathbf{TopRel}$ has a zero object and biproducts.
\begin{proof}

\end{proof}
\end{corollary}

\section{Symmetric Monoidal Closed structure}

\marginnote{
\begin{rem}[Product Topology]
We denote the product topology of $X^\tau$ and $Y^\sigma$ as $(X \times Y)^{(\tau \times \sigma)}$. $\tau \times \sigma$ is the topology on $X \times Y$ generated by the basis $\{t \times s : t \in \mathfrak{b}_\tau, s \in \mathfrak{b}_\sigma\}$, where $\mathfrak{b}_\tau$ and $\mathfrak{b}_\sigma$ are bases for $\tau$ and $\sigma$ respectively.
\end{rem}
}

\begin{proposition}
$(\mathbf{TopRel},\{\star\}^{\bullet},X^\tau \otimes Y^\sigma := (X \times Y)^{(\tau \times \sigma)})$ is a symmetric monoidal closed category.
\begin{proof}

\newthought{Tensor Unit:} The one-point space $\bullet$. Explicitly, $\{\star\}$ with topology $\{\varnothing,\{\star\}\}$.

\marginnote{
    \begin{rem}[Product of relations]
    For relations between sets $R: X \rightarrow Y, S: A \rightarrow B$, the product relation $R \times S: X \times A \rightarrow Y \times B$ is defined to be \[ \{ ((x,a),(y,b)) : (x,y) \in R, (a,b) \in S \} \]
    \end{rem}
}

\newthought{Tensor Product:} For objects, $X^\tau \otimes Y^\sigma$ has base set $X \times Y$ equipped with the product topology $\tau \times \sigma$. For morphisms, $R \otimes S$ the product of relations. We show that the tensor of continuous relations is again a continuous relation. Take continuous relations $R: X^\tau \rightarrow Y^\sigma$, $S: A^\alpha \rightarrow B^\beta$, and let $U$ be open in the product topology $(\sigma \times \beta)$. We need to prove that $(R \times S)^\dag(U) \in (\tau \times \alpha)$. We may express $U$ as $\bigcup\limits_{i \in I} y_i \times b_i$, where the $y_i$ and $b_i$ are in the bases $\mathfrak{b}_\sigma$ and $\mathfrak{b}_\beta$ respectively. Since for any relations we have that $R(A \cup B) = R(A) \cup R(B)$ and $(R \times S)^\dag = R^\dag \times S^\dag$:

\begin{align*}
&(R \times S)^\dag(\bigcup\limits_{i \in I} y_i \times b_i)\\
 &= \bigcup\limits_{i \in I}(R \times S)^\dag(y_i \times b_i)\\
 &= \bigcup\limits_{i \in I}(R^\dag \times S^\dag)(y_i \times b_i)
 \end{align*}

Since each $y_i$ is open and $R$ is continuous, $R^\dag(y_i) \in \tau$. Symmetrically, $S^\dag(b_i) \in \alpha$. So each $(R^\dag \times S^\dag)(y_i \times b_i) \in (\tau \times \alpha)$. Topologies are closed under arbitrary union, so we are done.

\newthought{Unitors:} The left unitors $\lambda_{X^{\tau}}: \bullet \times X^\tau \rightarrow X^\tau$ maps $(\star,x) \mapsto x$, and we reverse the direction of the mapping to obtain the inverse $\lambda^{-1}_{X^{\tau}}$. The construction is symmetric for the right unitors $\rho_{X^{\tau}}$. 

\newthought{Associators:}

The associators $\alpha_{X^{\tau},Y^{\sigma},Z^{\rho}}$

\newthought{Braids:}

...

\newthought{Coherences:}

...

\end{proof}
\end{proposition}

\begin{fullwidth}

\section{\textbf{TopRel} diagrammatically}

\subsection{Relations that are always continuous}

\newthought{Here are five continuous relations for any $X^\tau$:}

\[\scalebox{0.75}{\tikzfig{bestiary/generators}}\]

\newthought{Copy and delete obey the following equalities:}

\[\scalebox{0.75}{\tikzfig{bestiary/basicrelations}}\]

\newthought{The copy map can also be used to distinguish the deterministic maps -- points and functions -- which we notate with an extra dot.}

\[\scalebox{0.75}{\tikzfig{structure/determinism}}\]

\newthought{Everything, delete, nothing-states and nothing-tests combine to give two numbers, one and zero.} There are extra expressions in grey squares above: they anticipate the tape-diagrams we will later use to graphically express another monoidal product of \textbf{TopRel}, the direct sum $\oplus$.

\[\scalebox{0.75}{\tikzfig{bestiary/scalarrelations}}\]

\newthought{Zero scalars turn entire diagrams into zero morphisms.} There is a zero-morphism for every input-output pair of objects in \textbf{TopRel}. 

\[\scalebox{0.75}{\tikzfig{bestiary/zerorelations}}\]

\end{fullwidth}