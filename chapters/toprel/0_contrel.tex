\section{Continuous Relations}

\newthought{To the best of my knowledge, the study of \textbf{ContRel} is a novel contribution. I venture two potential reasons.}

\newthought{First, it is because and not despite of the na\"{i}vity of the construction.} Usually, the relationship between \textbf{Rel} and \textbf{Set} is often understood in sophisticated general methods which are inappropriate in different ways. I have tried applying Kliesli machinery which generalises to "relationification" of arbitrary categories via appropriate analogs of the powerset monad to relate \textbf{Top} and \textbf{ContRel}, but it is not evident to me whether there is such a monad. The view of relations as spans of maps in the base category should work, since \textbf{Top} has pullbacks, but this makes calculation difficult and especially cumbersome when monoidal structure is involved. The na\"{i}ve approach I take is to observe that the preimages of functions are precisely relational converses when functions are viewed as relations, so the preimage-preserves-opens condition that defines continuous functions directly translates to the relational case.

\newthought{Second, the relational nature of \textbf{ContRel} means that the category has poor exactness properties.} Even if the sophisticated machinery mentioned in the first reason do manage to work, relational variants of \textbf{Top} are poor candidates for any kind of serious mathematics because they lack many limits and colimits. Since we take an entirely "monoidal" approach -- a relative newcomer in terms of mathematical technique -- we are able to find and make use of the rich structure of \textbf{ContRel} with a different toolkit.\\

In the end, we want to formalise doodles, so perhaps there is some virtue in proceeding by elementary means.

\marginnote{
\begin{rem}[Topological Space]
A \emph{topological space} is a pair $(X,\tau)$, where $X$ is a set, and $\tau \subset \mathcal{P}(X)$ are the \emph{open sets} of $X$, such that:
\begin{description}
    \item["nothing" and "everything" are open]  \[\varnothing,X \in \tau\]
    \item[Arbitrary unions of opens are open] \[\{ U_i : i \in I \} \subseteq \tau \Rightarrow \bigcup\limits_{i \in I} U_i \in \tau \]
    \item[Finite intersections of opens are open] $n \in \mathbb{N}$: \[U_1,\cdots, U_n \in \tau \Rightarrow \bigcap\limits_{1\cdots, i , \cdots n} U_i \in \tau\]
\end{description}
\end{rem}
}

\marginnote{
\begin{rem}[Relational Converse]
Recall that a relation $R: S \rightarrow T$ is a subset $R \subseteq S \times T$. \[R^\dag : T \rightarrow S := \{ (t,s) : (s,t) \in R \}\]
\end{rem}
}

\marginnote{
\begin{rem}[Continuous function]
A function between sets $f: X \rightarrow Y$ is a continuous function between topologies $f: (X,\tau) \rightarrow (Y,\sigma)$ if \[U \in \sigma \Rightarrow f^{-1}(U) \in \tau\] where $f^{-1}$ denotes the inverse image.
\end{rem}
}

Recall that functions are relations, and the inverse image used in the definition of continuous maps is equivalent to the relational converse when functions are viewed as relations. So we can na\"{i}vely extend the notion of continuous maps to continuous relations between topological spaces.

\begin{defn}[Continuous Relation]\label{defn:Contrelation}
A continuous relation $R: (X,\tau) \rightarrow (Y,\sigma)$ is a relation $R: X \rightarrow Y$ such that \[U \in \sigma \Rightarrow R^{\dag}(U) \in \tau\] where $\dag$ denotes the relational converse.
\end{defn}

\begin{notation}
For shorthand, we denote the topology $(X,\tau)$ as $X^{\tau}$. As special cases, we denote the discrete topology on $X$ as $X^{\star}$, and the indiscrete topology $X^{\circ}$.
\end{notation}

The symmetric monoidal structure is that of product topologies on objects, and products of relations on morphisms.

\marginnote{
\begin{rem}[Product Topology]
We denote the product topology of $X^\tau$ and $Y^\sigma$ as $(X \times Y)^{(\tau \times \sigma)}$. $\tau \times \sigma$ is the topology on $X \times Y$ generated by the basis $\{t \times s : t \in \mathfrak{b}_\tau, s \in \mathfrak{b}_\sigma\}$, where $\mathfrak{b}_\tau$ and $\mathfrak{b}_\sigma$ are bases for $\tau$ and $\sigma$ respectively.
\end{rem}
}

\marginnote{
    \begin{rem}[Product of relations]
    For relations between sets $R: X \rightarrow Y, S: A \rightarrow B$, the product relation $R \times S: X \times A \rightarrow Y \times B$ is defined to be \[ \{ ((x,a),(y,b)) : (x,y) \in R, (a,b) \in S \} \]
    \end{rem}
}

\section{\textbf{ContRel} diagrammatically}

\subsection{Relations that are always continuous}

\newthought{Here are five continuous relations for any $X^\tau$:}

\[\scalebox{0.75}{\tikzfig{bestiary/generators}}\]

\newthought{Copy and delete obey the following equalities:}

\[\scalebox{0.75}{\tikzfig{bestiary/basicrelations}}\]

\newthought{The copy map can also be used to distinguish the deterministic maps -- points and functions -- which we notate with an extra dot.}

\[\scalebox{0.75}{\tikzfig{structure/determinism}}\]

\newthought{Everything, delete, nothing-states and nothing-tests combine to give two numbers, one and zero.} There are extra expressions in grey squares above: they anticipate the tape-diagrams we will later use to graphically express another monoidal product of \textbf{ContRel}, the direct sum $\oplus$.

\[\scalebox{0.75}{\tikzfig{bestiary/scalarrelations}}\]

\newthought{Zero scalars turn entire diagrams into zero morphisms.} There is a zero-morphism for every input-output pair of objects in \textbf{ContRel}. 

\[\scalebox{0.75}{\tikzfig{bestiary/zerorelations}}\]

The reason for this (as is also the case in \textbf{Rel}, \textbf{Vect}, and any category with biproducts and zero-objects []) is that the zero scalar is an absorbing element of multiplication, and that multiplying any process by a zero scalar sends it to its corresponding zero-morphism.

\[\scalebox{0.75}{\tikzfig{bestiary/zeroscalar}}\]

So whenever a zero-process appears in a diagram, it spawns zero scalars which infect all other processes, turning them all into zero-processes. The same holds for whenever a zero-scalar appears; it makes copies of itself to infect all other processes.

\[\scalebox{0.75}{\tikzfig{bestiary/contagion}}\]