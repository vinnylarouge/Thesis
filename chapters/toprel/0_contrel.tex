\subsection{The category \textbf{ContRel}}

\marginnote{
The monoidal unit is the singleton space. Objects (wires) are topological spaces $X^\tau$, where $\tau$ is the topology. Here are 5 continuous relations for any $X^\tau$.
}

\begin{marginfigure}
\centering
\tikzfig{bestiary/everything}
\caption{The \emph{everything} state is the relation $\{((\star,x) \ | \ x \in X\}$, notated as above.}
\label{fig:everything}
\end{marginfigure}

\begin{marginfigure}
\centering
\tikzfig{bestiary/delete}
\caption{The \emph{delete} test, $\{((x,\star) \ | \ x \in X\}$.}
\label{fig:delete}
\end{marginfigure}

\begin{marginfigure}
\centering
\tikzfig{bestiary/copy}
\caption{The \emph{copy} map, $\{((x,\begin{bmatrix} x \\ x \end{bmatrix}) \ | \ x \in X\}$.}
\label{fig:delete}
\end{marginfigure}

\begin{defn}[\textbf{ContRel}]
The (purported) category \textbf{ContRel} has topological spaces for objects and continuous relations for morphisms.
\end{defn}

\begin{proposition}[\textbf{ContRel} is a category]
continuous relations form a category $\mathbf{ContRel}$.
\begin{proof}
\newthought{Identities:} Identity relations, which are always continuous since the preimage of an open $U$ is itself.

\newthought{Composition:} The normal composition of relations. We verify that the composite $X^\tau \overset{R}{\rightarrow} Y^\sigma \overset{S}{\rightarrow} Z^\theta$ of continuous relations is again continuous as follows:
\[U \in \theta \implies S^\dag(U) \in \sigma \implies R^\dag \circ S^\dag(U) = (S \circ R)^\dag \in \tau\]

\newthought{Associativity of composition:} Inherited from \textbf{Rel}.
\end{proof}
\end{proposition}

\subsection{Symmetric Monoidal structure}

\begin{proposition}
$(\mathbf{ContRel},\bullet,X^\tau \otimes Y^\sigma := (X \times Y)^{(\tau \times \sigma)})$ is a symmetric monoidal closed category.
\begin{proof}

\newthought{Tensor Unit:} The one-point space $\bullet$. Explicitly, $\{\star\}$ with topology $\{\varnothing,\{\star\}\}$.

\newthought{Tensor Product:} For objects, $X^\tau \otimes Y^\sigma$ has base set $X \times Y$ equipped with the product topology $\tau \times \sigma$. For morphisms, $R \otimes S$ the product of relations. We show that the tensor of continuous relations is again a continuous relation. Take continuous relations $R: X^\tau \rightarrow Y^\sigma$, $S: A^\alpha \rightarrow B^\beta$, and let $U$ be open in the product topology $(\sigma \times \beta)$. We need to prove that $(R \times S)^\dag(U) \in (\tau \times \alpha)$. We may express $U$ as $\bigcup\limits_{i \in I} y_i \times b_i$, where the $y_i$ and $b_i$ are in the bases $\mathfrak{b}_\sigma$ and $\mathfrak{b}_\beta$ respectively. Since for any relations we have that $R(A \cup B) = R(A) \cup R(B)$ and $(R \times S)^\dag = R^\dag \times S^\dag$:
\begin{align*}
&(R \times S)^\dag(\bigcup\limits_{i \in I} y_i \times b_i)\\
 &= \bigcup\limits_{i \in I}(R \times S)^\dag(y_i \times b_i)\\
 &= \bigcup\limits_{i \in I}(R^\dag \times S^\dag)(y_i \times b_i)
 \end{align*}
Since each $y_i$ is open and $R$ is continuous, $R^\dag(y_i) \in \tau$. Symmetrically, $S^\dag(b_i) \in \alpha$. So each $(R^\dag \times S^\dag)(y_i \times b_i) \in (\tau \times \alpha)$. Topologies are closed under arbitrary union, so we are done.

\newthought{The natural isomorphisms are inherited from \textbf{Rel}}. We will be explicit with the unitor, but for the rest, we will only check that the usual isomorphisms from \textbf{Rel} are continuous in \textbf{ContRel}. To avoid bracket-glut, we will vertically stack some tensored expressions.

\newthought{Unitors:} The left unitors are defined as the relations $\lambda_{X^{\tau}}: \bullet \times X^\tau \rightarrow X^\tau := \{(\begin{pmatrix}\star \\x \end{pmatrix}, x) \ | \ x \in X\}$, and we reverse the pairs to obtain the inverse $\lambda^{-1}_{X^{\tau}}$. These relations are continuous since the product topology of $\tau$ with the singleton is homeomorphic to $\tau$: $U \in \tau \iff (\bullet,U) \in (\bullet \times \tau)$. These relations are evidently inverses that compose to the identity. The construction is symmetric for the right unitors $\rho_{X^{\tau}}$.

\newthought{Associators:}
The associators $\alpha_{X^{\tau}Y^{\sigma}Z^{\rho}} : ((X \times Y) \times Z)^{((\tau \times \sigma) \times \rho)} \rightarrow (X \times (Y \times Z))^{(\tau \times (\sigma \times \rho))}$ are inherited from \textbf{Rel}. They are:
\[\alpha_{X^{\tau}Y^{\sigma}Z^{\rho}} := \{\big( \ (\begin{pmatrix} x \\ y \end{pmatrix} , z) \ , \ (x, \begin{pmatrix} y \\ z \end{pmatrix}) \big) \quad | \quad x \in X \ , \ y \in Y \ ,\ z \in Z \}\]
To check the continuity of the associator, observe that product topologies are isomorphic in \textbf{Top} up to bracketing, and these isomorphisms are inherited by \textbf{ContRel}. The inverse of the associator has the pairs of the relation reversed and is evidently an inverse that composes to the identity.

\newthought{Braids:}
The braidings $\theta_{X^{\tau}Y^{\sigma}} : (X \times Y)^{\tau \times \sigma} \rightarrow (Y \times X)^{\sigma \times \tau}$ are defined:
\[\{(\begin{pmatrix} x \\ y \end{pmatrix} \ , \ \begin{pmatrix} y \\ x \end{pmatrix}) \quad | \quad x \in X \ , \ y \in Y  \}\]
The braidings inherit continuity from the isomorphisms between $X^\tau \times Y^\sigma$ and $Y^\sigma \times X^\tau$ in \textbf{Top}. They inherit everything else from \textbf{Rel}

\newthought{Coherences:}
Since we have verified all of the natural isomorphisms are continuous, it suffices to say that the coherences [] are inherited from the symmetric monoidal structure of \textbf{Rel} up to marking objects with topologies.
\end{proof}

\newthought{Monoidal Closure:}
Here is the evaluator.
\[placeholder\]
\end{proposition}

\subsection{Rig category structure}

\begin{defn}[Biproducts and zero objects]
A \emph{biproduct} is simultaneously a categorical product and coproduct. A \emph{zero object} is both an initial and a terminal object. \textbf{Rel} has biproducts (the coproduct of sets equipped with reversible injections) and a zero object (the empty set).
\end{defn}

\begin{proposition}
$\mathbf{ContRel}$ has a zero object.
\begin{proof}
As in \textbf{Rel}, there is a unique relation from every object to and from the empty set with the empty topology.
\end{proof}
\end{proposition}

\begin{proposition}
$\mathbf{ContRel}$ has biproducts.
\begin{proof}
The biproduct of topologies $X^\tau$ and $Y^\sigma$ is their direct sum topology $(X \sqcup Y)^{(\tau + \sigma)}$ -- the coarsest topology that contains the disjoint union $\tau \sqcup \sigma$. As in \textbf{Rel}, the (in/pro)jections are partial identities, which are continuous by construction. To verify that it is a coproduct, given continuous relations $R: X^\tau \rightarrow Z^\rho$ and $S: Y^\sigma \rightarrow Z^\rho$, where the disjoint union $X \sqcup Y$ of sets is $\{x_1 \ | \ x \in X\} \cup \{y_2 \ | \ y \in Y\}$, we observe that $R + S := \{ (x_1,z) \ | \ (x,z) \in R \} \cup \{ (y_2,z) \ | \ y \in S \}$ is continuous and commutes with the injections as required. The argument that it is a product is symmetric.
\end{proof}
\end{proposition}

\begin{remark}
Biproducts yield another symmetric monoidal structure which the $\times$ monoidal product distributes over appropriately to yield a rig category. Throughout the chapter we have been using $\cup$, but we could have also "diagrammatised" $\cup$ by treating it as a monoid internal to \textbf{ContRel} viewed as a symmetric monoidal category with respect to the biproduct. There are two diagrammatic formalisms for rig categories that we could have used, [] and []. Neither case is perfectly suitable due to the fact that we sometimes took unions over arbitrary indexing sets, which is alright in topology but not depictable as a finite diagram in the $\oplus$-structure. A neat fact that follows is that a topological space is compact precisely when any arbitrarily indexed $\cup$ of tests in the $\times$-structure is \emph{depictable} in the $\oplus$-structure of either diagrammatic calculus for rig categories. \textbf{FdHilb} also has a monoidal product notated $\otimes$ that distributes over the monoidal structure given by biproducts $\oplus$. In contrast, we have used $\times$ -- the cartesian product notation -- for the monoidal product of \textbf{ContRel} since that is closer to what is familiar for sets.
\end{remark}

\section{\textbf{ContRel} diagrammatically}

\subsection{Relations that are always continuous}

\newthought{Here are five continuous relations for any $X^\tau$:}

\[\scalebox{0.75}{\tikzfig{bestiary/generators}}\]

\newthought{Copy and delete obey the following equalities:}

\[\scalebox{0.75}{\tikzfig{bestiary/basicrelations}}\]

\newthought{The copy map can also be used to distinguish the deterministic maps -- points and functions -- which we notate with an extra dot.}

\[\scalebox{0.75}{\tikzfig{structure/determinism}}\]

\newthought{Everything, delete, nothing-states and nothing-tests combine to give two numbers, one and zero.} There are extra expressions in grey squares above: they anticipate the tape-diagrams we will later use to graphically express another monoidal product of \textbf{ContRel}, the direct sum $\oplus$.

\[\scalebox{0.75}{\tikzfig{bestiary/scalarrelations}}\]

\newthought{Zero scalars turn entire diagrams into zero morphisms.} There is a zero-morphism for every input-output pair of objects in \textbf{ContRel}. 

\[\scalebox{0.75}{\tikzfig{bestiary/zerorelations}}\]

The reason for this (as is also the case in \textbf{Rel}, \textbf{Vect}, and any category with biproducts and zero-objects []) is that the zero scalar is an absorbing element of multiplication, and that multiplying any process by a zero scalar sends it to its corresponding zero-morphism.

\[\scalebox{0.75}{\tikzfig{bestiary/zeroscalar}}\]

So whenever a zero-process appears in a diagram, it spawns zero scalars which infect all other processes, turning them all into zero-processes. The same holds for whenever a zero-scalar appears; it makes copies of itself to infect all other processes.

\[\scalebox{0.75}{\tikzfig{bestiary/contagion}}\]