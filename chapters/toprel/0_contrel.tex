\section{Continuous Relations}

\marginnote{
\begin{rem}[Topological Space]
A \emph{topological space} is a pair $(X,\tau)$, where $X$ is a set, and $\tau \subset \mathcal{P}(X)$ are the \emph{open sets} of $X$, such that:
\begin{description}
    \item["nothing" and "everything" are open]  \[\varnothing,X \in \tau\]
    \item[Arbitrary unions of opens are open] \[\{ U_i : i \in I \} \subseteq \tau \Rightarrow \bigcup\limits_{i \in I} U_i \in \tau \]
    \item[Finite intersections of opens are open] $n \in \mathbb{N}$: \[U_1,\cdots, U_n \in \tau \Rightarrow \bigcap\limits_{1\cdots, i , \cdots n} U_i \in \tau\]
\end{description}
\end{rem}
}

\marginnote{
\begin{rem}[Relational Converse]
Recall that a relation $R: S \rightarrow T$ is a subset $R \subseteq S \times T$. \[R^\dag : T \rightarrow S := \{ (t,s) : (s,t) \in R \}\]
\end{rem}
}

\marginnote{
\begin{rem}[Continuous function]
A function between sets $f: X \rightarrow Y$ is a continuous function between topologies $f: (X,\tau) \rightarrow (Y,\sigma)$ if \[U \in \sigma \Rightarrow f^{-1}(U) \in \tau\] where $f^{-1}$ denotes the inverse image.
\end{rem}
}

Recall that functions are relations, and the inverse image used in the definition of continuous maps is equivalent to the relational converse when functions are viewed as relations. So we can na\"{i}vely extend the notion of continuous maps to continuous relations between topological spaces.

\begin{defn}[Continuous Relation]\label{defn:toprelation}
A continuous relation $R: (X,\tau) \rightarrow (Y,\sigma)$ is a relation $R: X \rightarrow Y$ such that \[U \in \sigma \Rightarrow R^{\dag}(U) \in \tau\] where $\dag$ denotes the relational converse.
\end{defn}

For shorthand, we denote the topology $(X,\tau)$ as $X^{\tau}$. As special cases, we denote the discrete topology on $X$ as $X^{\bullet}$, and the indiscrete topology $X^{\circ}$.