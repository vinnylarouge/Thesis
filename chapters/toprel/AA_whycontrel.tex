\section{Continuous Relations: A model setting for text circuits}

\begin{fullwidth}

In this chapter, we introduce \emph{continuous relations}, which are a na\"{i}ve extension of the category \textbf{Top} of topological spaces and continuous functions towards continuous relations. We choose this category (as opposed to plain \textbf{Rel}, the category of sets and relations) because it satisfies several requirements arrived at by introspection of some of the demands of modelling language. These justifications involve basic reflections upon language use, and the consequent mathematical constraints those affordances impose on any interpretation of text circuits in a symmetric monoidal category; A priori it could well be that there is no non-trivial process theory that satisfies these constraints, so the onus is on us to show that there exists such a process theory. I outline these justifications after this subsection.\\

Second, we introduce \textbf{TopRel}, demonstrate that it is a symmetric monoidal category, and we relate it to the well studied categories \textbf{Rel} and \textbf{Loc}. To the best of my knowledge, the study of this category is a novel contribution, reasons I list prefacing the bookwork. After the bookwork calculations, which may be skipped if too soporific, we introduce the diagrammatic calculus of \textbf{TopRel} along with worked examples as a more grounded alternative to familiarise the reader with the category.\\

Third, once we have defined a stage to perform calculations in \textbf{TopRel}, our aim is to introduce some actors. We seek to make formal the kinds of informal schemata we might doodle on paper to animate various processes occurring in space. One good reason for doing this is to establish formal foundations for the semantics of metaphor, some of the most commonly used of which involve spatial processes in a way that is fundamentally topological []. For example, in the \emph{conduit metaphor} [], \texttt{words} are considered \emph{containers} for \texttt{ideas}, and \texttt{communication} is considered a \emph{conduit} along which those containers are sent.

\[\scalebox{0.75}{\tikzfig{topology/conduitmetaphor}}\]

If you are already happy to treat such doodles as formal, then I think you're alright, you can skip the rest of this chapter. If you are a category theorist or just curious, hello, how are you, please do write me to share your thoughts if you care to, and please skip the rest of this paragraph. If you are still reading, I assume you are some kind of smelly epistemic-paranoiac Bourbaki-thrall sets-and-lambdas math-phallus-worshipping truth-condition-blinded symbol-pusher who takes things too seriously. I want you to know that I feel pity and disdain for you with the same flaccid energy anyone could muster to care about you, and I promise you that my feelings towards you are still warmer than the oblivion of irrelevance that awaits us all. I hope you are alienated enough to stop reading because you would have choked on the mathematics in this chapter anyway. I will begin the intimidation immediately.\\

We provide a generalisation (Definition \ref{defn:stickyspider}) of special commutative frobenius algebras in \textbf{TopRel} that cohere with idempotents in the category. The relation of ($\dagger$)-special commutative algebras in \textbf{FdHilb} to model observables in quantum mechanics is well-studied [], as is the role of idempotents in generalisations of quantum logic to arbitrary categories [], therefore this generalisation may be viewed as a unification of these ideas to define doodles in \textbf{TopRel}. N.B. we are not quite taking the Karoubi envelope of \textbf{TopRel}, as we are restricting the idempotents we wish to consider to only those that also behave appropriately as observables. The reason for this restriction lies in Theorem \ref{thm:stickygraphical} which provides a diagrammatic characterisation result that allows us to precisely identify when any idempotent in the category splits though a discrete topology. The discrete topology thus acts as a set of labels, where the (pre)images of each element under section and retract behave as the kind of shapes we wish to consider. As an interlude, we demonstrate how we may construct families of such idempotent-coherent special commutative frobenius algebras on $\mathbb{R}^2$ to provide a monoidal generalisation (c.f. the monoidal computer framework in []) of \textbf{FinRel} equipped with a Turing object [], thus satisfying Justification \ref{just:3}. Then we proceed to define configuration spaces of collections of shapes up to rigid displacement, and we develop a relational analogue of homotopy to model motions as paths in configuration space, along with appropriate extensions to accommodate nonrigid phenomena. If you are still reading and not already a category theorist nor just curious, I will drop the jargon now, because you are probably either questioning or deeply committed to your false ideals, both of which I respect, but just to spite the latter, I will proceed to do everything diagrammatically as much as possible.

\subsection{Justification 1: bookwork demonstration}

First, and simplest, is so that we have an excuse to build up a symmetric monoidal category from scratch, just to see what formal work is involved in doing so.\\



\subsection{Justification 3: we need both discrete symbolic and spatial conceptual behaviours}\label{just:rel}

Third, we have a cognitive consideration: how do we move between concepts -- however they are represented -- and symbolic representation and manipulation? Here we sidestep the debate around what concepts are, aligning ourselves close to G\"{a}rdenfors: we assume that there are \emph{conceptual spaces} that organise concepts of similar domain, and that regions of these spaces correspond to concepts. G\"{a}rdenfors' stance is backed by empirical data [], but even if he is wrong, he is at least interestingly so for our purposes.\\

The classic example of colour space is also one of the best studied and implemented: there are many different embeddings of the space of visible colours in Euclidean space []. In this setting we can mathematically model the action of categorising a particular point in colour space as \texttt{blue} by checking to see whether that point falls within the region in colourspace that the symbol \texttt{blue} is associated with. So we find that this view is conducive to modelling concepts as spatial entities -- a very permissive and expressive framework, which will allow us to calculate interesting things -- whilst also having the ability to handle them using symbolic labels.\\

But once we have a stock of symbols referring to entities in space, we can start talking about pairs of entities (e.g. \texttt{red or blue}), subsets of sets of entities (e.g. \texttt{autumnal colours}), arbitrary relations between the set of entities in one space and entities of another (e.g. how the colour of a banana relates to its probable textures and tastes). Given enough time and patience, we can linguistically construct -- at least -- any finite relation between finite sets. So we are asking for the following:

\begin{requirement}

\end{requirement}

\subsection{Justification 4: topology is a good setting to model concepts spatially}

Where we differ from G\"{a}rdenfors' is that we only ask for topological spaces, rather than the stronger notion of metric spaces. There are several reasons for this choice. First, topology is a basic mathematical framework for space. Second, we can view topology as a framework for conceptual spaces, where we consider the open sets of a topology to be the concepts. On this latter point, \'{E}scardo provides a the following correspondence [], which we extend with "Point" and "Subset", and an additional column for interpretation as conceptual space:

\begin{table}[]
\begin{tabular}{|c|c|c|}
\hline
\textbf{Topology} & \textbf{Type Theory} & \textbf{Conceptual Spaces}  \\ \hline
Space & Type & Conceptual space \\  \hline
Point & Element of set & Copyable instance \\  \hline
Subset & Subset & Instance \\  \hline
Open set & Semi-decidable set & Concept \\ \hline
Closed set & Set with semi-decidable complement & - \\ \hline
Clopen set & Decidable set & A concept the negation of which is also a concept \\ \hline
Discrete topology & Type with decidable equality & A conceptual space where any collection of instances forms a concept \\ \hline
Haussdorf topology & Type with semi-decidable inequality & A conceptual space where any pair of distinct instances can be described as belonging to two disjoint concepts \\ \hline
Compact set & Exhaustively searchable set, in a finite number of steps & A conceptual space $\mathfrak{C}$ such that for any joint concept $R$ on $\mathfrak{C}$ and another conceptual space $\mathfrak{D}$, $\forall c_{ \in \mathfrak{C}}R(c,-)$ is a concept in $\mathfrak{D}$ \\ \hline
\end{tabular}
\end{table}

In \textbf{TopRel}, open sets are precisely tests. Modelling concepts as open sets or tests aligns them with semi-decidability. Given any state-instance, we can test whether it overlaps with a concept graphically: success returns a unit scalar, failure returns the zero scalar.

\end{fullwidth}
