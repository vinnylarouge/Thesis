\section{Continuous Relations for iconic semantics}

Sometimes it helpful to illustrate concepts using iconic representations in cartoons. For instance in the \emph{conduit metaphor} [], \texttt{words} are considered \emph{containers} for \texttt{ideas}, and \texttt{communication} is considered a \emph{conduit} along which those containers are sent.

\[\resizebox{1.5\textwidth}{!}{\tikzfig{topology/conduitmetaphor}}\]

The aim of this chapter is to formalise cartoon doodles as in Figure \ref{fig:conduit} in a symmetric monoidal category so that we can give semantics to text circuits in terms of graphical, iconic representations. To do so, we introduce the category \textbf{ContRel} of \emph{continuous relations}, which are a na\"{i}ve extension of the category \textbf{Top} of topological spaces and continuous functions towards continuous relations.

The main reason we prefer \textbf{ContRel} to either \textbf{Rel} or \textbf{Top} for our purposes is that we can diagrammatically characterise set-indexed collections of mutually disjoint open sets as \emph{sticky-spiders}: a generalisation of spiders that interact with idempotents. We can then treat the indexing set as a collection of labels, and an indexed open set as a doodle. Notably, spiders don't exist in cartesian \textbf{Top} except for the one-point space, and the spatial structure of open sets doesn't exist in \textbf{Rel}.

\[placeholder: stickyspider laws\]

But there are all kinds of poorly behaved open sets even on the plane, so enter the next benefit: In \textbf{ContRel}, we can diagrammatically characterise the reals as a topological space up to homeomorphism, which gives us a diagrammatic handle on paths and homotopies, mathematical concepts that enable us to diagrammatically characterise when open sets are connected, how they might move and transform continuously in space, and when open sets are contained inside others.

\[real characterisation\]

\newpage

Second, we explore \textbf{ContRel} diagrammatically. In the appendix for this chapter we do the bookwork demonstrating that it is a symmetric monoidal category, and we relate it to the well studied categories \textbf{Rel} and \textbf{Loc}. To the best of my knowledge, the study of this category is a novel contribution, for reasons I list prefacing the bookwork.\\

Third, once we have defined a stage to perform calculations in \textbf{ContRel}, . We seek to make formal the kinds of informal schemata we might doodle on paper to animate various processes occurring in space. One good reason for doing this is to establish formal foundations for the semantics of metaphor, some of the most commonly used of which involve spatial processes in a way that is fundamentally topological []. 



We provide a generalisation (Definition \ref{defn:stickyspider}) of special commutative frobenius algebras in \textbf{ContRel} that cohere with idempotents in the category. The relation of ($\dagger$)-special commutative algebras in \textbf{FdHilb} to model observables in quantum mechanics is well-studied [], as is the role of idempotents in generalisations of quantum logic to arbitrary categories [], therefore this generalisation may be viewed as a unification of these ideas to define doodles in \textbf{ContRel}. N.B. we are not quite taking the Karoubi envelope of \textbf{ContRel}, as we are restricting the idempotents we wish to consider to only those that also behave appropriately as observables. The reason for this restriction lies in Theorem \ref{thm:stickygraphical} which provides a diagrammatic characterisation result that allows us to precisely identify when any idempotent in the category splits though a discrete topology. The discrete topology thus acts as a set of labels, where the (pre)images of each element under section and retract behave as the kind of shapes we wish to consider. As an interlude, we demonstrate how we may construct families of such idempotent-coherent special commutative frobenius algebras on $\mathbb{R}^2$ to provide a monoidal generalisation (c.f. the monoidal computer framework in []) of \textbf{FinRel} equipped with a Turing object [], thus satisfying Justification \ref{just:3}. Then we proceed to define configuration spaces of collections of shapes up to rigid displacement, and we develop a relational analogue of homotopy to model motions as paths in configuration space, along with appropriate extensions to accommodate nonrigid phenomena.

\subsection{Why not use something already out there?}

\newthought{Justification 1: We want a symmetric monoidal category.} We want to work process-theoretically as much as possible, because, as we have seen, this potentially allows whatever we construct in this setting to generalise to other process theories. Also, we want an excuse to build up and play with a symmetric monoidal category from scratch, just to see what formal work is involved in doing so.

\newthought{Justification 2: We want to model concepts.}\label{just:rel} Second, we have some cognitive considerations: how do we move between concepts -- however they are represented -- and symbolic representation and manipulation? Here we sidestep the debate around what concepts actually \emph{are}, aligning ourselves close to G\"{a}rdenfors: we assume that there are \emph{conceptual spaces} that organise concepts of similar domain -- such as colour, taste, motion -- and that regions of these spaces correspond to concepts. G\"{a}rdenfors' stance is backed by empirical data [], but even if he is wrong, he is at least interestingly so for our purposes. As an example, the classic example of colourspace is also one of the best studied and implemented: there are many different embeddings of the space of visible colours in Euclidean space []. In this setting we can mathematically model the action of categorising a particular point in colour space as \texttt{blue} by checking to see whether that point falls within the region in colourspace that the symbol \texttt{blue} is associated with. So we find that this view is conducive to modelling concepts as spatial entities -- a very permissive and expressive framework, which will allow us to calculate interesting things -- whilst also having the ability to handle them using symbolic labels. The question remains: how do we model this association between space and sets of labels? This leads us to the following consideration: In a category for concept-compliant text-circuits $\mathcal{T}$, any conceptual space object $\Gamma$ should possess a split idempotent through a set of concept-labels, thus encoding the association between concepts-qua-spaces and concepts-qua-symbols.\\

A further complication arises. Once we have a stock of symbols referring to entities in space, we can start talking about pairs of entities (e.g. \texttt{red or blue}), subsets of sets of entities (e.g. \texttt{autumnal colours}), arbitrary relations between the set of entities in one space and entities of another (e.g. how the colour of a banana relates to its probable textures and tastes). Given enough time and patience, we can linguistically construct -- at least -- any finite relation between finite sets. As we will elaborate in Section \ref{sec:lassos}, the real challenge is that every time we define a new concept in this way, we can again treat it as a symbolic concept, that is, label it with a noun. Since nouns are first-class citizens that travel along wires in our framework, we are asking that any putative process theory that satisfies these considerations about concepts has to have some object, some wire $\Xi$ for nouns, such that all finite relations fit into it. This leads us to the following consideration: A category for concept-compliant text-circuits $\mathcal{T}$ ought to have an object $\Xi$ such that all of \textbf{FinRel} can be encoded within $\Xi$ somehow. But we already know how to encode using split idempotents, so we can translate the two considerations of the last two paragraphs into one requirement. In prose; the first item gives us a method within the category $\mathcal{T}$ to associate concepts-qua-spaces to concepts-qua-symbols; the second item gives us a noun-wire in $\mathcal{T}$ that permits us to encode and manipulate concepts however we would like to treat finite relations.

\begin{requirement}
We call a text circuit category $\mathcal{T}$ \emph{concept-compliant} if:
\begin{enumerate}
\item{Any wire $\Gamma$ used to model a conceptual space possesses a split idempotent through a set of concept-symbols $\mathfrak{L}$}
\item{Anything one can do with sets of concept-symbols in \textbf{FinRel} must be doable in $\mathcal{T}$; in particular, there exists a noun-wire $\Xi$ such that $\textbf{FinRel}$ embeds as a category into the subcategory of $\mathcal{T}$ generated by the split idempotents on $\Xi$}
\end{enumerate}
\end{requirement}

\newthought{Justification 3: We think topology is a good setting to model concepts spatially.}

Where we differ from G\"{a}rdenfors is that we only ask for topological spaces, rather than his stronger requirements for metric spaces and convex concepts, so we are following the spirit but not the letter. There are several reasons for this choice. First, topology is a primitive mathematical framework for space; all metric spaces are topological spaces, but not vice versa, so this is a conservative generalisation of G\"{a}rdenfors. Second, there are technical reasons that \textbf{ContRel} is desirable, For example, we can process-theoretically characterise continuous maps from the unit interval fairly easily to model things going on in space and time; without topology around, for instance in the setting of just \textbf{Rel}, I do not know if it is even possible to pick out the continuous maps from all the others purely process-theoretically. Third and perhaps most interestingly, there are also good reasons to think that this is the right way to think about conceptual spaces. We can view topology as a framework for conceptual spaces where we consider the open sets of a topology to be the concepts. \'{E}scardo provides a the following correspondence [], which we extend with "Point" and "Subset", and an additional column for interpretation as conceptual space:

\begin{table}[h]
\begin{tabular}{|c|c|c|}
\hline
\textbf{Topology} & \textbf{Type Theory} & \textbf{Conceptual Spaces}  \\ \hline
Space & Type & Conceptual space \\  \hline
Point & Element of set & Copyable instance \\  \hline
Subset & Subset & Instance \\  \hline
Open set & Semi-decidable set & Concept \\ \hline
Closed set & Set with semi-decidable complement & - \\ \hline
Clopen set & Decidable set & A concept the negation of which is also a concept \\ \hline
Discrete topology & Type with decidable equality & A conceptual space where any collection of instances forms a concept \\ \hline
Haussdorf topology & Type with semi-decidable inequality of elements & A conceptual space where any pair of distinct instances can be described as belonging to two disjoint concepts \\ \hline
Compact set & Exhaustively searchable set, in a finite number of steps & A conceptual space $\mathfrak{C}$ such that for any joint concept $R$ on $\mathfrak{C} \times \mathfrak{D}$, $\forall c_{ \in \mathfrak{C}}R(c, - )$ is a concept in $\mathfrak{D}$ \\ \hline
\end{tabular}
\end{table}

\textbf{ContRel} reflects the above correspondences diagrammatically. Open sets are precisely tests, and it is always possible to construct finite intersections of opens using copy-relations. Modelling concepts as open sets or tests aligns them with semi-decidability. Given any state-instance, we can test whether it overlaps with a concept graphically: success returns a unit scalar, failure returns the zero scalar. Moreover, another independent diagrammatic calculus for concepts by Tull [] agrees with ours; concepts are effects, copy maps take intersections of concepts, and so on. Tull's diagrams are interpreted in \textbf{Stoch}, the category of stochastic processes, and as a whole his design philosophy closely adheres to G\"{a}rdenfors. All this is to suggest that if you take G\"{a}rdenfors seriously, then there is something worth taking seriously about modelling concepts as effects in a monoidal category with copy-maps. However, taking diagrammatic conceptual spaces seriously also yields a no-go result for topological spaces -- which includes G\"{a}rdenfors' metric spaces with convex concepts: you can only do first-order logic with equality on your concepts if your base space is discrete and finite. We explain this below.\\

The correspondence between spaces and type-theory extends to conceptual spaces as follows: "niceness conditions on your conceptual space correspond to the ability to form new concepts using logical operations." For example, this means that if we denote colourspace with $\mathfrak{C}$, we can only construct a concept \texttt{different colours} on $\mathfrak{C} \times \mathfrak{C}$ if we model $\mathfrak{C}$ using a Hausdorff space, such as Euclidean space. If we want to model \texttt{not X} as the complement of a colour \texttt{X} in colourspace, asking that \texttt{not X} also be a concept requires $\mathfrak{C}$ be locally indiscrete -- i.e. every open set is also closed; Euclidean space is not locally indiscrete, so we cannot use set-complement as negation, we have to use something else, like the interior of the complement. If we want to have access to universal quantifiers in colourspace, so that we can sensibly construct concepts such as \texttt{the taste that apples of all colours have in common}, then we require $\mathfrak{C}$ to be compact, which Euclidean space is not, but bounded Euclidean space is. Here then is the conflict: if we take modelling concepts with spaces seriously, and we also care to do logic with concepts, there are tradeoffs to be made. In order to take equality as a concept \texttt{same colour} on $\mathfrak{C} \times \mathfrak{C}$ requires that $\mathfrak{C}$ have discrete topology, but discrete topologies on infinite sets are not compact, so you cannot also have universal quantification at the same time. Conversely, if you want universal quantification on colourspace, then you can at best have approximate equality on colours as a concept, never exact. Now we can summarise this tension. All topological spaces, including G\"{a}rdenfors conceptual spaces with metrics and convex concepts, start off with regular logic -- $\exists, \wedge, \top$ -- for free []. In order to obtain first-order logic with equality on the points of the space, the underlying space must be compact (to support universal quantification) and discrete (to support equality of points), and the latter condition implies locally indiscrete (to support negation). Only finite sets with the discrete topology are compact; you can only do first-order logic with equality on your concepts if your base space is discrete and finite.\\

This no-go just provides another perspective of the ancient observation that logical thought really does seem symbolic. It is also important to note that the essential idea of conceptual spaces is to circumvent this no-go; in our language, asking for split idempotents through (finite) discrete topologies is precisely what allows logical constructions to work on spatial representations of concepts, when those split idempotents exist. What this no-go does mean is that nobody has to waste their time looking for \emph{precise} unifications of conceptual spaces and first-order logic-with-equality where every logical operator is viewed as a space-preserving transformation that sends concepts to concepts and behaves properly on the space of points; there is only "good enough".

\newthought{Summary of justifications.} We want a symmetric monoidal category to keep the prospect of general, process-theoretic reasoning. We want the objects of this category to be topological spaces because we want to model conceptual spaces and calculate interesting things. While the category \textbf{Top} of topological spaces and continuous functions is already symmetric monoidal with respect to categorical product, for linguistic and concept-related considerations we want finite relations to come into play diagrammatically, and because \textbf{Top} is cartesian monoidal it doesn't work for our purposes. So we construct \textbf{ContRel}.
