\section{Lassos for generalised anaphora}

\newthought{A problem arises when anything can be a noun wire.} By linguistic introspection, we realise we must account for \emph{Entification} and \emph{Processising} -- the process of turning non-nouns into noun-entities and back again. If we think about English, we find that just about any word can be turned into a noun and back again (e.g. \texttt{run} by gerund to \texttt{running}, \texttt{quick} by a suffix to \texttt{quickness}, and even entire sentences \texttt{Bob drinks Duvel} can become a noun \texttt{the fact that Bob drinks Duvel}).\\

This consideration carries some linguistic interest as well. In the usual treatment of anaphora resolution, pronouns refer to nouns, for instance: \texttt{Bob drinks a beer. \underline{It} is cold.}, where \texttt{it} refers to the beer. But there are situations where pronouns can point to textual data that are not nouns. For instance: \texttt{Jono was paid minimum wage. He didn't mind \underline{it}.}, where \texttt{it} would like to refer to something like \texttt{the fact that Jono was paid minimum wage}. While there are extensions of discourse reference theory to accommodate event structures [], the issue at hand is that pronouns in the appropriate context seem to be able to refer to \emph{any meaningful part of text}. For example, \texttt{The tiles were grey. \underline{It} was a particularly depressing shade.}, where \texttt{it} seems to refer just to the entified adjective \texttt{the greyness (of the tiles)}. Or, \texttt{Alice's cake was tastier than Bob's, but \underline{it} wasn't enough so for the judges to decide unanimously.}, where \emph{it} seems to refer the entified tastiness differential of \texttt{tastier}: \texttt{the difference in tastiness between Alice and Bob's cakes}.\\

Since we have so far built up a theory around noun-wires as first-class citizens, these observations present nontrivial mathematical constraints for interpretations of text circuits. Now we try to interpret these constraints in mathematical terms, staying within the graphical confines we have established in \textbf{TopRel} as much as possible. Let us denote the noun-wire type by $\Xi$. First we observe that any finite collection of noun wires $\bigotimes^n \Xi$ has to be \emph{encodable} in a single noun wire $\Xi$, because we can always interpose with \texttt{and}. We take this to mean that there will exist morphisms such that:

\[placeholder\]

Second, for any word-gate $w$ of grammatical type $\mathfrak{g}$, we ought to have noun-states and an evaluator process that witness entification and processising:

\[placeholder\]

Second-and-a-half, any morphism (or "meaningful part of text") $T \in \[\bigotimes^n \Xi,\bigotimes^m \Xi\]$ for any $n,m \in N$ -- has to be encodable as a state of $\Xi$. This is expressed as the following graphical condition:

\[placeholder\]

Condition two-and-a-half follows if the former conditions are met, provided that all text circuits are made up of a fixed stock of grammatical-gate-types:

\[placeholder\]

If we have all the above, then we can grab any part of a circuit and turn it into a noun. We can notate this using a \emph{lasso}.

\[\]

Recall that Lassos -- a graphical gadget that can encode arbitrary morphisms into a single wire -- can be interpreted in a monoidal computer. Recall that monoidal computers require a universal object $\Xi$. Here we show how in \textbf{TopRel}, by taking $\Xi := \squarehvfill$ the open unit square, we have a monoidal computer in \textbf{Rel} restricted to countable sets and the relations between them. We will make use of sticky spiders. We have to show that; $\squarehvfill$ has a sticky-spider corresponding to every countable set; how there is a suitable notion of sticky-spider morphism to establish a correspondence with relations; what the continuous relations are on $\squarehvfill$ that mimick various compositions of relations.

\begin{proposition}[$(0,1) \times (0,1)$ splits through any countable set $X$]
For any countable set $X$, the open unit square $\squarehvfill$ has a sticky spider that splits through $X^\star$.
\begin{proof}
The proof is by construction. We'll assume the sticky-spiders to be mereologies, so that cores and halos agree. Then we only have to highlight the copiable open sets. Take some circle and place axis-aligned open squares evenly along them, one for each element of $X$. The centres of the open squares lie on the circumference of the circle, and we may shrink each square as needed to fit all of them.
\[\scalebox{1}{\tikzfig{spatialencoding/circencodingconstruct}}\]
\end{proof}
\end{proposition}

\begin{defn}[Morphism of sticky spiders]
A morphism between sticky spiders is any morphism that satisfies the following equation.
\[\scalebox{1}{\tikzfig{spatialencoding/stickymorphismdefn}}\]
\end{defn}

\begin{proposition}[Morphisms of sticky spiders encode relations]
For arbitrary split idempotents through $A^\star$ and $B^\star$, the morphisms between the two resulting sticky spiders are in bijection with relations $R: A \rightarrow B$.
\[\scalebox{1}{\tikzfig{spatialencoding/arbsetclaim}}\]
\begin{proof}
\[\scalebox{1}{\tikzfig{spatialencoding/arbset}}\]
\end{proof}
\end{proposition}

\begin{construction}[Representing sets in their various guises within $\squarehvfill$]
We can represent the direct sum of two $\squarehvfill$-representations of sets as follows.
\[\scalebox{1}{\tikzfig{spatialencoding/directsumconstruct}}\]
The important bit of technology is the homeomorphism that losslessly squishes the whole unit square into one half of the unit square. The decompressions are partial continuous functions, with domain restricted to the appropriate half of the unit square.
\[\scalebox{1}{\tikzfig{spatialencoding/leftrightcompressions}}\]
We express the ability of these relations to encode and decode the unit square in just either half by the following graphical equations.
\[\scalebox{1}{\tikzfig{spatialencoding/leftrightcompressions2}}\]
Now, to put the two halves together and to take them apart, we introduce the following two relations. In tandem with the squishing and stretching we have defined, these will behave just as the projections and injections for the direct-sum biproduct in \textbf{Rel}.
\[\scalebox{1}{\tikzfig{spatialencoding/leftrightcompressions3}}\]
The following equation tells us that we can take any two representations in $\squarehvfill$, put them into a single copy of $\squarehvfill$, and take them out again. Banach and Tarski would approve.
\[\scalebox{1}{\tikzfig{spatialencoding/leftrightcompressions4}}\]
We encode the tensor product $A \otimes B$ of representations by placing copies of $B$ in each of the open boxes of $A$.
%\[\scalebox{0.75}{\tikzfig{spatialencoding/directsummap}}\]
%\[\scalebox{0.75}{\tikzfig{spatialencoding/directsummap2}}\]
\[\scalebox{1}{\tikzfig{spatialencoding/tensorconstruct}}\]
The important bit of technology here is a family of homeomorphisms of $\squarehvfill$ parameterised by axis-aligned open boxes. We depict the parameters outside the body of the homeomorphism for clarity. The squish is on the left, the stretch on the right.
\[\scalebox{1}{\tikzfig{spatialencoding/boxcompression}}\]
Now, for every representation of a set in $\squarehvfill$ by a sticky-spider, where each element corresponds to an axis-aligned open box, we can associate each element with a squish-stretch homeomorphism via the parameters of the open box, which we notate with a dot above the name of the element.
\[\scalebox{1}{\tikzfig{spatialencoding/obtainboxposition}}\]
Now we can define the "tensor $X$ on the left" relation $\_ \rightarrow X \otimes \_$ and its corresponding cotensor.
\[\scalebox{1}{\tikzfig{spatialencoding/tensordetensor}}\]
The tensor and cotensor behave as we expect from proof nets for monoidal categories.
\[\scalebox{1}{\tikzfig{spatialencoding/tensordetensor2}}\]
And by construction, the (co)tensors and (co)pluses interact as we expect, and they come with all the natural isomorphisms between representations we expect. For example, below we exhibit an explicit associator natural isomorphism between representations.
\[\scalebox{1}{\tikzfig{spatialencoding/tensordetensor3}}\]
\end{construction}

\begin{construction}[Representing relations between sets and their composition within $\squarehvfill$]
With all the above, we can establish a special kind of process-state duality; relations as processes are isomorphic to states of $\squarehvfill$, up to the representation scheme we have chosen.
\[\scalebox{1}{\tikzfig{spatialencoding/relcomp1}}\]
Moreover, we have continuous relations that perform sequental composition of relations.
\[\scalebox{1}{\tikzfig{spatialencoding/relcomp2}}\]
And we also know how to take the parallel composition of relations by tensors.
\[\scalebox{1}{\tikzfig{spatialencoding/relcomp3}}\]
\end{construction}

\end{fullwidth}