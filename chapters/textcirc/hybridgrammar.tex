\section{A hybrid grammar for text}\label{sec:grammar}
%\TODOb{Refer to this grammar as `our hybrid grammar', without a specific name. The connectedness graphs are the real deal, and maybe we should make that point explicitly clear.}

\begin{marginfigure}
\centering
\scalebox{0.8}{\tikzfig{textcirc/ex}}
\caption{We will depict derivations of strings as planar "trees". The diagrams are read from top to bottom.}
\end{marginfigure}

\begin{marginfigure}
\centering
\scalebox{0.8}{\tikzfig{textcirc/AgBtBcopy}}
\caption{These "trees" may have multiple edges from a parent node to a child node. We drop symbolic labels for intermediate symbols, and replace them by coloured edges. For example, \texttt{NP} becomes a black edge and \texttt{IVP} becomes a green edge. As we introduce the rules, we will also keep to a coloring convention for typed wires, such that later on we may omit typings such as \texttt{IVP} from diagrams without confusion.}
\end{marginfigure}

\begin{marginfigure}
\centering
\scalebox{0.8}{\tikzfig{textcirc/IVcsg}}
\caption{We now introduce a phrase structure grammar by giving the tree-fragments for the grammatical types, initially for what we call \emph{simple} sentences, which have a single verb that does not take a sentential complement. A simple sentence may contain a single \textbf{intransitive} or \textbf{transitive} verb. In the former case, the sentence consists of a noun-phrase followed by a intransitive-verb-phrase (e.g. \texttt{\underline{ALICE RUNS.}}). $\texttt{S} \mapsto \texttt{NP} \cdot \texttt{IVP}$}
\end{marginfigure}

\newthought{We now introduce our hybrid grammar} which is generative and aims to capture linguistic connectedness. We introduce the rules in the margin, saving the main body for worked examples. We develop the hybrid grammar in three main steps. First, we start with a context-sensitive grammar for simple sentences that only contain one verb, where the context-sensitivity is invoked for adpositions that modify verbs depending on whether they are transitive or intransitivity. Second, we introduce the notion of pronominal links, which identify recurring nouns, and pronouns with their referents. Further, we will introduce rules that allow us to fuse together simple sentences with recurring nouns via relative pronouns. Third, we introduce the notion of verbs that accept a sentential complement and phrase scope boundary. This expands our fragment to deal with compound sentences containing subphrases that are themselves sentences. Going forward, we refer to our hybrid grammar of this section as "hybrid grammar" or just "grammar" when there is no confusion.

\newthought{Our aim is to build a minimal grammar that allows us to generate text circuits from text and state crisp mathematical results.} To this end, we use a Frankensteinesque hybrid of basic ideas from different formalisms: we use Chomsky's transformational phrase structure grammars, adjusted with features of Lambek's pregroups; pronominal links are inspired by discourse representation theory, and phrase boundaries are inspired by dependency grammars. Section \ref{sec:discussion} outlines these relationships in more detail. For the sake of clarity, we do not deal with some grammatical phenomena (like tense), omit certain grammatical patterns (we assume adverbs always come before the verb), and only deal with part of language (we do not consider determiners and quantifiers here, and we assume that ditransitive verbs have equivalent presentations as transitive verbs with adpositions.) This approach also has the usual shortcomings of formal grammars -- such as not taking into account adjective order for purpose, origin, etc. in languages like English -- that can be dealt with in the usual ways, requiring grammar to be mixed up with meanings. In fact, DisCoCat/DisCoCirc are all about combining grammar and meaning, and we (as many others) believe that ultimately they shouldn't exist independently, but should mutually inform each other. Practical NLP has empirically shown that this is essential for producing efficient tools. Pronominal link and phrase boundary data are not uniquely determined by text when given as just a string of words, without any further context -- but then again, neither are grammatical types in many cases. The reason for including them is their necessity for obtaining \emph{disambiguated} text structure which we can reason about mathematically.\\

\newthought{We model phrase structure using a string-rewrite system.} Phrase structure is what constrains the order of words in a sentence. String rewrite systems consist of (a usually finite collection of) \emph{production rules} \cite{hopcroft_introduction_1979}:
\[  
\alpha \mapsto \beta
\]
where $\alpha$ and $\beta$ are strings of symbols.  In our case, they may be phrase components, such as \texttt{NP}, or English words, such as \texttt{\underline{BOB}}. We underline the latter in order to indicate that such symbols are \emph{terminal} i.e.~not rewriteable further by production rules.

\begin{marginfigure}
\centering
\scalebox{0.8}{\tikzfig{textcirc/TVcsgcopy}}
\caption{In the latter case, a sentence consists of a noun-phrase, transitive-verb-phrase, and another noun-phrase (e.g.~\texttt{\underline{ALICE LIKES BOB.}}). $\texttt{S} \mapsto \texttt{NP}_1 \cdot \texttt{TVP} \cdot \texttt{NP}_2$}
\end{marginfigure}

\begin{marginfigure}
\centering
\scalebox{0.8}{\tikzfig{textcirc/IVcsgL}}
\caption{There also are the terminal rules for verbs, where the terminal symbols of the grammar are verbs of the appropriate type e.g.~intransitive: $\texttt{IVP} \mapsto \texttt{\underline{IV}}$}
\end{marginfigure}

\begin{marginfigure}
\centering
\scalebox{0.8}{\tikzfig{textcirc/TVcsgL}}
\caption{Or transitive: $\texttt{TVP} \mapsto \texttt{\underline{TV}}$. Going forward, we omit the terminal rules in favour of giving examples of finished derivations, from which terminals can be inferred.}
\end{marginfigure}

\newthought{All such string-rewrite systems specify languages.} A language is viewed as a collection of strings of symbols as follows. A special \emph{start symbol} $\texttt{S}$ is specified, and the language associated with the rewrite-system is the collection of all strings of terminal symbols that can be produced by (finite) applications of the production rules available, for example, given rules:
\begin{align}
\texttt{S} \mapsto & \ \texttt{NP} \cdot \texttt{IVP}\label{rule1}\\   
\texttt{NP} \mapsto & \ \texttt{\underline{BOB}}\label{rule2}\\ 
\texttt{IVP} \mapsto & \ \texttt{\underline{DRINKS}}\label{rule3}
\end{align}
where $\cdot$ is notation for string concatenation, we can produce simple sentences such as:
\begin{align*}
\texttt{S} \overset{(\ref{rule1})}{\mapsto} & \ \texttt{NP} \cdot \texttt{IVP}\\
\overset{(\ref{rule2})}{\mapsto} & \ \texttt{\underline{BOB}} \cdot \texttt{IVP}\\
\overset{(\ref{rule3})}{\mapsto} & \ \texttt{\underline{BOB}} \cdot \texttt{\underline{DRINKS}}
\end{align*}

\begin{marginfigure}
\centering
\scalebox{0.8}{\tikzfig{textcirc/ADJcsg}}
\caption{\textbf{Adjectives} can appear before a noun-phrase (e.g.~\texttt{\underline{DRUNK HAPPY BOB.}}) $\texttt{NP} \mapsto \texttt{ADJ} \cdot \texttt{NP}$.}
\end{marginfigure}

\begin{marginfigure}
\centering
\scalebox{0.8}{\tikzfig{textcirc/ADJiscsg}}
\caption{Or, using the copular \texttt{\underline{IS}} considered as a verb, a single adjective can appear after a noun-phrase (e.g.~\texttt{\underline{BOB IS DRUNK.}}) $\texttt{S} \mapsto \texttt{NP} \cdot \texttt{\underline{IS}} \cdot \texttt{ADJ}$}
\end{marginfigure}

\begin{marginfigure}
\centering
\scalebox{0.8}{\tikzfig{textcirc/ADVIVcsg}}
\caption{\textbf{Adverbs} can appear before a verb (e.g. \texttt{\underline{ALICE QUICKLY HAPPILY RUNS.}}) $\texttt{IVP} \mapsto \texttt{ADV} \cdot \texttt{IVP}$}
\end{marginfigure}

\begin{marginfigure}
\centering
\scalebox{0.8}{\tikzfig{textcirc/ADVTVcsg}}
\caption{$\texttt{TVP} \mapsto \texttt{ADV} \cdot \texttt{TVP}$}
\end{marginfigure}

\begin{marginfigure}
\centering
\scalebox{0.8}{\tikzfig{textcirc/ADPIVcsg}}
\caption{\textbf{Adpositions} can appear to the right of an intransitive-verb-phrase, followed by a noun-phrase (e.g.~from \texttt{\underline{ALICE RUNS.}} to \texttt{\underline{ALICE RUNS TOWARDS BOB.}}). $\texttt{IVP} \mapsto \texttt{IVP} \cdot \texttt{ADP} \cdot \texttt{NP}$.}
\end{marginfigure}

\begin{marginfigure}
\centering
\scalebox{0.8}{\tikzfig{textcirc/ADPTVcsgcopy}}
\caption{In the presence of a transitive-verb-phrase flanked by a noun-phrase to the right, we may add to the right an adposition followed by a noun-phrase (e.g.~from \texttt{\underline{ALICE THROWS BEER.}} to \texttt{\underline{ALICE THROWS BEER TOWARDS BOB.}}) $\texttt{TVP} \cdot \texttt{NP}_1 \mapsto \texttt{TVP} \cdot \texttt{NP}_1 \cdot \texttt{ADP} \cdot \texttt{NP}_2$}
\end{marginfigure}

\begin{example} First applying the \texttt{TVP}-production rule we obtain $\texttt{NP} \cdot \texttt{TVP} \cdot \texttt{NP}$, which corresponds to the grammatical structure of \texttt{\underline{ALICE GIVES BEER}}. Then applying the $\texttt{TVP} \cdot \texttt{NP}$-production rule, we transform \texttt{\underline{GIVES BEER}} (i.e.~both a green and a black wire) into \texttt{\underline{GIVES BEER TO BOB}}. Altogether, we obtain the following "tree", where the dashed line indicates the two "subtrees": 
\[
\tikzfig{textcirc/AgBtB} 
\]
\end{example}

%\subsection{Compound sentence structure I: pronominal links}\label{sec:pronom-links}

\newthought{We call a sentence with more than one verb \emph{compound}.} We consider two ways in which compound sentences arise: relative pronouns, and phrase scope. We focus now on the first case. Relative pronouns fuse simple sentences together. We model the data of a pronoun using a \textbf{pronominal link}, which identifies nouns from possibly different sentences. We depict these as arrows under the trees, pointing at identified nouns.

\begin{marginfigure}
\centering
\scalebox{0.8}{\tikzfig{textcirc/srprule}}
\caption{\textbf{Subject relative pronouns} replace the subject noun of a parse tree $\texttt{S}_2$, and points to a noun in another, previous parse tree $\texttt{S}_1$, usually the object noun.}
\end{marginfigure}

\begin{example}
For example, in the text: \texttt{\underline{ALICE IS SOBER.} \underline{ALICE GIVES BEER TO BOB.}}, we can identify the two pronominally-linked occurrences of \texttt{\underline{ALICE}}.
\[\tikzfig{textcirc/AiSAgBtB}\]
In the presence of a pronominal link, later occurrences of a noun are interchangeable with a pronoun that refers to an earlier occurrence. We informally outline a convention here, stated in more detail later as Convention \ref{conv:pronpoint}. We assume that pronominal pointers in text always point from nouns in later text to nouns in earlier text. i.e.~from right to left. We also assume that every noun has at most one outgoing pronominal pointer, and at most one incoming pronominal pointer.
\[\tikzfig{textcirc/AiSAgBtB2}\]
\end{example}

\newthought{Relative pronouns may be deconstructed in terms of pronominal links between simple sentences.}
One way compound sentences emerge is by the use of relative pronouns, which allow a noun to be modified by more than one verb. 

where the big triangle on the right hand side is a visual convention to indicate that the two sentences are `fused' as a single sentence. A rewrite rule like (\ref{transgram}) is called a \emph{transformational grammar rule}. Transformations modify phrase structure trees. Below we encounter more examples of transformational rules for relative pronouns. In each example we will start with pronominally linked simple sentences to obtain a meaning-equivalent compound sentence with a relative pronoun. The important takeaway is that we can without loss of generality forget about relative pronouns by always considering their sense-equivalent in terms of pronominal links between simple sentences.

\newthought{Subject Relative Pronouns.}

\begin{example}
For the text \texttt{\underline{ALICE LIKES BOB.}} \texttt{\underline{BOB HATES CLAIRE.}}
we start with two trees. We identify the occurrences of \texttt{\underline{BOB}} with a pronominal link. We can now fuse the trees by replacing the second occurrence of \texttt{\underline{BOB}} with the relative pronoun \texttt{\underline{WHO}}, yielding:
\[
\tikzfig{textcirc/srpex2}
\]
\end{example}

\begin{marginfigure}
\centering
\scalebox{0.8}{\tikzfig{textcirc/ssrprule}}
\caption{In English there is another special use of the subject relative pronoun to coordinate a single noun across two subsequent phrases. Given parse trees corresponding to sentences $\texttt{S}_1$ and $\texttt{S}_2$ in that order, this special case arises when the subject noun of $\texttt{S}_2$ points towards a subject noun in $\texttt{S}_1$. The result of the tree-transformation is that we have, in order: the noun-phrase (along with any adjectives) of $\texttt{S}_1$, followed by $\texttt{S}_1$ with the later noun replaced by the relative pronoun, followed by $\texttt{S}_2$ with its pointing noun removed. We use a multiarrow to point out more than two pronominally identified nouns.}
\end{marginfigure}

\begin{example}
We revisit a previous example:
\[
\tikzfig{textcirc/AiSAgBtB}
\]
By applying the special rule for subject relative pronouns, we obtain the following sentence:
%\TODOb{Why not use the same circle as for trees without the \texttt{!}, like a unit.}
\[
\tikzfig{textcirc/ssrpex}
\]
We have not encountered an isolated noun -- here indicated by \texttt{!} -- or blank labels \textvisiblespace \ before. Observe that these artefacts disappear when we treat relative pronouns as pronominally linked simple sentences.
\end{example}

\begin{marginfigure}
\centering
\scalebox{0.8}{\tikzfig{textcirc/orprule}}
\caption{For \textbf{object relative pronouns}, we take consecutive sentences $\texttt{S}_1$ and $\texttt{S}_2$. If the object noun of $\texttt{S}_2$ points to the object noun of $\texttt{S}_1$, the object relative pronoun comes after the first occurrence, and the second occurrence of the noun is replaced by a blank.}
\end{marginfigure}

\begin{example} We start with the text \texttt{\underline{ALICE LIKES BOB.}} \texttt{\underline{CLAIRE GIVES BEER TO BOB.}}
\[\tikzfig{textcirc/orpex1}\]
Applying the rule for object relative pronouns, we obtain: 
\[\tikzfig{textcirc/orpex2}\]
\end{example}

This concludes our tour of the first case of compound sentences. Now to the second case.

\newthought{Phrase scope occurs when a subphrase of a sentence is itself a full sentence.}\label{sec:phrscope} 
We introduce two such cases. We first introduce these cases intuitively, then we introduce the refinement of \emph{phrase scope} to eliminate a source of ambiguity.

\begin{marginfigure}
\centering
\scalebox{0.8}{\tikzfig{textcirc/SCVsimple}}
\caption{\textbf{Verbs with Sentential Complement} require phrase scope. We treat verbs with a sentential complement -- such as to \texttt{\underline{SEE}} or \texttt{\underline{THINK}} -- as their own grammatical class of verb.$\texttt{S} \mapsto \texttt{NP} \cdot \texttt{SCV} \cdot \texttt{S}$}
\end{marginfigure}

\begin{example} A sentence may consist of a noun-phrase, followed by a verb with sentential complement, followed by a sentence (e.g.~from \texttt{\underline{BOB DANCES.}} to \texttt{\underline{ALICE SEES BOB DANCE.}}).
\[
\tikzfig{textcirc/AsBdcsg}
\]
\end{example}

\begin{marginfigure}
\centering
\scalebox{0.8}{\tikzfig{textcirc/CNJsimple}}
\caption{\textbf{Conjunctions} we treat similarly to verbs with a sentential complement} except with two phrase-bounded regions rather than one, on each side of a conjunction. $\texttt{S} \mapsto \texttt{S} \cdot \texttt{CNJ} \cdot \texttt{S}$}
\end{marginfigure}

\begin{example} From \texttt{\underline{BOB DANCES.}} and \texttt{\underline{ALICE LAUGHS.}} to \texttt{\underline{BOB DANCES SO ALICE LAUGHS.}}:
\[
\tikzfig{textcirc/AduBgHcsg}
\]
\end{example}

There is still a source of ambiguity to be addressed in the above formulation, which is where phrase scope comes into play. For the sentence \texttt{\underline{ALICE SEES DRUNK BOB DANCE.}} there is only one phrase structure:
\[
\tikzfig{textcirc/AsdBdcsg}
\]
which places all of \texttt{DRUNK BOB DANCE(S)} under the scope of what \texttt{ALICE SEES}. However, we wish to further distinguish between the following cases, where:
\begin{itemize}
    \item Alice sees \textbf{both} that Bob is drunk and that Bob dances.
    \item Alice sees \textbf{only} that Bob dances, but not that Bob is drunk.
\end{itemize}
To make the distinction between these two cases we introduce two  `formal' types \bB(\e and \bB)\e, which represent the left and right boundaries of phrase scope respectively.

\begin{marginfigure}
\centering
\scalebox{0.8}{\tikzfig{textcirc/SCVscopecsg}}
\caption{We re-express the rules for verbs with sentential complement and conjunctions to additionally express phrase boundaries. $\texttt{S} \mapsto \texttt{NP} \cdot \texttt{SCV} \cdot \bB ( \e \ \cdot \ \texttt{S} \ \cdot \ \bB ) \e$}
\end{marginfigure}

\begin{marginfigure}
\centering
\scalebox{0.8}{\tikzfig{textcirc/CNJscopecsg}}
\caption{$\texttt{S} \mapsto \bB ( \e \ \cdot \ \texttt{S} \ \cdot \ \bB ) \e \cdot \texttt{CNJ} \cdot \bB ( \e \ \cdot \ \texttt{S} \ \cdot \ \bB ) \e$}
\end{marginfigure}

\begin{marginfigure}
\centering
\scalebox{0.8}{\tikzfig{textcirc/scopeexitcsg}}
\caption{To model the permission of nouns to partially live outside the scope of a phrase as in the example above, we include the following rules that allow a \texttt{NP} to `cross the border' of a phrase boundary. $\bB ( \e \ \cdot \ \texttt{NP} \mapsto \texttt{NP} \cdot \bB ( \e$}
\end{marginfigure}

\begin{marginfigure}
\centering
\scalebox{0.8}{\tikzfig{textcirc/scopeexitRcsg}}
\caption{$\texttt{NP} \ \cdot \ \bB ) \e \mapsto \ \bB ) \e \ \cdot \texttt{NP}$}
\end{marginfigure}

\begin{example}
To disambiguate that in the sentence
\[\texttt{\underline{ALICE SEES DRUNK BOB DANCE.}}\]
Alice sees \textbf{both} that Bob is drunk and that Bob dances, we have the following phrase structure:
\[
\tikzfig{textcirc/AsdBdcsg1}
\]
To disambiguate that Alice \textbf{only} sees that Bob dances, and not that he is drunk, we have:
\[
\tikzfig{textcirc/AsdBdcsg2}
\]
\end{example}
  
\begin{figure}[h!]
    \centering
    \scalebox{0.75}{\tikzfig{textcirc/comic_hybridgrammar}}
    \caption{Generating text with hybrid grammar, illustrated}
    \label{fig:comic1}
\end{figure} 