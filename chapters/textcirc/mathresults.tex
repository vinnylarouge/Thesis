\section{Mathematical results}\label{sec:thm}

\subsection{Main Text Circuit Theorem}\label{thm:main}
We assume some finite vocabulary of words with grammatical types we have considered so far.
Let $\mathfrak{T}$ denote the set of all raw text generated by our grammar.
Let $\mathfrak{C}$ denote the set of all text circuits. We prove the following results:
\begin{enumerate}
    \item We show that the translation rules of Section \ref{graph2gateredux} algorithmically associate any  text with  grammar to a text circuit.
    \item All text circuits are obtainable in this way.
\end{enumerate}

\begin{theorem}
The translation rules of Section \ref{graph2gateredux} define a surjection $\mathfrak{T} \twoheadrightarrow \mathfrak{C}$. 
\end{theorem}

\subsection{Refinements, extensions, conventions}\label{sec:asscon}

There are clarifications to be made before proving the Theorem, which is a claim about \emph{all} text circuits, and \emph{all} raw texts generated by hybrid grammar. Convention \ref{conv:wireorder} clarifies what exactly counts as a text circuit for our purposes. We address an edge-case circuit -- where a wire appears in a box unconnected to any gates -- in Refinement \ref{asm:exists}. For the proof-strategy of Lemma \ref{lem:surj}, which relies on a step that treats multiple gates in parallel as single sentences, we accommodate textual elements such as commas and `contentless' conjunctions such as \texttt{AND ALSO} by means of a generic conjunction \texttt{[\&]}. We note in Refinement \ref{conv:complement} that the phrase-scope distinctions of hybrid grammar are properly expressible in raw text by means of a generic contentless complementiser \texttt{[THAT]}. Finally, we extend the definition of text circuits to include `reflexive pronoun boxes' as structural components.

\begin{convention}[Wire ordering]\label{conv:wireorder}
We elaborate Convention \ref{conv:sliding} for depicting circuits. Every wire of a circuit is labelled with a noun, and when depicted these noun-labels are forced to take a left-to-right order. By convention, this order is kept the same for inputs and outputs of a circuit, by introducing wire twists where necessary. Note that two circuits with the same connectivity but different orders of noun-labels are still considered equal:
\[
\tikzfig{textcirc/wiretwisteq}
\]
To give a negative example, the following diagram \textbf{\bR violates\e} the visual convention, because the input and output wires have a different noun order:
\[
\bR\tikzfig{textcirc/twistviolation}\e
\]
The contents of a box are  circuits as well, which must obey the same input-output agreement rules:
\[
\tikzfig{textcirc/validboxex} \quad\quad\quad\quad\quad\quad \bR \tikzfig{textcirc/invalidboxex} \e
\]
\end{convention} 

\begin{refinement}[The verb \texttt{EXISTS}]\label{asm:exists}
We introduce a special structural rule for an identity wire that does not participate in any gates. In these cases, we interpret the identity wire as the special transitive verb \texttt{EXISTS} at the level of raw text. The rewrite rule for \texttt{EXISTS} from text diagrams to circuits is as follows:
\[
\tikzfig{textcirc/existsfig}
\]
\end{refinement}  

\begin{refinement}[Sentence composition using \texttt{[\&]}]\label{asm:conj}
So far, we have mostly considered listing sentences as text to compose them, e.g.:
\[
\texttt{\underline{ALICE RUNS. BOB DRINKS.}}
\]
For the purposes of Lemma \ref{lem:surj}, we define a special conjunction \texttt{[\&]} at the level of text, which allows us to consider multiple gates in parallel as arising from a single sentence in an unambiguous way. In English, there are multiple `contentless' ways that two sentences may be composed as a single sentence. For example, by an ampersand:
\[
\texttt{\underline{ALICE RUNS [\&] BOB DRINKS.}}
\]
or by `contentless' conjunctive phrases such as \texttt{AND ALSO}~e.g.: 
\[
\texttt{\underline{ALICE RUNS AND ALSO BOB DRINKS.}}
\]
The special conjunction \texttt{[\&]} is intended to be a catchall for these kinds of `contentless' conjunctions. At the level of circuits \texttt{[\&]} is interpreted as gate composition, and its rewrite rule from text diagrams to circuits is as follows:
\[
\tikzfig{textcirc/sentconj}
\]
Note that in the case where the noun wires are disjoint, the rewrite rule for \texttt{[\&]} just yields parallel composition of circuits:
\[
\tikzfig{textcirc/sentconjdisj}
\]
\end{refinement}

\begin{refinement}[Sentence composition within phrase scope, and the complementizer \texttt{[THAT]}]\label{conv:complement}
Consider the sentence:
\[
\texttt{\underline{CLAIRE SEES ALICE RUNS \bR[\&] BOB DRINKS\e.}}
\]
For us, when the raw text is equipped with hybrid grammar structure, there is no ambiguity as to whether Claire sees both that Alice runs and Bob drinks, or just the former. Recalling Refinement \ref{asm:conj}, neither reading of \texttt{[\&]} as a comma or \texttt{AND ALSO} resolves this ambiguity. Since our broad claim is that we address a restriction of natural language, we have to justify that this distinction made by hybrid grammar structure is one that actually has a counterpart at the level of raw text. To disambiguate in a similar manner as our previous refinements, we introduce a complementiser \texttt{[THAT]}, which behaves much like the phrase scope formal types \texttt{(} and \texttt{)} introduced in Section \ref{sec:phrscope}. Complementisers are words such as \texttt{HOW} or \texttt{WHAT}, which prefix sentential complements~e.g.:
\[
\texttt{\underline{CLAIRE SEES HOW ALICE RUNS.}}
\]
\[
\texttt{\underline{CLAIRE SEES WHAT BOB DRINKS.}}
\]
We do not consider complementisers in general here. Instead we use just one `contentless' complementiser \texttt{[THAT]}, which we freely omit when it is implicit. Returning to the initial example, the presence of a complementiser allows us to distinguish, at the level of raw text, the case where Claire \textbf{only} sees Alice running:
\[
\texttt{\underline{CLAIRE SEES [THAT] ALICE RUNS [\&] BOB DRINKS.}}
\]
from the case where Claire sees \textbf{both} that Alice runs and that Bob drinks:
\[
\texttt{\underline{CLAIRE SEES [THAT] ALICE RUNS [\&] [THAT] BOB DRINKS.}}
\]
So, when we encounter the conjunction \texttt{[\&]} within a phrase scope, we interpret it at the level of text as \texttt{AND ALSO THAT}. More generally, we use \texttt{[THAT]} to signify the open-bracket that begins a scoped phrase in raw text.
%\footnote{This convention leads to grammatical inaccuracies in dealing with what Lambek calls `verbs of causation' \cite{LambekBook}, e.g.~\bR\texttt{\underline{CLAIRE HELPS (THAT) BOB DRINKS.}}\e as opposed to \texttt{\underline{CLAIRE HELPS BOB (TO) DRINK.}} We leave further consideration for future work.}
\end{refinement}

%\bR
%\subsubsection{Reflexive Pronouns}\label{sec:reflpron}
%
%\underline{Warning: technical content, necessary for proof.} In this section we deal exclusively with the pronominal links that arise from reflexive pronouns, which cannot na\"{i}vely be eliminated as wire composition.
%\e

\begin{convention}[Reflexive pronoun boxes]\label{asm:reflpron}
We extend the definition of text circuits to accommodate reflexive pronouns, such as \texttt{HERSELF}, \texttt{ITSELF}. We treat pronominal links as gate composition in all other cases, but by definition, a reflexive pronoun in a circuit occurs when an output noun wire must compose with an input noun wire in its relative past, which cannot be drawn as a circuit. So, at the level of circuits, we assert a family of `reflexive pronoun boxes'. For all $k > 1 \in \mathbb{N}$, and for all pairs of indices $1 \leq i < j \leq k$, we have a box:
\[
\tikzfig{textcirc/reflpronbox}
\]
These boxes are the targets of rewrites from text diagrams when a pronominal link elimination would connect an output of a circuit to an input, `doubling back' a noun wire:
%\footnote{\bR
%As a technical aside, we note that there are natural interpretations/implementations of such boxes when the circuits are interpreted in traced monoidal or compact closed categories, or whenever noun wires possess an internal monoid and comonoid:  ONLY RIGHT ONE IS COPY-MERGE; REFER TO INTERNAL WIRE PAPER. 
%\[\tikzfig{textcirc/refloptions}\]\e}
\[
\tikzfig{textcirc/reflboxredux} 
\]
\end{convention}

\begin{refinement}[Coherence of reflexive pronoun boxes]\label{ref:proncoherence} 
A \emph{critical pair} in a rewrite system is a situation in which two rules are applicable to a term, and result in different final outcomes. Critical pairs are hence the only obstacle to the claim of Proposition \ref{lem:function}. Reflexive pronouns are involved in all critical pairs that arise in rewriting text diagrams into circuits. Consider the following example, for the text diagram corresponding to the text:
\[\texttt{\underline{BOB DRINKS. BOB LIKES BOB.}}\]
\[\tikzfig{textcirc/criticalpair}\]
So the order of pronominal link eliminations and (their special case) reflexive pronoun box introductions matters. As we show in Corollary \ref{cor:reflnorm}, we may generally deal with these critical pairs by a convention that all reflexive pronoun box introductions come before the elimination of other pronominal links. Recalling the footnote in Section \ref{sec:pronom-links}, a pronominal link in a text diagram can take two directions. Neither is strictly appropriate in the case of reflexive pronouns, where a pronominal link should \emph{identify} two wires. So here we introduce syntactic sugar for the reflexive pronoun box, and introduce rewrite rules such that identification of wires is respected. We introduce the following shorthand for reflexive pronominal links:
\[\tikzfig{textcirc/syntacticsugar1}\]
Identifying wires is \emph{associative}, which we enforce by the following bidirectional rewrites:
\[\tikzfig{textcirc/reflfuse}\]
A reflexive pronoun box with only identity wires inside can be eliminated by an \emph{identity} rule:
\[\tikzfig{textcirc/reflid}\]
Conversely, we can introduce new reflexive pronoun boxes by \emph{splitting} existing ones along identified wires:
\[\tikzfig{textcirc/reflid2}\]
And finally, a special aspect of reflexive pronoun boxes is that gates on one wire may \emph{slide} out of them. This reflects the fact that sentences that only modify a single noun never require a reflexive pronoun, which would refer to another noun argument:
\[\tikzfig{textcirc/reflsingleslide}\]
\end{refinement}


\subsection{Text diagrams to text circuits}\label{graph2gateredux}

This section presents lemmas that together constitute the hard work in the proof of Proposition \ref{lem:function}, which claims the existence of a function $\mathfrak{T} \rightarrow \mathfrak{C}$.

First, Lemma \ref{lem:shrinking} and Corollary \ref{cor:reflnorm} constructively organise the rewrites of Refinement \ref{ref:proncoherence} into a function from text diagrams to text diagrams with no pronominal links.
Second, Lemma \ref{lem:gatenorm} introduces new rewrite rules and ancilla types which constructively constitute a function that takes text diagrams with no pronominal links nor phrase scoping into a normal form that corresponds to a single circuit gate.
Third, Lemma \ref{lem:gatenorm} is extended by Lemma \ref{lem:diagnorm} to account for text diagrams with phrase scope as well.
When taken together with the content of Section \ref{sec:congraphas} -- which provides a correspondence that sends every hybrid grammar text to a text diagram -- the three stages above complete the description of a function from hybrid grammar text to text circuits.

\begin{lemma}[Shrinking reflexive pronouns]
\label{lem:shrinking}
The associativity, identity, splitting, and sliding rewrite rules for reflexive pronouns suffice to recast text circuits in a form where every reflexive pronoun box contains exactly one gate.
\begin{proof}
Nested reflexive pronoun boxes can be fused and rearranged according to the associativity rule. Any reflexive pronoun box containing two or more gates composed sequentially along an identified wire can be split by the splitting rule. The splitting and identity rules may be applied to eliminate twists in reflexive pronoun boxes containing a single gate, such that the inputs and outputs align.
\end{proof}
\end{lemma}

\begin{corollary}\label{cor:reflnorm}
Text circuits obtained by the shrinking lemma coincide with text circuits obtained from text diagrams where reflexive pronoun box reductions are applied before other pronominal link eliminations.
\begin{proof}
By construction, reflexive pronouns only occur within a sentence, and sentences correspond to individual gates. The rewrite rules of Lemma \ref{lem:shrinking} preserve gate connectivity, which is determined by non-reflexive pronominal links.
\end{proof}
\end{corollary}

\begin{example}[An application of the shrinking lemma]
Consider a text circuit that glosses as the text:
\[\underline{\texttt{BOB TELLS BOB ABOUT BOB. BOB DRINKS. BOB LIKES BOB.}}\]
\[\tikzfig{textcirc/reflreductionexample}\]
\end{example}

\begin{figure}[h!]
\centering
\begin{tabular}{|c|c|}
\hline
Rule name & Text diagram rewrite \\ \hline
\texttt{IS}-elimination & \tikzfig{textcirc/ADJistransform} \\ \hline
\texttt{ADV}-gather & \tikzfig{textcirc/advgather} \\ \hline
\texttt{ADV}-assoc. & \tikzfig{textcirc/advassoc} \\ \hline
\texttt{ADP(IV)}-ancilla & \tikzfig{textcirc/adpIVancilla} \\ \hline
\texttt{ADP(TV)}-ancilla & \tikzfig{textcirc/adpTVancilla} \\ \hline
\texttt{adp}-gather & \tikzfig{textcirc/adpgather}  \\ \hline
\texttt{adp}-assoc. & \tikzfig{textcirc/adpassoc} \\ \hline
\texttt{adp-ADV}-order & \tikzfig{textcirc/adorder} \\ \hline
\end{tabular}
\caption{Gate normalisation rewrites}
\label{gatenormalisationrules}
\end{figure}

\begin{lemma}\label{lem:gatenorm}
We present ancillas and rewrites in Table \ref{gatenormalisationrules} such that every text diagram without pronominal links and without phrase scope can be viewed as a unique text circuit constructed out of gates of the following forms:
\[\tikzfig{textcirc/gatenormals}\]
\begin{proof}

We introduce an ancillary wire type \texttt{adp}, along with the following ancillary text diagram components.

\[\tikzfig{textcirc/ancillas}\]

We present normalisation rewrite rules in the table of Figure \ref{gatenormalisationrules}.

The \texttt{NP} wires of a text diagram correspond to the noun wires in a text circuit. For a text diagram from a hybrid grammar structure without pronominal links and phrase scope, there are six non-label text diagram nodes that involve \texttt{NP} wires; each such node corresponds to either a gate or a box.
\[(1) \tikzfig{textcirc/ADJcsg} (2) \tikzfig{textcirc/ADJisgraph} (3) \tikzfig{textcirc/IVgraph} (4) \tikzfig{textcirc/TVgraph} (5) \tikzfig{textcirc/ADPIVgraph} (6) \tikzfig{textcirc/ADPTVgraph}\]
The \texttt{IS}-elimination rule recasts all instances of case (2) nodes as case (1) nodes. Recalling by Convention \ref{conv:onlyconnect} that in text diagrams only connectivity matters, we may contract the \texttt{\underline{ADJ}} label associated with each instance of case (1), interpreting as a gate in a text circuit as follows.
\[\tikzfig{textcirc/adjnormalex}\]
Cases (5) and (6) are adpositions, which can only appear in the presence of an \texttt{IV} or \texttt{TV} wire respectively, which we refer to collectively as a \texttt{V} wire for simplicity. The only way such wires appear is by cases (3) and (4) respectively, and the only way they end is by \texttt{\underline{V}} labels.

We cannot repeat the same contraction trick as in \texttt{ADJ} labels for \texttt{V} labels directly, because in between a node of case (3) or (4) and its associated \texttt{\underline{V}} label, there may be \texttt{ADV} nodes, or \texttt{ADP} nodes as in (5) and (6). However, all such nodes have one input and one output \texttt{V} wire, so all \texttt{ADV} and \texttt{ADP} nodes, and their labels, can be unambiguously associated with a single \texttt{\underline{V}} label.

The ancillas and rewrites are in service of obtaining a normal form for the \texttt{ADV} and \texttt{ADP} nodes and labels contracted to their parent \texttt{V} node of case (3) or (4). We analyse the three possible cases for a given \texttt{V} wire in the following order to illustrate the purpose of the rules; only \texttt{ADV} nodes appear; only \texttt{ADP} nodes appear; both kinds of nodes appear.

In the first case, when two \texttt{ADV} nodes appear adjacently on the same \texttt{V} wire, we apply the \texttt{ADV}-gather rule. The \texttt{ADV}-assoc bidirectional rewrite rule makes the order of \texttt{ADV}-gather applications irrelevant in the case of multiple adverbs for the same verb. Contracting labels, these rules allow re-expression of multiple \texttt{ADV} nodes on the same \texttt{V} wire as follows:
\[\tikzfig{textcirc/advgatherex}\]
In the second case -- which also addresses nodes of type (5) and (6) -- when there are multiple \texttt{ADP} nodes, we first apply the \texttt{ADP}-ancilla rules to introduce the \texttt{adp} ancilla type. We may then apply \texttt{adp}-gather, mirroring the treatment for \texttt{ADV} wires: similarly, the \texttt{adp}-assoc rule makes the order of \texttt{adp}-gather rewrites irrelevant in the case of multiple adpositions on the same verb. These rules allow re-expression of multiple \texttt{ADP} nodes on the same wire as follows, which we illustrate for two \texttt{ADP(TV)}.
\[\tikzfig{textcirc/adpgatherex}\]
In the third and final case where there are \texttt{ADV} and \texttt{ADP} nodes on a \texttt{V} wire, we first eliminate all \texttt{ADP} nodes of type (5) or (6) by \texttt{ADP}-ancilla rewrites. Then by applying the \texttt{adp}-\texttt{ADV}-order rule, we may arrange all \texttt{adp} nodes above all \texttt{ADV} nodes on the \texttt{V} wire, then we apply the treatment of the previous two cases of only \texttt{ADV} or only \texttt{adp} nodes.
Contracting \texttt{\underline{ADV}} and \texttt{\underline{ADP}} labels, this procedure obtains the following normal forms for every node of type (3) and (4) respectively.
\[\tikzfig{textcirc/gatenormalformIV} \qquad\qquad\qquad \tikzfig{textcirc/gatenormalformTV}\]
Which we may read as gates of the following form respectively, colour-coded for legibility.
\[\tikzfig{textcirc/gatenormalIV} \quad\quad\quad\quad \tikzfig{textcirc/gatenormalTV}\]
These are equivalent up to syntactic sugar to the last two gates in the claim. Uniqueness of the resulting circuit follows because none of the rules change the relative connectivity of nodes of type (1) through (4).
\end{proof}
\end{lemma}

\begin{lemma}\label{lem:diagnorm}
Every text diagram without pronominal links can be viewed as a unique text circuit.
\begin{proof}
We only need extend the claim of Lemma \ref{lem:gatenorm} to accommodate phrase scope. Recall by Convention \ref{conv:phrscope} that phrase scope only allows \texttt{NP} wires and pronominal links to pass through, that \texttt{\underline{NP}} labels may only occur outside of phrase scope, and that phrase scope nesting anywhere in the diagram is unambiguous. We prove the claim by (strong) induction, where the inductive hypothesis is that all text diagrams with phrase scope that nest at most $k$-levels deep can be viewed as a unique text circuit.
For the base case, where there is no phrase scope, we are done by Lemma \ref{lem:gatenorm}.
For induction, assume we have a text diagram with phrase scope nesting of depth $k+1$. The topmost phrase scope node is either an \texttt{SCV} node or a \texttt{CNJ} node.
\paragraph{Verbs with sentential complement} Since the same number of \texttt{NP} wires must enter as leave the phrase scope, by the inductive hypothesis we may express the content of phrase the phrase scope as a text circuit $\mathcal{C}$, which we may unambiguously read off as follows (modulo adverbs and adpositions for the \texttt{\underline{SCV}} label as considered previously.)
\[\tikzfig{textcirc/SCVredux}\]
The remainder of the text diagram falls to the inductive hypothesis.
\paragraph{Conjunctions}
Similarly as before, we may express the contents of the left and right conjuncts as text circuits $\mathcal{C}_1$ and $\mathcal{C}_2$. There are two subcases we consider in order: where the conjuncts do not share \texttt{NP} wires, and when they do. In the former subcase, we obtain the following gate from a \texttt{CNJ} node.
\[\tikzfig{textcirc/CNJredux}\]
In the latter subcase where the two conjuncts share noun wires -- or, anticipating later work, when there are pronominal links between the conjuncts -- we can introduce a conjugate box with overlapping arguments as syntactic sugar standing in for a `normal' \texttt{CNJ} box inside reflexive pronoun boxes:
\[
\tikzfig{textcirc/CNJredux2}
\]
\end{proof}
\end{lemma}

\begin{example}
Figure \ref{fig:comic2} illustrates how to obtain text circuits from text diagrams.

\begin{figure}[!]
    \centering
    \scalebox{0.75}{\tikzfig{textcirc/comic_text2circ}}
    \caption{Obtaining text circuits from text diagrams.} 
    \label{fig:comic2}
\end{figure}
\end{example}

\subsection{Proof of Theorem}

It suffices to demonstrate the following. First, that the translation procedure from text to circuit given in Section \ref{graph2gateredux} yields a function $\mathfrak{T} \rightarrow \mathfrak{C}$. Second, that this function is surjective.

%\begin{figure}
%\centering
%\scalebox{0.75}{
%\begin{tabular}{|c|c|c|c|}
%\hline
%\text{Rule} & \text{Text grammar} & \text{Text diagram} & \text{Text circuit rewrite} \\ \hline
%Intrans.Verb & \tikzfig{textcirc/IVcsg} & \tikzfig{textcirc/IVgraph} & \tikzfig{textcirc/IVreduxcopy} \\ \hline
%Trans.Verb & \tikzfig{textcirc/TVcsg} & \tikzfig{textcirc/TVgraph} & \tikzfig{textcirc/TVredux} \\ \hline
%Adjective(Pre.) & \tikzfig{textcirc/ADJcsg} & \tikzfig{textcirc/ADJcsg} & \tikzfig{textcirc/ADJredux} \\ \hline
%Adjective(Post.) &  \tikzfig{textcirc/ADJiscsg} &  \tikzfig{textcirc/ADJisgraph} & \tikzfig{textcirc/ADJistransform} \\ \hline
%Adverb(IV) & \tikzfig{textcirc/ADVIVcsg} & \tikzfig{textcirc/ADVIVcsg} & \tikzfig{textcirc/ADVredux}  \\ \hline
%Adverb(TV) & \tikzfig{textcirc/ADVTVcsg} & \tikzfig{textcirc/ADVTVcsg} & \tikzfig{textcirc/ADVTVredux} \\ \hline
%Adposition(IV) & \tikzfig{textcirc/ADPIVcsg} & \tikzfig{textcirc/ADPIVgraph} & \tikzfig{textcirc/ADPIVredux} \\ \hline
%Adposition(TV) & \tikzfig{textcirc/ADPTVcsg} & \tikzfig{textcirc/ADPTVgraph} & \tikzfig{textcirc/ADPTVredux} \\ \hline
%Sent.Comp.Verb & \tikzfig{textcirc/SCVscopecsg} & \tikzfig{textcirc/SCVgraph} & \tikzfig{textcirc/SCVredux} \\ \hline
%Conjunction & \tikzfig{textcirc/CNJscopecsg} & \tikzfig{textcirc/CNJcsg} & \tikzfig{textcirc/CNJredux} \\ \hline
%Phrase Scope (L) & \tikzfig{textcirc/scopeexitcsg} & \tikzfig{textcirc/phrenter} \quad \vline \quad\quad \tikzfig{textcirc/phrexit} & \text{N/A} \\ \hline
%Phrase Scope (R) & \tikzfig{textcirc/scopeexitRcsg} & \tikzfig{textcirc/phrenterR} \quad \vline \quad\quad \tikzfig{textcirc/phrexitR} & \text{N/A} \\ \hline
%\end{tabular}
%}
%\caption{Table of generators and correspondences}
%\label{tab:rules}
%\end{figure}

\begin{proposition}\label{lem:function}  
 We have a function $\mathfrak{T} \rightarrow \mathfrak{C}$.
\begin{proof}
 Lemma \ref{lem:diagnorm} handles all text diagrams obtained from text without pronominal links. It remains to be shown that pronominal links can be resolved uniquely in the setting of text diagrams. The following transformations eliminate pronominal links.
\[
\tikzfig{textcirc/prontransforms}  
\]
\[
\tikzfig{textcirc/prontransforms2}  
\]
The only obstacle to interpreting these transformations directly as composition of text circuits is addressed by reflexive pronoun boxes, in Convention \ref{asm:reflpron}. The critical pairs that arise as a result of reflexive pronoun boxes are addressed by Refinement \ref{ref:proncoherence}, and a normal form for reflexive pronoun boxes is constructively provided by Lemma \ref{lem:shrinking}. So the initial hybrid text determines the corresponding circuit.
\end{proof}
\end{proposition}

\begin{proposition}\label{lem:surj}
$\mathfrak{T} \rightarrow \mathfrak{C}$ is surjective.
\begin{proof}
Surjectivity amounts to showing that an arbitrary circuit corresponds to a text that translates into that circuit; in other words, every text circuit can be textualised. First, we divide the circuit into horizontal slices, such that every slice either contains at least one gate, or contains only twisting wires. Here we exploit Refinement \ref{asm:exists}; when a wire inside the context of a box does not connect to any gates, we assume it connects to an  \texttt{EXISTS} gate. We iterate this slicing within the holes of boxes:  
%\newpage %%%%%%%%%%%%%%%%%%%%%%%%%%%%%%%%%%%%
\[
\tikzfig{textcirc/circ2textex1} 
\]
Now we exploit Refinement \ref{asm:conj}; every horizontal slice with gates obtained this way is a collection of gates composed in parallel, possibly with identity wires. Identity wires inside the holes of boxes that do not connect to gates we treat as connected to an \texttt{EXISTS} gate, as in the hole of \texttt{SEES} in the example. Other identity wires we do not need to mention in text.  We deal with adjective gates by using the copular \texttt{IS} construction, for instance \texttt{DRUNK} in the example. We deal with reflexive pronouns by introducing the appropriate pronominal link, and just restating the noun, for instance \texttt{HIMSELF} in the example.  We obtain a text diagram for each slice with gates:
\[
\scalebox{0.8}{\tikzfig{textcirc/circ2textex2}}
\]
Then we join the diagrams of each slice with pronominal links that mirror the connectivity of the wire twisting slices:
\[
\scalebox{0.8}{\tikzfig{textcirc/textualisation3}}
\]
The structure of pronominal links in text coincides with the compositional structure of circuits. By the rules of Section \ref{sec:congraphas} relating text diagrams to hybrid grammar, and by the interpretation rules of Refinement \ref{asm:conj}, we obtain a hybrid grammar text:
\[
\scalebox{0.8}{\tikzfig{textcirc/circ2textexFIN}}
\]
\[
\footnotesize{
\texttt{\underline{ALICE SEES THAT BOB IS DRUNK AND ALSO THAT CLAIRE EXISTS.}}
}\]
\[
\footnotesize{
\texttt{\underline{ALICE TELLS CLAIRE THAT DENNIS HATES DEE AND ALSO THAT DEE LIKES DENNIS, BOB LAUGHS AT BOB.}}
}\]
By construction, our rewrite rules will turn the text back into our starting circuit.
\end{proof}
\end{proposition}

\begin{example}
In Figure \ref{fig:comic3} we give the example of textualisation of a text circuit that appeared in all of our previous such illustrations.

\begin{figure}[h!]
    \centering
    \scalebox{0.75}{\tikzfig{textcirc/comic3}}
    \caption{Textualisation of circuits, illustrated.}
    \label{fig:comic3}
\end{figure}
\end{example}

\subsection{Extending grammar by means of equations}

An immediate consequence of the Text Circuit Theorem is the existence of a nontrivial equivalence relation on parsed text. In prose: we consider two texts equivalent if they result in equal text circuits.  So we may define  equivalence of texts $\mathcal{T}_1 \equiv \mathcal{T}_2$ as follows:
\[
\left.\begin{array}{c}
\mbox{exists rewrite}\ \  \mathcal{T}_1 \Rightarrow \mathcal{C}_1\\ 
\mbox{exists rewrite}\ \  \mathcal{T}_2 \Rightarrow \mathcal{C}_2 
\end{array}\right\} \ \ \mbox{such that}\ \  \mathcal{C}_1 =_{\mathfrak{C}} \mathcal{C}_2  
\] 
where $\mathcal{T} \Rightarrow \mathcal{C}$ denotes a rewrite of text with grammar $\mathcal{T}$ to text circuit $\mathcal{C}$, and $\mathcal{C}_1 =_{\mathfrak{C}} \mathcal{C}_2$ denotes equality of text circuits -- cf.~Convention \ref{conv:sliding}.

We have seen so far how text circuits appear to capture the essential connectivity of meaning underneath the bureaucracy of natural language. By this consideration, the equivalence relation on text we have defined looks a lot like `meaning equivalence'. Now we turn this observation into a guiding assumption.

%\TODOb{IN a way this is an observation from what we did earlier: text that becomes the same circuit seems to always mean the same thing.  Here this observation is used as an assumption.}
\begin{thesis}[Text Circuit Thesis]
Equal circuits stand for equal text meanings. 
\end{thesis}
 
The thesis provides a recipe for engineering extensions of our grammar  
to accommodate grammatical phenomena beyond what we have considered here. We will explicitly demonstrate the inclusion of passive voice, possessive pronouns, and adjectivalisation of verbs using \texttt{[ING]} below. 

When it is broadly agreed that $\mathcal{T}_1$ and $\mathcal{T}_2$ `mean the same thing',  then we will \underline{postulate} that to be the case, and from this reverse engineer what the pieces of the text diagrams and corresponding rewrite rules of the grammar should be.  If $\mathcal{T}_1$ makes use of a grammatical phenomenon $\mathcal{GP}$ beyond what we have considered thus far, while $\mathcal{T}_2$ is within our diagrams/grammar, then we make the circuits of $\mathcal{T}_1$ and $\mathcal{T}_2$ equal by adding  the pieces of the text diagrams and corresponding rewrite rules for $\mathcal{GP}$. This procedure is best illustrated by the following examples, the first three of which are explored in \cite{GramEqs}.\footnote{In \cite{GramEqs} rather than extending diagrams/grammar by postulating equations like we do here, we used grammatical structure and so-called `internal wirings' \cite{FrobMeanI, FrobMeanII, GrefSadr, KartsaklisSadrzadeh2014, CLM, CoeckeText, CoeckeMeich} to derive equations between sentences.  This was in fact the path we took in order to arrive at the results that we present here, and therefore the amount of credit we give here to those internal wirings.  A forthcoming paper will be dedicated to this broad topic of `internal wirings' and their relationship to the results presented here \cite{IntWire}, and we also briefly address them in Section \ref{sec:discussion}.}
 

\begin{example}[Passive voice]
In English, expressing a transitive verb in passive voice reverses the order of subject and object. For example, the following sentence in passive voice:
\[
\texttt{\underline{BOB \bB IS LIKED BY\e ALICE}}
\]
conveys the same factual data as the following sentence in active voice:
\[
\texttt{\underline{ALICE \bB LIKES\e BOB}}
\]
To accommodate the passive voice, we introduce a new wire type $\texttt{TVP}_{psv.}$. In the figure below, on the left we introduce a string diagram that allows a \texttt{TVP} to become a $\texttt{TVP}_{psv.}$ within a phrase scope, and on the right, we introduce a string diagram that closes the passive voice phrase scope by labelling with \texttt{\underline{IS}} and \texttt{\underline{BY}}:
\[
\tikzfig{textcirc/passivevoicegen1}
\]
We also ask that every label: 
\[
\tikzfig{textcirc/passivevoicegen2}
\]
has a passive voice counterpart:    
\[
\tikzfig{textcirc/passivevoicegen3}
\]
e.g.~\texttt{\underline{LIKES}} becomes \texttt{\underline{LIKED}}.
We also stipulate that, apart from the diagram on the left above, $\texttt{TVP}_{psv.}$ mimics \texttt{TVP} for all other diagrams. This is so that we may further generate text grammar/diagrams  
within the passive voice scope to obtain, e.g.:
\[
\texttt{\underline{BOB IS DEEPLY LIKED BY ALICE}}
\]
As a point of justification, the exception for the left diagram and having a separate $\texttt{TVP}_{psv.}$ is necessary to disallow the passive construction to be applied repeatedly, which for example overgenerates the \bR\textbf{ungrammatical}\e:
\[
\bR \texttt{\underline{BOB IS IS LIKED BY BY ALICE}} \e
\]
Returning to the example, we reformulate instances of passive voice into active voice by the following rewrite rule:
\[
\tikzfig{textcirc/passiveredux}
\]
So we obtain the following proof of text equivalence:
\[
\tikzfig{textcirc/passivecircexcopy}
\]
\end{example}  

\begin{example}[Possessive Pronouns]\label{ex:posspron}
Consider the possessive pronoun \texttt{HIS} in the phrase:
\[
\texttt{\underline{BOB DRINKS HIS BEER}}
\]
We can reformulate the noun phrase \texttt{HIS BEER} as the noun phrase \texttt{BEER THAT HE OWNS} without changing the core meaning. Applying this transformation, we obtain:
\[
\texttt{\underline{BOB DRINKS BEER THAT HE OWNS}}
\]
We know that the relative pronoun \texttt{THAT} in this case gives us the same circuit as the text:
\[
\texttt{\underline{BOB DRINKS BEER. BOB OWNS BEER.}}
\]
We can make this text equivalent to the original \texttt{\underline{BOB DRINKS HIS BEER}} by introducing the following generation and rewrite rules for possessive pronoun labels. We introduce a dotted-blue possessive pronominal link type with two generators, so that the possessor is pronominally referred to by the possessive pronoun generator:
\[
\tikzfig{textcirc/possprongen}
\]
The rewrite rule that eliminates the possessive pronoun and the possessive pronominal is as follows:
\[
\tikzfig{textcirc/posspronredux}
\]
So, returning to our sentence, we have a series of rewrites from text diagram to text circuit: 
\[
\tikzfig{textcirc/posspronexample}
\]
Note that while text diagrams from text are planar, in the process of rewrites, we may freely twist wires.\end{example}

\begin{example}[Adjectivalisation of Verbs by gerund \texttt{-ING}]
Appending \texttt{-ING} to a verb allows that verb to be used as if it were an adjective. For example, the noun phrase:
\[
\texttt{\underline{DANC-ING ALICE}}
\]
may be reformulated as the noun phrase: 
\[
\texttt{\underline{ALICE WHO DANCES}}
\]
We model \texttt{[ING]} as a generator that transforms an \texttt{ADJ} into a \texttt{IVP}.
\[
\tikzfig{textcirc/inggen}
\]
The reduction rule is straightforward:
\[
\tikzfig{textcirc/ingredux}
\]
As a result, we obtain equivalences between the sentences:
\begin{itemize}
    \item \texttt{\underline{ALICE IS DANCING.}}
    \item \texttt{\underline{ALICE DANCES.}}
\end{itemize}
and the noun-phrases:
\begin{itemize}
    \item \texttt{\underline{DANCING ALICE}}
    \item \texttt{\underline{ALICE WHO DANCES}}
\end{itemize}
\[
\tikzfig{textcirc/ingexamples}
\]
\end{example}