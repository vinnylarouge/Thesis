\section{A formal linguist's introduction to weak n-categories}

\subsection{String-rewrite systems as 1-object-2-categories}

Say we have an alphabet $\Sigma := \{\textcolor{green}{\alpha}, \textcolor{orange}{\beta}, \textcolor{cyan}{\gamma}\}$. Then the Kleene-star $\Sigma^*$ consists of all strings (including the empty string $\varepsilon$) made up of $\Sigma$, and we consider formal languages on $\Sigma$ to be subsets of $\Sigma^*$. Another way of viewing $\Sigma^*$ is as the free monoid generated by $\Sigma$ under the binary concatenation operation $(\_ \ \cdot \ \_)$ which is associative and unital with unit $\varepsilon$, the empty string. Associativity and unitality are precisely the conditions of composition of morphisms in categories, so we have yet another way to express $\Sigma^*$ as a finitely presented category; we consider a category with a single object $\bullet$, taking $\varepsilon$ to be the identity morphism $\textbf{1}_\bullet$ on the single object, and we ask for the category obtained when we consider the closure under composition of three non-identity morphisms $\textcolor{green}{\alpha}, \textcolor{orange}{\beta}, \textcolor{cyan}{\gamma}: \bullet \rightarrow \bullet$. In this category, every morphism $\bullet \rightarrow \bullet$ corresponds to a string in $\Sigma^*$. We illustrate this example in the margins. A string-rewrite system additionally consists of a finite number of string-transformation rules. Building on our example, we might have a named rule $\textcolor{magenta}{R}: \textcolor{green}{\alpha} \mapsto \textcolor{orange}{\beta} \cdot \textcolor{cyan}{\gamma}$, which we illustrate in Figure \ref{fig:ruleR}.\\

\begin{marginfigure}
\centering
\[\tikzfig{ncat/flower1}\]
\caption{The category in question can be visualised as a commutative diagram.}
\end{marginfigure}

\begin{marginfigure}
\centering
\[\resizebox{\textwidth}{!}{\tikzfig{ncat/table1}}\]
\caption{When there are too many generating morphisms, we can instead present the same data as a table of $n$-cells; there is a single 0-cell $\bullet$, and three non-identity 1-cells corresponding to $\textcolor{green}{\alpha}, \textcolor{orange}{\beta}, \textcolor{cyan}{\gamma}$, each with source and target 0-cells $\bullet$. Typically identity morphisms can be omitted from tables as they come for free. Observe that composition of identities enforces the behaviour of the empty string, so that for any string $x$, we have $\epsilon \cdot x = x = \epsilon \cdot x$.}
\end{marginfigure}

\begin{marginfigure}
\centering
\[\tikzfig{ncat/cstring1}\]
\caption{For a concrete example, we can depict the string $\textcolor{green}{\alpha} \cdot \textcolor{cyan}{\gamma} \cdot \textcolor{cyan}{\gamma} \cdot \textcolor{orange}{\beta}$ as a morphism in a commuting diagram.}
\end{marginfigure}

\begin{marginfigure}
\centering
\[\tikzfig{ncat/string1}\]
\caption{The string-diagrammatic view, where $\bullet$ is treated as a wire and morphisms are treated as boxes or dots is an expression of the same data under the Poincar\'{e} dual.}
\end{marginfigure}

\begin{marginfigure}\label{fig:ruleR}
\centering
\[\tikzfig{ncat/poindual}\]
\caption{We can visualise the rule as a commutative diagram where $\textcolor{magenta}{R}$ is a 2-cell between the source and target 1-cells. Just as 1-cells are arrows between 0-cell points in a commuting diagram, a 2-cell can also be conceptualised as a directed surface from a 1-cell to another. Taking the Poincar\'{e} dual of this view gives us a string diagram for the 2-cell $\textcolor{magenta}{R}$.}
\end{marginfigure}

\newthought{We consider rewrites to be equivalent, but not equal.} In a string-rewrite system, rewrites are applied one at a time. This means that even for our simple example, there are two possible rewrites from $\textcolor{green}{\alpha} \cdot \textcolor{green}{\alpha}$ to obtain $\textcolor{orange}{\beta} \cdot \textcolor{cyan}{\gamma} \cdot \textcolor{orange}{\beta} \cdot \textcolor{cyan}{\gamma}$. Here are the two rewrites viewed in two equivalent ways, first on the left informally where strings are nodes in a graph and rewrites are labelled transitions and secondly on the right as two distinct commuting 2-diagrams.

\[\resizebox{0.8\textwidth}{!}{\tikzfig{ncat/2ways}}\]

What should we say about how these two different rewrites relate to each other? Let's say Alice is a formal linguist who is only interested in what strings are reachable from others by rewrites -- this is \emph{de rigeur} when we consider formal languages to be subsets of $\Sigma^*$. She might be happy to declare that these two rewrites are simply equal; categorically this is tantamount to her declaring that any two 2-cells in the 1-object-2-category that share the same source and target are in fact the same, or equivalently, that any $n$-cells for $n \geq 3$ are identities. In fact, what Alice really cares to have is a category where the objects are strings from $\Sigma^*$, and the morphisms are a reachability relation by rewrites; this category is \emph{thin}, in that there is at most one arrow between each pair of objects, which forgets what rewrites are applied.

\[\tikzfig{ncat/thinalice}\]

Let's say Bob is a different sort of formal linguist who wants to model the two rewrites as nonequal but equivalent, with some way to keep track of how different equivalent rewrites relate to one another. Bob might want this for example because he wants to show that head-first rewrite strategies are the same as tail-first, so he wants to keep the observation that the two rewrites are equivalent in that they have the same source and target, while keeping the precise order of rewrites distinct. This order-independence of disjoint or non-interfering rewrites is reflected in the interchange law for monoidal categories, which in the case of our example is depicted as:

\[\tikzfig{ncat/bobfat}\]

In fact, Bob gets to express a new kind of rewrite in the middle: the kind where two non-conflicting rewrites happen \emph{concurrently}. The important aspect of Bob's view over Alice's is that equalities have been replaced by isomorphisms between syntactically inequal rewrites. This demotion of equalities to isomorphisms means that Bob is dealing with a \emph{weak} 1-object-2-category; Bob does have 3-cells that relate different 2-cells with the same source and target, but all of Bob's $n$-cells for $n \geq 3$ are isomorphisms, rather than equalities.

\subsection{A context free grammar to generate \texttt{Alice sees Bob quickly run to school}}

We can describe a context-free grammar with the same combinatorial rewriting data that specifies planar string diagrams as we have been illustrating so far. One aspect of rewrite systems we adapt for now is the distinction between terminal and nonterminal symbols; terminal symbols are those after which no further rewrites are possible. We capture this string-diagrammatically by modelling terminal rewrites as 2-cells with target equal to the 1-cell identity of the 0-cell $\bullet$, which amounts to graphically terminating a wire. The generators subscripted $L$ (for \emph{label} or \emph{leaf}) correspond to terminals of the CFG, and represent a family of generators indexed by a lexicon for the language. The generators subscripted $i$ (for \emph{introducing a type}) correspond to rewrites of the CFG.

\[\scalebox{0.5}{\tikzfig{tree2gate/cfg/cfgsignature}}\]

Consider the sentence \texttt{Alice sees Bob quickly run to school}, which we take to be generated by the following context-free grammar derivation, read from left-to-right. We additionally depict the breakdown of the derivation in terms of rewrites of lower dimension from our signature.

\[\scalebox{0.5}{\tikzfig{tree2gate/cfg/bigcfgbreakdown}}\]

\[\scalebox{0.5}{\tikzfig{tree2gate/workedexample/bigcfg}}\]

\[\tikzfig{tree2gate/workedexample/biginter}\]

\subsection{Tree Adjoining Grammars}

Here is a formal but unenlightening definition of tree adjoining grammars, which we will convert to diagrams.

\begin{defn}[Tree Adjoining Grammar: Classic Computer Science style]
A \textbf{TAG} is a tuple $(\mathcal{N}, \mathcal{N}^\downarrow, \mathcal{N}^\bullet, \Sigma, \mathcal{I}, \mathcal{A}, \mathbf{s} \in \mathcal{N})$. These denote, respectively:
\begin{itemize}
	\item{ The \emph{non-terminals}:
		\begin{itemize}
			\item{A set of \emph{non-terminal symbols} $\mathcal{N}$ -- these stand in for grammatical types such as $\texttt{NP}$ and $\texttt{VP}$.}
			\item{A bijection $\downarrow: \mathcal{N} \rightarrow \mathcal{N}^\downarrow$ which acts as $\texttt{X} \mapsto \texttt{X}^\downarrow$. Nonterminals in $\mathcal{N}$ are sent to marked counterparts in $\mathcal{N}^\downarrow$, and the inverse sends marked nonterminals to their unmarked counterparts. These markings are \emph{substitution markers}, which are used to indicate when certain leaf nodes are valid targets for a substitution operation -- discussed later.}
			\item{A bijection $\bullet: \mathcal{N} \rightarrow \mathcal{N}^\bullet$ -- the same idea as above. This time to mark \emph{foot nodes} on auxiliary trees, which is structural information used by the adjoining operation -- discussed later.}
	\end{itemize}
	}
	\item{A set of \emph{terminal symbols} $\Sigma$ -- these stand in for the words of the natural language being modelled.}
	\item{The \emph{elementary trees}:
		\begin{itemize}
			\item{A set of \emph{initial trees} $\mathcal{I}$, which satisfy the following constraints:
			\begin{itemize}
				\item{The interior nodes of an initial tree must be labelled with nonterminals from $\mathcal{N}$}
				\item{The leaf nodes of an initial tree must be labelled from $\Sigma \cup \mathcal{N}^{\downarrow}$}
			\end{itemize}
			}
	\item{A set of \emph{auxiliary trees} $\mathcal{A}$, which satisfy the following constraints:
		\begin{itemize}
			\item{The interior nodes of an auxiliary tree must be labelled with nonterminals from $\mathcal{N}$}
			\item{Exactly one leaf node of an auxiliary tree must be labelled with a foot node $\texttt{X}^{\bullet} \in \mathcal{N}^{\bullet}$; moreover, this labelled foot node must be the marked counterpart of the root node label of the tree.}
			\item{All other leaf nodes of an auxiliary tree are labelled from $\Sigma \cup \mathcal{N}^{\downarrow}$}
		\end{itemize}
		}
	\end{itemize}
	}
\end{itemize}
There are two operations to build what are called \emph{derived trees} from elementary and derived trees. These operations are called \emph{substitution} and \emph{adjoining}.
\begin{itemize}
	\item{\emph{Substitution} replaces a substitution marked leaf node $\texttt{X}^\downarrow$ in a tree $\alpha$ with another tree $\alpha'$ that has $\texttt{X}$ as a root node.}
	\item{\emph{Adjoining} takes auxiliary tree $\beta$ with root and foot nodes $\texttt{X},\texttt{X}^\bullet$, and a derived tree $\gamma$ at an interior node $\texttt{X}$ of $\gamma$. Removing the $\texttt{X}$ node from $\gamma$ separates it into a parent tree with an $\texttt{X}$-shaped hole for one of its leaves, and possibly multiple child trees with \texttt{X}-shaped holes for roots. The result of adjoining is obtained by identifying the root of $\beta$ with the $\texttt{X}$-context of the parent, and making all the child trees children of $\beta$s foot node $\texttt{X}^\bullet$.}
\end{itemize}
\end{defn}

The essence of a tree-\emph{adjoining} grammar is as follows: whereas for a CFG one grows the tree by appending branches and leaves at the top of the tree (substitution), in a TAG one can also sprout subtrees from the middle of a branch (adjoining). Now we show that this gloss is more formal than it sounds, by the following steps. First we show that the 2-categorical data of a CFG can be transformed into 3-categorical data -- which we call \emph{Leaf-Ansatz} -- which presents a rewrite system that obtains the same sentences as the CFG, by a bijective correspondence between composition of 2-cells in the CFG and constructed 3-cells in the leaf-ansatz. These 3-cells in the leaf ansatz correspond precisely to the permitted \emph{substitutions} in a TAG. Then we show how to model \emph{adjoining} as 3-cells. Throughout we work with a running example, the CFG grammar introduced earlier.

\begin{construction}[Leaf-Ansatz of a CFG]
Given a signature $\mathfrak{G}$ for a CFG, we construct a new signature $\mathfrak{G}'$ which has the same 0- and 1-cells as $\mathfrak{G}$. See the dashed magenta arrows in the schematic below. For each 1-cell wire type $\texttt{X}$ of $\mathfrak{G}$, we introduce a \emph{leaf-ansatz} 2-cell $\texttt{X}^\downarrow$. For each leaf 2-cell $\texttt{X}_L$ in $\mathfrak{G}$, we introduce a renamed copy $\texttt{X}'_L$ in $\mathfrak{G}'$. Now see the solid magenta arrows in the schematic below. We construct a 3-cell in $\mathfrak{G}'$ for each 2-cell in $\mathfrak{G}$, which has the effect of systematically replacing open output wires in $\mathfrak{G}$ with leaf-ansatzes in $\mathfrak{G}'$.
\[\scalebox{0.5}{\tikzfig{tree2gate/tag/CFGtoTAGpattern}}\]
\end{construction}

\begin{example}
The leaf-ansatz construction just makes formal the following observation: there are multiple equivalent ways of modelling terminal symbols in a rewrite system considered string-diagrammatically. One way (which we have already done) is to treat non-terminals as wires and terminals as effects, so that the presence of an open wire available for composition visually indicates non-terminality. Another (which is the leaf-ansatz construction) treats all symbols in a rewrite system as leaves, where bookkeeping the distinction between (non-)terminals occurs in the signature. So for a sentence like \texttt{Bob drinks}, we have the following derivations that match step for step in the two ways we have considered.
\[\scalebox{0.5}{\tikzfig{tree2gate/tag/leafansatzintuition}}\]
\end{example}

\begin{proposition}[Leaf-ansatzes of CFGs are precisely TAGs with only initial trees and substitution]\label{prop:cfgastag1}
\begin{proof}
By construction. Consider a CFG given by 2-categorical signature $\mathfrak{G}$, with leaf-ansatz signature $\mathfrak{G}'$. The types $\texttt{X}$ of $\mathfrak{G}$ become substitution marked symbols $\texttt{X}^{\downarrow}$ in $\mathfrak{G}'$. The trees $\texttt{X}_i$ in $\mathfrak{G}$ become initial trees $\texttt{X}^0$ in $\mathfrak{G}'$. The 3-cells $\texttt{X}_s$ of $\mathfrak{G}'$ are precisely substitution operations corresponding to appending the 2-cells $\texttt{X}_i$ of $\mathfrak{G}$.
\end{proof}
\end{proposition}

\begin{example}[Leaf-ansatz signature of \texttt{Alice sees Bob quickly run to school} CFG]
\[\scalebox{0.5}{\tikzfig{tree2gate/tag/CFGasTAGsign}}\]
\end{example}

\begin{example}[Adjoining is sprouting subtrees in the middle of branches]
One way we might obtain the sentence \texttt{Bob runs to school} is to start from the simpler sentence \texttt{Bob runs}, and then refine the verb \texttt{runs} into \texttt{runs to school}. This refinement on part of an already completed sentence is not permitted in CFGs, since terminals can no longer be modified. The adjoining operation of TAGs gets around this constraint by permitting rewrites in the middle of trees, as follows:
\[\scalebox{0.5}{\tikzfig{tree2gate/tag/3compareintuition}}\]
\end{example}

\begin{example}[TAG signature of \texttt{Alice sees Bob quickly run to school}]
The highlighted 2-cells are auxiliary trees that replace CFG 2-cells for verbs with sentential complement, adverbs, and adpositions. The highlighted 3-cells are the tree adjoining operations of the auxiliary trees.
\[\scalebox{0.5}{\tikzfig{tree2gate/tag/tagsignature}}\]
\end{example}

The construction yields as a corollary an alternate proof of Theorem [Joshi 6.1.1...]...

\begin{corollary}
For every context-free grammar $\mathfrak{G}$ there exists a tree-adjoining grammar $\mathfrak{G}'$ such that $\mathfrak{G}$ and $\mathfrak{G}'$ are strongly equivalent -- both formalisms generate the same set of strings (weak equivalence) and the same abstract syntactic structures (in this case, trees) behind the strings (strong equivalence).
\begin{proof}
Proposition \ref{prop:cfgastag1} provides one direction of both equivalences. For the other direction, we have to show that each auxiliary tree (a 2-cell) and its adjoining operation (a 3-cell) in $\mathfrak{G}'$ corresponds to a single 2-cell tree of some CFG signature $\mathfrak{G}$, which we demonstrate by construction. See the example above; the highlighted 3-cells of $\mathfrak{G}'$ are obtained systematically from the auxiliary 2-cells as follows: the root and foot nodes $\texttt{X},\texttt{X}^\bullet$ indicate which wire-type to take as the identity in the left of the 3-cell, and the right of the 3-cell is obtained by replacing all non-$\texttt{X}$ open wires $\texttt{Y}$ with their leaf-ansatzes $\texttt{Y}^\downarrow$. This establishes a correspondence between any 2-cells of $\mathfrak{G}$ considered as auxiliary trees in $\mathfrak{G}'$.
\end{proof}
\end{corollary}