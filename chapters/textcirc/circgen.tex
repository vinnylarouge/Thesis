\newpage

\section{A generative grammar for text circuits}

\subsection{A circuit-growing grammar}

There are many different ways to write an $n$-categorical signature that generates circuits. Mostly as an illustration of expressive capacity, I will provide a signature where the terms "surface" and "deep" structure are taken literally as metaphors; the generative grammar will grow a line of words in (roughly) syntactic order, and like mushrooms on soil, the circuits will appear as the mycelium underneath words.\\

\newthought{Simplifications}: Propositions only, no determiners, only one tense, no morphological agreement between nouns and their verbs and referring pronouns, and we assume that adverbs, adverbial adjunctions, and adjectives stack indefinitely and without further order requirements; e.g. \texttt{Yesterday yesterday yesterday Alice happily secretly finds red big toy shiny car that he gives to Bob.} we consider grammatical enough. For now, we consider only the case where adjectives and adverbs appear before their respective noun or verb. Note that all of these limitations can principle be overcome by the techniques we developed in Section \ref{sec:ncat} for restricted tree-adjoining and links.

\marginnote{
\begin{defn}[Lexicon]
We define a lexicon $\mathcal{L}$ to be a tuple $(\mathbf{N}, \mathbf{V}_1, \mathbf{V}_2, \mathbf{V}_{\texttt{S}}, \mathbf{A}_{\texttt{N}}, \mathbf{A}_{\texttt{V}}, \mathbf{A}_{\texttt{S}}, \mathbf{C})$
\end{defn}
}

\marginnote{
Where:
\begin{itemize}
\item $\mathbf{N}$ is a set of \emph{proper nouns}
\item $\mathbf{V}_1$ is a set of \emph{intransitive verbs}
\item $\mathbf{V}_2$ is a set of \emph{transitive verbs}
\item $\mathbf{V}_{\texttt{S}}$ is a set of \emph{sentential-complement verbs}
\item $\mathbf{A}_{\texttt{N}}$ is a set of \emph{adjectives}
\item $\mathbf{A}_{\texttt{V}}$ is a set of \emph{adverbs}
\item $\mathbf{A}_{\texttt{S}}$ is a set of \emph{adverbial adpositions}
\item $\mathbf{C}$ is a set of \emph{conjunctions}
\end{itemize}
}

\clearpage

\subsection{Text structure}

\begin{marginfigure}
\centering
\[
\resizebox{\textwidth}{!}{\tikzfig{mushroom/howtoread}}
\]
\caption{\textbf{How to read the diagrams in this section:} we will be making heavy use of pink and purple bubbles as frames to construct circuits. We will depict the bubbles horizontally, as we are permitted to by compact closure, or by reading diagrams with slightly skewed axes.}
\end{marginfigure}

We work in a dimension where wires behave symmetric monoidally by homotopy, and further assume compact closure rewrite rules for all wire-types. Our strategy is to generate "bubbles" for sentences, within which we can grow circuit structure piecemeal. We will only express the rewrite rules; the generators of lower dimension are implicit. We aim to recover the linear ordering of words in text (essential to any syntax) by traversing the top surface of a chain of bubbles representing sentence structure in text -- this order will be invariant despite compact closure. First we introduce rules to capture the broad structure of text as a structured collection of \emph{simple sentences}, where a simple sentence is one that only contains a single verb.\\

Starting from a single simple sentence bubble, we may always append another to obtain a list. However, more than list structure, we may have two simple sentences joined by a conjunction, e.g. \texttt{Alice dances \underline{while} Bob drinks}. In the same vein, the unary case of a conjunction of simple sentences is an adverbial adposition, e.g. \texttt{\underline{Yesterday} Alice dance(d)}. Finally, we may have verbs that take a sentential complement rather than a noun phrase, e.g. \texttt{Alice \underline{sees} Bob dance}; these verbs require nouns, which we depict as wires spanning bubbles.

\begin{figure}[h!]\label{fig:sentbestiary}
\centering
\[
\resizebox{\textwidth}{!}{\tikzfig{mushroom/bestiary2}}
\]
\caption{The blue-dotted guards are for circuit-translation later, to indicate the contents of boxes.}
\end{figure}

\marginnote{
\begin{defn}[Simple sentence structure]
In prose, a text is a list of sentences. A sentence can be a pair of sentences, each guarded by scopes with a conjunction in between. A sentence can carry an adverbial adposition at the start. As a CFG, these considerations are respectively depicted as:
\end{defn}
\[
\resizebox{\marginparwidth}{!}{\tikzfig{mushroom/textstructureCFG}}
\]
}

\marginnote{
\begin{proposition}
The sentence introduction rules are complete with respect to simple sentence structure.
\begin{proof}
Excluding verbs with sentential complements to be dealt with later, the sentence-introduction rewrites graphically correspond to the CFG rules.
\end{proof}
\end{proposition}
\[\resizebox{\marginparwidth}{!}{\tikzfig{mushroom/sentstructcomplete}}\]
}

\subsection{Simple sentences}

Simple sentences are sentences that only contain a single intransitive or transitive verb. Simple sentences will contain at least one noun, and may optionally contain adjectives, adverbs, and adpositions. The rules for generating simple sentences are as follows:

\[
\resizebox{\textwidth}{!}{\tikzfig{mushroom/simplesentences}}
\]

The $\texttt{N}_\uparrow$-intro rule introduces new unsaturated nouns from the end of a simple sentence. The \textcolor{green}{\texttt{IV}}-intro rule applies when there is precisely one unsaturated noun in the sentence, and the \textcolor{green}{\texttt{TV}}-intro rule applies when there are precisely two. Both verb-introduction rules saturate their respective nouns, which we depict with a black bulb. Adjectives may be introduced immediately preceding saturated nouns, and adverbs may be introduced immediately preceding any kind of verb. The position of adpositions in English is not context-free, and requires some aid from the deep structure beneath the surface. The $\textcolor{blue}{\texttt{ADP}}_{\textcolor{green}{\texttt{V}}}$-tendril rule allows an unsaturated adposition to appear immediately after a verb; a bulb may travel by homotopy to the right, seeking an unsaturated noun. Conversely, the bidirectional $\textcolor{blue}{\texttt{ADP}}_{\texttt{N}}$-tendril rule sends a mycelic tendril to the left, seeking a verb. The two pass-rules allow unsaturated adpositions to swap past saturated nouns and adjectives; note that by construction, neither verbs nor adverbs will appear in a simple sentence to the right of a verb, so unsaturated adpositions will move right until encountering an unsaturated noun, and in case it doesn't, the tendril- and pass- rules are bidirectional and hence reversible.



\subsection{Noun distribution and linking}

\marginnote{
We require a definition of coreference structure in order verify that our rewrite rules are complete with respect to them. One reasonable constraint on coreference in text is that coreference always goes backwards. One way to express this constraint mathematically is the following:
\begin{defn}[Coreference structure]
In a text with $K \in \mathbb{N}$ distinct nouns and their coreferents, the linear order of noun+referents in a text corresponds to a list of positive integers that:
\begin{enumerate}
\item Starts with 1
\item Contains every $k \in [1\cdots K]$
\item For all $k \in [1 \cdots K]$, the head of the list up to the first occurence of $k$ contains all $j < k$.
\end{enumerate}
\end{defn}
}

\marginnote{
\begin{proposition}
The noun-introduction and manipulation rules are expressively complete with respect to coreference structure.
\begin{proof}

\end{proof}
\end{proposition}
}

Now we deal with nouns. The major complication here is the accommodation of coreference. We want to keep track of which (pro)nouns share a reference so that we can ultimately eliminate the distinction between e.g. \texttt{Bob likes himself} and \texttt{Bob likes Bob}. So we will generate \emph{unsaturated} nouns and their coreference structure first. We will further ask for a distinction between \emph{unsaturated} and \emph{saturated} nouns -- only the latter of which are ready to be labelled -- which will resolve a minor complication regarding the context-sensitivity of adpositions. We define three classes of rules for nouns. The first is the introduction of novel unsaturated nouns, which can occur in any sentence-bubble. The second class handles the introduction of a new, coreferentially-linked noun to the right of an extant noun. The third concerns swapping the positions of unsaturated and linked nouns within and between bubbles.

\[
\resizebox{\textwidth}{!}{\tikzfig{mushroom/nounbestiary_new}}
\]



\newpage

Nouns require verbs in order to be saturated. From left-to-right; if there is precisely one unlabelled noun, we may introduce an unlabelled intransitive verb and saturate the noun so that it is now ready to grow a label; or if there are two unlabelled nouns, we may introduce an unlabelled transitive verb on the surface and saturate the two nouns that will be subject and object; and verbs may be labelled.
\[
\tikzfig{mushroom/simpbestiary}
\]

\subsection{Modifiers}

Modifiers are optional parts of sentences that modify (and hence depend on there being) nouns and verbs. We consider adjectives, adverbs, and adpositions. From left to right; we allow adjectives to sprout immediately before a saturated noun, and we allow adverbs to sprout immediately before any verb.

\[
\tikzfig{mushroom/adjadv}
\]

Adpositions modify verbs by tying in an additional noun argument; e.g. while \texttt{runs} is intransitive, \texttt{runs towards} behaves as a transitive verb. Some more advanced technology is required to place adpositions and their thematic nouns in the correct linear order on the surface. In the left column; an adposition tendril can sprout from a verb via an unsaturated adposition, seeking an unsaturated noun to the right; an unsaturated noun can sprout an tendril seeking a verb to connect to on the left. Both of these rewrites are bidirectional, as tendrils might attempt connection but fail, and so be retracted. In the centre, when an unsaturated adposition and its tendril find an unsaturated noun, they may connect, saturating the adposition so that it is ready to label. In the right column; an unsaturated adposition may move past a saturated noun in the same sentence, which allows multiple adpositions for the same verb; finally, a saturated adposition can be labelled.

\[
\tikzfig{mushroom/adpbestiary}
\]

\subsection{Rewriting to circuit-form}

\newthought{Resolving references}

\[
\tikzfig{mushroom/pronbestiary}
\]

\[
\tikzfig{mushroom/pronres}
\]

\newthought{Connecting circuits}

\newpage
\subsection{Putting it all together}

\begin{figure}[h!]
\centering
\[
\resizebox{\textwidth}{!}{\tikzfig{mushroom/bigex1}}
\]
\caption{Starting from the initial sentence bubble, we generate a new sentence, introduce some nouns and an SCV, and then we connect our references at the bottom.}
\end{figure}

\begin{figure}[h!]
\centering
\[
\resizebox{\textwidth}{!}{\tikzfig{mushroom/bigex2}}
\]
\caption{Then we introduce intransitive verbs to saturate nouns, and we may also sprout some modifying adjectives and adverbs.}
\end{figure}

\begin{figure}[h!]
\centering
\[
\resizebox{\textwidth}{!}{\tikzfig{mushroom/bigex3}}
\]
\caption{For the remaining unsaturated noun, we use the adposition introduction rules to sprout tendrils off of the other verb in the bubble, and connect.}
\end{figure}

\begin{figure}[h!]
\centering
\[
\resizebox{\textwidth}{!}{\tikzfig{mushroom/bigex4}}
\]
\caption{All of the non-noun words may now be labelled.}
\end{figure}

\begin{figure}[h!]
\centering
\[
\resizebox{\textwidth}{!}{\tikzfig{mushroom/bigex5}}
\]
\caption{To begin assigning nouns, observe that by compact closure of the bubble boundaries, we can deform the diagram to obtain suitable forms for our local rewrite rules for link-generation at the bottom.}
\end{figure}

\begin{figure}[h!]
\centering
\[
\resizebox{\textwidth}{!}{\tikzfig{mushroom/bigex6}}
\]
\caption{Now we can introduce our noun labels and linearise our link structure.}
\end{figure}

\begin{figure}[h!]
\centering
\[
\resizebox{\textwidth}{!}{\tikzfig{mushroom/bigex7}}
\]
\caption{Once the link structure is linearised, we can undo the deformation, and propagate links to the surface.}
\end{figure}

\begin{figure}[h!]
\centering
\[
\resizebox{\textwidth}{!}{\tikzfig{mushroom/bigex8}}
\]
\caption{The bubbles may be rearranged to respect circuit-form, such that the propagation of noun labels to the surface traces out the wires of the end circuit.}
\end{figure}

\clearpage
\newpage
\subsection{Extensions I: relative and reflexive pronouns}

\newthought{Subject relative pronouns}

\begin{example}

\end{example}

\newthought{Object relative pronouns}

\begin{example}

\end{example}

\newthought{Reflexive pronouns}

\begin{example}

\end{example}

\subsection{Extensions II: grammar equations}

\newthought{Attributive vs. predicative modifiers}

\begin{example}

\end{example}

\newthought{Copulas}

\begin{example}

\end{example}

\newthought{Possessive pronouns}

\begin{example}

\end{example}

\subsection{Extensions III: higher-order modifiers}

\newthought{Intensifiers}

\begin{example}

\end{example}

\newthought{Comparatives}

\begin{example}

\end{example}

\subsection{Equivalence to internal wirings}

\subsection{Text circuit theorem}

\subsection{Related work}