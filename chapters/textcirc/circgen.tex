\newpage

\section{A generative grammar for text circuits}\label{sec:gencirc}

\subsection{A circuit-growing grammar}

\marginnote{
\begin{defn}[Lexicon]\label{defn:lex}
We define a limited lexicon $\mathcal{L}$ to be a tuple of disjoint finite sets $(\mathbf{N}, \mathbf{V}_1, \mathbf{V}_2, \mathbf{V}_{\texttt{S}}, \mathbf{A}_{\texttt{N}}, \mathbf{A}_{\texttt{V}}, \mathbf{C})$
\end{defn}
}

\marginnote{
Where:
\begin{itemize}
\item $\mathbf{N}$ is a set of \emph{proper nouns}
\item $\mathbf{V}_1$ is a set of \emph{intransitive verbs}
\item $\mathbf{V}_2$ is a set of \emph{transitive verbs}
\item $\mathbf{V}_{\texttt{S}}$ is a set of \emph{sentential-complement verbs}
\item $\mathbf{A}_{\texttt{N}}$ is a set of \emph{adjectives}
\item $\mathbf{A}_{\texttt{V}}$ is a set of \emph{adverbs}
\item $\mathbf{C}$ is a set of \emph{conjunctions}
\end{itemize}
}

\begin{marginfigure}
\centering
\[
\resizebox{\textwidth}{!}{\tikzfig{mushroom/howtoread}}
\]
\caption{\textbf{How to read the diagrams in this section:} we will be making heavy use of pink and purple bubbles as frames to construct circuits. We will depict the bubbles horizontally, as we are permitted to by compact closure, or by reading diagrams with slightly skewed axes.}
\end{marginfigure}

\begin{marginfigure}
\centering
\[
\tikzfig{mushroom/sintro}
\]
\caption{Every derivation starts with a single blank sentence bubble, to which we may append more blank sentences.}
\end{marginfigure}

There are many different ways to write a weak $n$-categorical signature that generates circuits. Mostly as an illustration of expressivity, I will provide a signature where the terms "surface" and "deep" structure are taken literally as metaphors; the generative grammar will grow a line of words in syntactic order, and like mushrooms on soil, the circuits will behave as the mycelium underneath the words. It won't be the most efficient way to do it in terms of the number of rules to consider, but it will look nice and we'll be able to reason about it easily. As a note to the reader, there are a lot of worked examples in this section, so if you get confused about why a rule is the way it is, just skip over it and hopefully there will be a clarifying example later on.\\

\newthought{Simplifications and limitations}: For now we only consider word types as in Definition \ref{defn:lex}, though we will see how to engineer extensions later. We only deal with propositional statements, without determiners, in only one tense, with no morphological agreement between nouns and their verbs and referring pronouns, and we assume that adverbs and adjectives stack indefinitely and without further order requirements: e.g. \texttt{Alice happily secretly finds red big toy shiny car that he gives to Bob} is a sentence we consider grammatical enough. For now, we consider only the case where adjectives and adverbs appear before their respective noun or verb. Note that all of these limitations apart from the limited lexicon can principle be overcome by the techniques we developed in Section \ref{sec:weakn} for restricted tree-adjoining and links. As a historical remark, generative-transformational grammars fell out of favour linguistically due to the problem of overgeneration: the generation of nonsense or unacceptable sentences in actual language use. We're undergenerating and overgenerating at the same time, but we're also not concerned with empirical capture: we only require a concrete mathematical basis to build interesting things on top of. On a related note, there's zero chance that this particular circuit-growing grammar even comes close to how language is actually produced by humans, and I have no idea whether a generalised graph-rewriting approach is cognitively realistic.

\newthought{Mathematical assumptions}: We work in a dimension where wires behave symmetric monoidally by homotopy, and further assume strong compact closure rewrite rules for all wire-types. Our strategy will be to generate "bubbles" for sentences, within which we can grow circuit structure piecemeal. We will only express the rewrite rules; the generators of lower dimension are implicit. We aim to recover the linear ordering of words in text (essential to any syntax) by traversing the top surface of a chain of bubbles representing sentence structure in text -- this order will be invariant up to compact closed isomorphisms. The diagrammatic consequence of these assumptions is that we will be working with a conservative generalisation of graph-rewriting defined by local rewriting rules. The major distinction is that locality can be redefined up to homotopy, which allows locally-defined rules to operate in what would be a nonlocal fashion in terms of graph neighbourhoods, as in Figure \ref{fig:locality}. The minor distinction is that rewrite rules are sensitive to twists in wires and the radial order in which wires emanate from nodes, though it is easy to see how these distinctions can be circumvented by additional by imposing the equivalent of commutativity relations as bidirectional rewrites. It is worth remarking that one can devise weak n-categorical signatures to simulate turing machines, where output strings are e.g. 0-cells on a selected 1-cell, so rewrite systems of the kind we propose here are almost certainly expressively sufficient for anything; the real benefit is the interpretable geometric intuitions of the diagrams.

\begin{figure}[h!]\label{fig:locality}
\centering
\[
\resizebox{\textwidth}{!}{\tikzfig{mushroom/locality}}
\]
\caption{In this toy example, obtaining the same rewrite that connects the two yellow nodes with a purple wire using only graph-theoretically-local rewrites could potentially require an infinite family of rules for all possible configurations of pink and cyan nodes that separate the yellow, or would otherwise require disturbing other nodes in the rewrite process. In our setting, strong compact closure homotopies handle navigation between different spatial presentations so that a single rewrite rule suffices: the source and target notated by dotted-black circles. Despite the expressive economy and power of finitely presented signatures, we cannot "computationally cheat" graph isomorphism: formally we must supply the compact-closure homotopies as part of the rewrite, absorbed and hidden here by the $\simeq$ notation.}
\end{figure}

\newthought{The broad plan}: We'll first display the rules and demonstrate their usage, then we'll prove the text circuit theorem by relating our rules to text.

\newthought{The rules:} We start with simple sentences that only contain a single intransitive or transitive verb, which correspond to gates and typed-boxes. Then we consider more general sentences with nesting sentential structure, which correspond to untyped-boxes. Then we introduce coreferential structure on nouns, which corresponds to symmetric monoidal composition of text circuits.

\newthought{The theorem:} We characterise the expressive capacity of our rules for simple and complex sentences in terms of a context-sensitive grammar that corresponds to the surface structure of the derivations, which tells us that the generated text is sensible. Then we provide a mathematical characterisation of coreferential structure and a completeness result of our rules with respect to processive connectivity, which tells us that all circuit connectivity patterns are achievable. Then we (re)state and prove the text circuit theorem: that the fragment of language generated by the grammar surjects onto text circuits.

\clearpage

\begin{myboxB}
\begin{rules}[Simple sentences]\label{rules:simp}
Simple sentences are sentences that only contain a single intransitive or transitive verb. Simple sentences will contain at least one noun, and may optionally contain adjectives, adverbs, and adpositions. The rules for generating simple sentences are as follows:

\[
\resizebox{0.9\textwidth}{!}{\tikzfig{mushroom/simplesentences}}
\]

The $\texttt{N}_\uparrow$-intro rule introduces new unsaturated nouns. The \textcolor{green}{\texttt{IV}}-intro and \textcolor{green}{\texttt{TV}}-intro rules apply when there are precisely one or two unsaturated nouns in the sentence respectively, saturating their respective nouns. Adjectives may be introduced immediately preceding saturated nouns. Adverbs may be introduced immediately preceding verbs. To capture context-sensitive placement of adposition introductions, the $\textcolor{blue}{\texttt{ADP}}_{\textcolor{green}{\texttt{V}}}$-tendril rule allows an unsaturated adposition to succeed a verb; a bulb may travel by homotopy to the right seeking an unsaturated noun. Conversely, the bidirectional $\textcolor{blue}{\texttt{ADP}}_{\texttt{N}}$-tendril rule sends a mycelic tendril to the left, seeking a verb. The two pass-rules allow unsaturated adpositions to swap past saturated nouns and adjectives. By construction, neither verbs nor adverbs will appear in a simple sentence to the right of a verb, so unsaturated adpositions will move right until encountering an unsaturated noun. In case it doesn't, the tendril- and pass- rules are reversible.
\end{rules}
\end{myboxB}
\clearpage

\begin{myboxB}
\begin{rules}[Complex sentences]\label{rules:comp}
Now we consider two refinements; conjunctions, and verbs that take sentential complements. We may have two sentences joined by a conjunction, e.g. \texttt{Alice dances \underline{while} Bob drinks}. We may also have verbs that take a sentential complement rather than a noun phrase, e.g. \texttt{Alice \underline{sees} Bob dance}; these verbs require nouns, which we depict as wires spanning bubbles.
\[
\resizebox{0.75\textwidth}{!}{\tikzfig{mushroom/Sbestiary}}
\]
The dotted-blue wires do not contentfully interact with anything else, but the points at which they connect on the surface magenta wire serve as blockers that prevent overgeneration cases where adpositional phrases might interject between \texttt{SCV} verbs and their sentential complement, e.g. \textcolor{red}{\texttt{Alice sees \underline{at lunch} Bob drink}}. Later, they serve as visual indicators for the contents of untyped-boxes in text circuits.
\end{rules}
\end{myboxB}
\clearpage

\begin{myboxR}
\begin{example}[\texttt{sober} $\alpha$ \texttt{sees drunk} $\beta$ \texttt{clumsily dance.}]\label{ex:soberA}
Now we can see our rewrites in action for sentences. As a matter of convention -- reflected in how the various pass- rules do not interact with labels -- we assume that labelling occurs after all of the words are saturated. We have still not introduced rules for labelling nouns: we delay their consideration until we have settled coreferential structure. For now they are labelled informally with greeks.
\end{example}
\[
\resizebox{0.75\textwidth}{!}{\tikzfig{mushroom/soberA}}
\]
\end{myboxR}
\clearpage

\begin{myboxR}
\begin{example}[$\alpha$ \texttt{laughs at} $\beta$]\label{ex:Alaughs}
Adpositions form by first sprouting and connecting tendrils under the surface. Because the tendril- and pass- rules are bidirectional, extraneous tendrils can always be retracted, and failed attempts for verbs to find an adpositional unsaturated noun argument can be undone. Though this seems computationally wasteful, it is commonplace in generative grammars to have the grammar overgenerate and later define the set of sentences by restriction, which is reasonable so long as computing the restriction is not computationally hard. In our case, observe that once a verb has been introduced and its argument nouns have been saturated, only the introduction of adpositions can saturate additionally introduced unsaturated nouns. Therefore we may define the finished sentences of the circuit-growing grammar to be those that e.g. contain no unsaturated nodes on the surface, which is a very plausible linear-time check by traversing the surface. It is an invariant of the rules by construction that left-to-right traversal of the surface is always meaningful and yields the desired linear ordering of text in finished derivations.
\end{example}
\[
\resizebox{0.5\textwidth}{!}{\tikzfig{mushroom/Alaughs}}
\]
\end{myboxR}
\clearpage

\begin{myboxB}
\begin{rules}[Coreferential structure and noun labels]\label{rules:coref}
%\bR TODO: need more linked-intro variants to handle leaving right-side of conjunction, nested SCV, CNJ to SCV, and SCV to CNJ. Consider using the same graywire convention to simplify case analysis? \e
% dealt with in prose.
\[
\resizebox{0.75\textwidth}{!}{\tikzfig{mushroom/nounbestiary_newnew}}
\]
The linked-\texttt{N}-intro rules introduce a new unsaturated noun in the next sentence that coreferences the noun in the previous sentence that generated it, with variants for each pair of sentences involved. We depict three: simple to simple, between \textcolor{blue}{\texttt{CNJ}}-related sentences, and from either a \textcolor{blue}{\texttt{CNJ}} or \textcolor{green}{\texttt{SCV}} to a simple sentence.\\

\texttt{N}-shift rules allow any unsaturated noun to move into the next sentence, again with variants for different pairs of sentences. Observe that nouns with a forward coreference have two dotted-black wires leaving the root of their wires, which distinguishes them from nouns that only have a backward coreference or no coreference at all, which only have a single dotted-black wire leaving the root of their wire.\\

The \texttt{N}-swap rule variants allow a unsaturated noun with no forward coreferences to swap places with any unsaturated noun that immediately succeeds it.
\end{rules}
\end{myboxB}
\clearpage

\begin{myboxB}
\begin{rules}[Labelling nouns]\label{rules:labels}
When the structure of coreferences is set, we propagate noun labels from the head of each list. The rules for noun-label propagation are as follows:
\[
\resizebox{\textwidth}{!}{\tikzfig{mushroom/nounbestiary_newnew2}}
\]
The $n \in \mathbf{N}$ notation indicates a family of rewrites (and generators) for each noun in the lexicon. Link-label assigns a noun to a diagrammatically linked collection of coreferent nouns, and link-propagation is a case analysis that copies a link label and distributes is across coreferent nouns. Link-rise is a case analysis to connect labels to the surface, and finally \texttt{N}-label allows a saturated noun to inherit the label of its coreference class, which may either be a noun \texttt{n} or a pronoun appropriate for the noun, notated $^\texttt{*}\texttt{n}$
\end{rules}
\end{myboxB}
\clearpage

\begin{myboxR}
\begin{example}[\texttt{sober Alice sees Bob clumsily dance. She laughs at him.}]\label{ex:corefex1}
We start the derivation by setting up the sentence structure using \texttt{S}- and \texttt{SCV}-intro rules, and two instances of \texttt{N}-intro, one for Alice, and one for Bob. Observe how the \texttt{N}-intro for Bob occurs within the subsentence scoped over by the \texttt{SCV}-rule.
\end{example}
\[
\resizebox{0.75\textwidth}{!}{\tikzfig{mushroom/corefex1}}
\]
By homotopy, we can rearrange the previous diagram to obtain the source of the linked-\texttt{N}-intro rewrite in the dashed-box visual aid. Observe how we drag in the root of what is to be Alice's wire. Then we use the \texttt{IV}-intro in the second sentence, which sets up the surface structure \texttt{she laughs}, and the deep structure for bookkeeping that \texttt{she} refers to \texttt{Alice}.
\[
\resizebox{0.75\textwidth}{!}{\tikzfig{mushroom/corefex2}}
\]
By homotopy again, we can do the same for Bob, this time setting up for the $\gamma$ variant of linked-\texttt{N}-intro which handles the case when the spawning noun is within the scope of an SCV. Then by applying a series of $\texttt{N}_\uparrow$-swaps, the unsaturated noun is placed to the right of the intransitive verb phrase.
\[
\resizebox{0.75\textwidth}{!}{\tikzfig{mushroom/corefex3}}
\]
We've already done the surface derivation for the two sentences separately in Examples \ref{ex:soberA} and \ref{ex:Alaughs}; since neither of those derivations touch the roots of noun-wires, we can emulate those derivations and skip ahead.
\end{myboxR}
\clearpage

\begin{myboxR}

\[
\resizebox{0.75\textwidth}{!}{\tikzfig{mushroom/corefex4}}
\]
\end{myboxR}
\clearpage

\clearpage
\vfill

\begin{myboxB}
\begin{rules}[Text to circuit]\label{cons:wirejoin}
We turn finished text diagrams into text circuits by operating \emph{in situ}, with extra rules outside the grammatical system that handle connecting noun wires.
\[
\resizebox{\textwidth}{!}{\tikzfig{mushroom/sentjoin}} 
\]
\end{rules}
\end{myboxB}

\vfill

\newpage
\clearpage

\vfill

\begin{myboxR}
\begin{example}[Text to circuit in action]
In the first step below, by Lemmas \ref{prop:linkedlist} and \ref{prop:norefl}, we can always rearrange a finished text diagram such that the noun wires are processive. In the second step, use the first rewrite of Construction \ref{cons:wirejoin} to prepare the wires for connection.
\[
\resizebox{\textwidth}{!}{\tikzfig{mushroom/sentjoinex}} 
\]
In the third step, we just ignore the existence of the bubble-scaffolding and the loose scalars. We could in principle add more rewrites to melt the scaffolding away if we wanted to. In the fourth step, we apply the second and third rewrites of Construction \ref{cons:wirejoin} to connect the wires and eliminate nodules underneath labels. We can also straighten up the wires a bit and make them look proper. At this point, we're actually done, because the resulting diagram \emph{is already a text circuit up to a choice of notation}.
\[
\resizebox{\textwidth}{!}{\tikzfig{mushroom/sentjoinex2}} 
\]
\end{example}
\end{myboxR}

\begin{myboxB}
\begin{example}[Growing circuits directly]\label{ex:directgrowth}
Here is the combined content of Examples \ref{ex:soberA}, \ref{ex:Alaughs}, \ref{ex:corefex1}, this time presented directly as string-diagram rewrites treating text circuits as a primitive syntax, where dotted-boxes indicate sentence scopes.
\[\resizebox{\textwidth}{!}{\tikzfig{mushroom/shadowcircuit}}\]
\end{example}
The point is this: if it is possible to relate a string-diagram rewrite system to a linear syntax by graphical means, then (so I hope to have demonstrated) it may be done with a suitably specified weak $n$-category. So we may declare string diagram rewrite systems as we please and treat them as in-principle-formal generative grammars. I have demonstrated here how to construct a concrete witness for such a correspondence and to prove well-behaviour properties.
\end{myboxB}

\vfill

\newpage
\clearpage

\newpage

\subsection{Text circuit theorem}

\newthought{Now we would like to forget about the messy details of the circuit-growing grammar} and treat the text circuits themselves as a generative grammar for text, where the primitive operations are string-diagrammatic composition and nesting within boxes: we aim to prove that all text circuits that one might draw correspond to grammatically acceptable text. Moreover, this correspondence has to hold in such a way that connectively equivalent ways to present text circuits correspond to texts that "mean the same thing", e.g. up to rearrangement or grouping of sentences respecting constraints on pronominal reference.\\

\newthought{Our strategy has two phases.} Since we have already demonstrated that the circuit-growing grammar yields text circuits it will suffice to demonstrate grammatical acceptability for the circuit-growing grammar, and separately exhibit a well-behaved translation from arbitrary text circuits to circuit-growing grammars. Altogether this achieves our desired correspondence between text circuits and finished circuit-growing derivations, and text circuits will inherit the grammatical acceptability properties we demonstrate of circuit-growing grammars.\\

\newthought{We proceed in steps.} First, we relate the circuit-growing rules for simple and complex sentences to pedestrian CSGs, which establishes grammatical sensibility at the sentential level. Second, we analyse the pronoun connection rules of the circuit-growing grammar to establish that the text produced is sensible. Third, we expand the rules for our circuit-growing grammar so that all of the diagrammatic idiosyncrasies of text circuits have something to correspond to in the circuit-growing grammar. Finally, we demonstrate how to convert text circuits into finished circuit-growing derivations.
\clearpage

\begin{construction}[CSG for simple sentences]\label{dfn:simpCSG}
We may characterise simple sentences (containing only one verb) with the context-sensitive grammar in Figures \ref{fig:simpleCFG} and \ref{fig:simpleADP}.
\end{construction}
\begin{figure}[h!]\label{fig:simpleCFG}
\centering
\[
\resizebox{0.75\textwidth}{!}{\tikzfig{mushroom/simpleCFG}}
\]
\caption{
Reading each diagram from top-to-bottom, from left-to-right we have generators for intransitive verbs, transitive verbs, adjectives, and adverbs. Generators for verbs require a number of $\texttt{N}_{\uparrow}$ matching their arity as input, hence a CSG.
}
\end{figure}
\vspace{-1.5cm}
\begin{figure}[h!]\label{fig:simpleADP}
\centering
\[
\resizebox{0.75\textwidth}{!}{\tikzfig{mushroom/simpleADP}}
\]
\caption{
Adpositions require several helper-generators, which are the components within dashed boxes in the depicted example demonstrating the process of appending adpositions to an intransitive verb.
}
\end{figure}

\begin{proposition}\label{prop:simpsent}
Up to labels, the simple-sentence rules of the circuit-growing grammar are strongly equivalent to the CSG; in particular, they yield the same sentences.
\begin{proof}
By graphical correspondence between CSG rules and how the generators of the circuit-growing grammar changes surface nodes (Figure \ref{fig:correspondence}).
\end{proof}
\end{proposition}
\vspace{-1cm}
\begin{figure}[h!]\label{fig:correspondence}
\centering
\[
\resizebox{0.45\textwidth}{!}{\tikzfig{mushroom/simpcorr}}
\]
\caption{
Viewing nodes on the pink surface of circuit-growing grammar as 1-cells, each rewrite rule yields a 2-cell; e.g. the dashed-blue helper lines for adpositions correspond to the \textcolor{blue}{\texttt{ADP}}-pass rules in circuit-growing grammar. The correspondence between the \textcolor{green}{\texttt{IV}}-intro rules of both grammars is depicted.
}
\end{figure}

\begin{proposition}\label{prop:compsent}
Up to labels, Rules \ref{rules:simp} and \ref{rules:comp} for simple and complex sentences yield the same sentences as the combined CSG of Construction \ref{dfn:simpCSG} with the additional rules depicted in Figure \ref{fig:compsentCSG}.
\end{proposition}
\begin{proof}
Same correspondence as Proposition \ref{prop:simpsent}, ignoring the dotted-blue guards.
\end{proof}

\begin{figure}[h!]\label{fig:compsentCSG}
\centering
\[
\resizebox{0.75\textwidth}{!}{\tikzfig{mushroom/complexsentenceCSG}}
\]
\caption{
The first rule instantiates the left and right boundaries of a sentence, corresponding the starting bubble in circuit-growing grammar. The second corresponds to $\texttt{N}_\uparrow$-intro, the third $\textcolor{blue}{\texttt{CNJ}}$-intro, and the fourth $\textcolor{green}{\texttt{SCV}}$-intro.
}
\end{figure}

Now we address complex sentences and text by connecting nouns (Figure \ref{fig:nounconnection}). This presents no issue across distinct simple sentences, but a complication arises when connecting nouns within the same simple sentence with reflexive pronouns e.g. \texttt{Alice likes herself}. Reflexive coreference would violate of the processivity condition of string diagrams for symmetric monoidal categories. Not all symmetric monoidal categories possess the appropriate structure to interpret such reflexive pronouns, but we several options exist, explored in Figure \ref{fig:reflcomp}.

\begin{figure}[h!]\label{fig:nounconnection}
\centering
\[
\resizebox{0.75\textwidth}{!}{\tikzfig{mushroom/circuitplan}}
\]
\caption{We choose the convention of connecting from left-to-right and from bottom-to-top, so that we might read circuits as we would English text: the components corresponding to words will be arranged in the reverse order, left-to-right and top-to-bottom.}
\end{figure}

\begin{term}[Kinds of nouns with respect to coreference]\label{term:nounkinds}
\[
\resizebox{0.75\textwidth}{!}{\tikzfig{mushroom/nounkinds}}
\]
We classify kinds of nouns by their tails. \emph{Lonely} nouns have no coreferences, their tails connect to nothing. \emph{Head} nouns have a forward coreference in text; they have two tails, one that connects to nothing and the other to a noun later in text. \emph{Middle} nouns have a forward and backward coreference; they have two tails, one that connects to a noun in some preceding sentence, and one that connects forward to a noun in a succeeding sentence. \emph{Foot} nouns only have a backward coreference; they have a single tail connecting to a noun in some preceding sentence.
\end{term}

\begin{lemma}\label{prop:linkedlist}
The unsaturated nouns in Terminology \ref{term:nounkinds} are exhaustive, hence nouns that share a coreference are organised as a diagrammatic linked-list.
\begin{proof}
The \texttt{N}-intro rule creates lonely nouns. Head nouns can only be created by the linked-\texttt{N}-intro applied to a lonely noun. Any new noun created by linked-\texttt{N}-intro is a foot noun. The linked-\texttt{N}-intro rule turns foot nouns into middle nouns. These two intro- rules are the only ones that introduce unsaturated nouns, so it remains to demonstrate that no other rules can introduce noun-kinds that fall outside our taxonomy. The \texttt{N}-shift rule changes relative position of either a lonely or foot noun but cannot change its kind. The \texttt{N}-swap rule may start with either a lonely or foot noun on the left and either a head or middle noun on the right, but the outcome of the rule cannot change the starting kinds as tail-arity is conserved and the local nature of rewrites cannot affect the ends of tails.
\end{proof}
\end{lemma}

\begin{lemma}\label{prop:norefl}
No nouns within the same sentence are coreferentially linked.
\begin{proof}
Novel linked nouns can only be obtained from the linked-\texttt{N}-intro rule, which places them in succeeding sentences. The swap rules only operate within the same sentence and keep the claim invariant. The \texttt{N}-shift rules only apply to nouns with no forward coreferences; nouns with both forward and backward coreferences cannot leave the sentence they are in. Moreover, \texttt{N}-shift is unidirectional and only allows the rightmost coreference in a linked-list structure to move to later sentences. So there is no danger of an \texttt{N}-shift breaking the invariant.
\end{proof}
\end{lemma}

\begin{figure}[h!]\label{fig:reflcomp}
\centering
\[
\resizebox{0.75\textwidth}{!}{\tikzfig{mushroom/reflcomplication}}
\]
\caption{From left to right in roughly decreasing stringency, compact closed categories are the most direct solution for reflexive pronouns. Traced symmetric monoidal categories also suffice. So long as the noun wire possesses a monoid and comonoid, a convolution works. We also can just specify a new gate. We provide a purely syntactic treatment in \citep{wang-mascianicaDistillingTextCircuits2023a}; for now we treat them as if they were just verbs of lower arity.}
\end{figure}

\begin{defn}[Finished text diagram]\label{defn:finished}
The circuit-growing grammar produces \emph{text diagrams}. We call a text diagram \emph{finished} if all surface nodes are labelled.
\end{defn}

\begin{proposition}
Finished text diagrams yield text, up to interpreting distinct sentences as concatenated with punctuation \texttt{.}, \texttt{,}, contentless conjunctions or complementisers -- such as \texttt{and}, or \texttt{that} respectively.
\begin{proof}
Sentence-wise grammaticality is gauged by Propositions \ref{prop:simpsent} and \ref{prop:compsent}. When multiple sentences occur within the scope of a \texttt{SCV} we might prefer the use of contentless complementisers and conjunctions, e.g. \texttt{Alice sees \underline{that} Bob draws \underline{and} Charlie drinks \underline{, and} Dennis dances \underline{.}} is grammatically preferable but meaningfully equivalent to \texttt{Alice sees (Bob draws Charlie drinks) Dennis dances}. For our purposes it makes no difference whether surface text has these decorations, as the deep structure of text diagrams encodes all the information we care to know.
\end{proof}
\end{proposition}

\begin{proposition}[Finished text diagrams yield unique text circuits (up to processive isotopies)]\label{prop:text2circ}
\begin{proof}
Every sentence corresponds to a gate up to notation, and we have a handle on sentences via Propositions \ref{prop:simpsent} and \ref{prop:compsent}. Lemmas \ref{prop:linkedlist} and \ref{prop:norefl} guarantee processivity. Uniqueness-up-to-processive-isotopy is inherited: text diagrams themselves are already specified up to connectivity, which entails equivalence up to processive isotopy. Therefore, for any circuit $C$ obtained from a text diagram $T$ by Construction \ref{cons:wirejoin}, $T$ can be modified up to processive-isotopy on noun wires to yield $T'$ and another circuit $C'$ that only differs from $C$ up to processive isotopy, and all $C'$ may be obtained in this way.
\end{proof}
\end{proposition}

The converse of Proposition \ref{prop:text2circ} would be that any text circuit that can be formed by the composition of symmetric monoidal categories and of plugging gates into boxes yields a text diagram. This would mean that text circuit composition is acceptable as a generative grammar for text. Establishing this converse requires elaboration of some conventions.

\marginnote{
\begin{remark}
There are some oddities about our conventions that will make sense later when we consider semantics. For example, Convention \ref{conv:twist} an acceptable thing to ask for syntactically but quite odd to think about at the semantic level, where we would like to think that distinct nouns manifest as different states on the same noun-wire-type. A semantic interpretation that makes use of this convention will become clearer later in Sketch \ref{sec:config}. Similarly, Convention \ref{conv:exists} wouldn't be true if we consider the order of text to reflect the chronological ordering of events, in which case there are implicit \texttt{... and then ...} conjunctions that distinguish ordered gates from parallel gates conjoined by an implicit \texttt{... while ...}. This and the distinction in Convention \ref{conv:gaps} between typed and "untyped" higher-order processes will be given a suitable semantic interpretation in Sketch \ref{sec:turing}.
\end{remark}
}

\begin{convention}[Wire twisting]\label{conv:twist}
Wires are labelled by nouns. We consider two circuits the same if their gate-connectivity is the same. In particular, this means that we can eliminate unnecessary twists in wires to obtain diagrammatically simpler representations (Figure \ref{fig:twistsimple}).
\end{convention}

\begin{figure}[h!]\label{fig:twistsimple}
\centering
\[
\resizebox{0.5\textwidth}{!}{\tikzfig{textcirc/gatecompextgt}}
\]
\caption{
Only connectivity matters in text circuits, which we may use to freely rearrange and simplify presentations.}
\end{figure}

\begin{convention}[Arbitary vs. fixed holes]\label{conv:gaps}
Diagrammatically, adverbs and adpositions are depicted with no gap between the bounding box and their contents, whereas conjunctions and verbs with sentential complement are depicted with a gap; this is a visual indication that the former are type-sensitive, and the latter can take any circuit.
\end{convention}

\begin{convention}[Sliding]\label{conv:sliding}
Since only gate-connectivity matters, we consider circuits the same if all that differs is the horizontal positioning of gates composed in parallel (Figure \ref{fig:sliding}).
\end{convention}

\begin{figure}[h!]\label{fig:sliding}
\centering
\[
\resizebox{0.5\textwidth}{!}{\tikzfig{textcirc/gateeqslide}} 
\]
\caption{While sequential composition in process theories often has implicit temporality, this is not necessarily the case for text circuits, which may just (for instance) represent relational constraints. Temporality may be achieved in text circuits by interpreting them in premonoidal settings \citep{jeffreyPremonoidalCategoriesFlow1998}, at the cost of the interchange rule depicted here.}
\end{figure}

\vspace{1cm}

\begin{convention}[Reading text circuits]\label{conv:reading}
Pronominal connection conventions are to be chosen so that text circuits may be read in the same directional reading conventions of the language they aim to represent.
\end{convention}

\begin{convention}[Contentless conjunctions]\label{conv:and}
Conventions \ref{conv:sliding} and \ref{conv:reading} require something else to allow them to work at the same time: something to distinguish when the gates are parallel. In terms of text diagrams, we want rewrites that introduces such contentless conjunctions and witnesses their associativity, as in Figure \ref{fig:contentlessCNJ}:
\end{convention}

\begin{figure}[h!]\label{fig:contentlessCNJ}
\centering
\[
\tikzfig{mushroom/contentlesscnj}
\]
\caption{
Parallel gates represent compound sentences with contentless conjunctions. In English, some examples might be a punctuation mark such as a comma, or phrases such as \texttt{and also}.
}
\end{figure}

\begin{convention}[Lonely wires]\label{conv:exists}
There's only a single kind of text circuit we can draw that does not obviously correspond to a text diagram, and that's one where gates are missing (left). In process theories, wires are identity processes that do nothing to their inputs. So to deal with lonely wires in terms of text diagrams, we want a rewrite that introduces such contentless verbs (Figure \ref{fig:exists}).
\end{convention}

\begin{figure}[h!]\label{fig:exists}
\centering
\[
\tikzfig{textcirc/lonelywires} 
\qquad\qquad\qquad
\tikzfig{mushroom/exists} 
\]
\caption{
Lonely wires in text circuits are identity processes. We require a text diagram analogue, and an intransitive "null-verb" in English that seems to work is \texttt{is}, in the sense of \texttt{exists}.
}
\end{figure}

\clearpage

\begin{myboxB}
\begin{construction}[Circuit to text]\label{cons:circ2text}
In the presence of additional rewrites from Conventions \ref{conv:and} and \ref{conv:exists}, every text circuit is obtainable from some text diagram, up to Conventions \ref{conv:twist} and \ref{conv:sliding}. Starting with a circuit, we may use Convention \ref{conv:twist} to arrange the circuit into alternating slices of twisting wires and (possibly tensored) circuits, and this arrangement recurses within boxes. Slices with multiple tensored gates will be treated using Convention \ref{conv:and}. By convention \ref{conv:exists}, we decorate lonely wires with formal \texttt{exists} gates, as in the \texttt{Frank sees} box. Observe how verbs with sentential complement are depicted with grey gaps, whereas the adverb and adposition combination of \texttt{Mac crazily laughs at Cricket} is gapless, according to Convention \ref{conv:gaps}.
\[
\resizebox{0.5\textwidth}{!}{\tikzfig{textcirc/sunny1}}
\]
\end{construction}
\end{myboxB}

\begin{myboxB}
We then linearise the slices, representing top-to-bottom composition as left-to-right. Twist layers are eliminated, replaced instead by dotted connections indicating processive connectivity. The dashed vertical line distinguishes slices. This step of the procedure always behaves well, guaranteed by Proposition \ref{prop:linkedlist}. Noun wires that do not participate in earlier slices can be shifted right until the slice they are introduced.
\[
\resizebox{0.75\textwidth}{!}{\tikzfig{textcirc/sunny2}}
\]
We recurse the linearisation procedure within boxes until there are no more sequentially composed gates. The linearisation procedure evidently terminates for finite text circuits. At this point, we have abstracted away connectivity data, and we are left with individual gates.
\[
\resizebox{0.75\textwidth}{!}{\tikzfig{textcirc/sunny3}}
\]

\end{myboxB}

\clearpage

\begin{myboxB}
By Proposition \ref{prop:compsent}, gates are equivalent to sentences up to notation, so we swap notations \emph{in situ}. Conventions \ref{conv:and} and \ref{conv:exists} handle the edge cases of parallel gates and lonely wires. Observe that the blue-dotted wiring in text diagrams delineates the contents of boxes that accept sentences.
\[
\resizebox{0.9\textwidth}{!}{\tikzfig{textcirc/sunny4}}
\]
Recursing notation swaps outwards and connecting left-to-right slices as sentence-bubbles connect yields a text circuit, up to the inclusion of rewrites from Conventions \ref{conv:and} and \ref{conv:exists}: applying the reverse of those rewrites and the reverse of text-diagram rewrites yields a valid text-diagram derivation, by Propositions \ref{prop:compsent} and \ref{prop:linkedlist}. We haven't formally included transitive verbs with sentential complement in our vocabulary, but it should be obvious at this point how they function with our existing machinery.
\[
\resizebox{0.9\textwidth}{!}{\tikzfig{textcirc/sunny5}}
\]
\end{myboxB}

\newpage
\clearpage

\begin{myboxB}
\begin{theorem}[Text Circuit Theorem]
Text generated by the circuit-growing grammar is sensible and surjects onto text circuits. Hence:
\[\textbf{Text circuits are a generative grammar for text}\]
\begin{proof}
Sensibility at the sentential level is established by Propositions \ref{prop:simpsent} and \ref{prop:compsent}. Proposition \ref{prop:text2circ} establishes a map from text to circuits, and Construction \ref{cons:circ2text} witnesses its surjectivity. The conventions required for the construction are accompanied by justifications of sensibility.
\end{proof}
\end{theorem}
\end{myboxB}

\clearpage
\newpage
