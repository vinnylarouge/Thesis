\newpage

\section{A generative grammar for text circuits}

\subsection{A circuit-growing grammar}

\marginnote{
\begin{defn}[Lexicon]
We define a lexicon $\mathcal{L}$ to be a tuple $(\mathbf{N}, \mathbf{V}_1, \mathbf{V}_2, \mathbf{V}_{\texttt{S}}, \mathbf{A}_{\texttt{N}}, \mathbf{A}_{\texttt{V}}, \mathbf{C})$
\end{defn}
}

\marginnote{
Where:
\begin{itemize}
\item $\mathbf{N}$ is a set of \emph{proper nouns}
\item $\mathbf{V}_1$ is a set of \emph{intransitive verbs}
\item $\mathbf{V}_2$ is a set of \emph{transitive verbs}
\item $\mathbf{V}_{\texttt{S}}$ is a set of \emph{sentential-complement verbs}
\item $\mathbf{A}_{\texttt{N}}$ is a set of \emph{adjectives}
\item $\mathbf{A}_{\texttt{V}}$ is a set of \emph{adverbs}
\item $\mathbf{C}$ is a set of \emph{conjunctions}
\end{itemize}
}

\begin{marginfigure}
\centering
\[
\resizebox{\textwidth}{!}{\tikzfig{mushroom/howtoread}}
\]
\caption{\textbf{How to read the diagrams in this section:} we will be making heavy use of pink and purple bubbles as frames to construct circuits. We will depict the bubbles horizontally, as we are permitted to by compact closure, or by reading diagrams with slightly skewed axes.}
\end{marginfigure}

There are many different ways to write an $n$-categorical signature that generates circuits. Mostly as an illustration of expressive capacity, I will provide a signature where the terms "surface" and "deep" structure are taken literally as metaphors; the generative grammar will grow a line of words in (roughly) syntactic order, and like mushrooms on soil, the circuits will appear as the mycelium underneath words.\\

\newthought{Simplifications}: Propositions only, no determiners, only one tense, no morphological agreement between nouns and their verbs and referring pronouns, and we assume that adverbs, adjectives stack indefinitely and without further order requirements; e.g. \texttt{Alice happily secretly finds red big toy shiny car that he gives to Bob} is a sentence we consider grammatical enough. For now, we consider only the case where adjectives and adverbs appear before their respective noun or verb. Note that all of these limitations can principle be overcome by the techniques we developed in Section \ref{sec:ncat} for restricted tree-adjoining and links.

\newthought{Mathematical assumptions}: We work in a dimension where wires behave symmetric monoidally by homotopy, and further assume strong compact closure rewrite rules for all wire-types. Our strategy is to generate "bubbles" for sentences, within which we can grow circuit structure piecemeal. We will only express the rewrite rules; the generators of lower dimension are implicit. We aim to recover the linear ordering of words in text (essential to any syntax) by traversing the top surface of a chain of bubbles representing sentence structure in text -- this order will be invariant despite compact closure. The diagrammatic consequence of these assumptions is that we will be working with a conservative generalisation of graph-rewriting defined by local rewriting rules. The major distinction is that locality can be redefined up to homotopy, which allows locally-defined rules to operate in what would be a nonlocal fashion in terms of graph neighbourhoods, as in Figure \ref{fig:locality}.

\begin{figure}[h!]\label{fig:locality}
\centering
\[
\resizebox{\textwidth}{!}{\tikzfig{mushroom/locality}}
\]
\caption{In this toy example, obtaining the same rewrite that connects the two yellow nodes using graph-theoretically local rules could potentially require an infinite family of rules, for all possible configurations of pink and cyan nodes that separate the yellow. In our setting strong compact closure handles navigation between different spatial presentations, so that a single rewrite rule suffices. Despite the gain in expressive power, we cannot "computationally cheat" graph isomorphism: formally we must supply the compact-closure homotopies as part of the rewrite, notated by $\simeq$.}
\end{figure}

The minor distinction is that rewrite rules are sensitive to twists in wires and the radial order in which wires emanate from nodes, though it is easy to see how these distinctions can be circumvented by additional by imposing the equivalent of commutativity relations as bidirectional rewrites.

\newthought{Communication strategy}: We start with simple sentences that only contain a single intransitive or transitive verb. Then we consider more general sentences. For these two steps, we characterise the expressive capacity of our rules in terms of a context-sensitive grammar that corresponds to the surface structure of the derivations. Then we introduce text structure as lists of sentences with coreferential structure on nouns, along with a mathematical characterisation of coreferential structure and a completeness result of our rules with respect to them. Then we (re)state and prove the text circuit theorem: that the fragment of language we have built with the syntax surjects onto text circuits. Finally we examine how we may model extensions to the expressive capacity of text circuits by introduction of new rewrite rules.

\clearpage

\subsection{Simple sentences}

\marginnote{
\begin{defn}[CSG for simple sentences]\label{dfn:simpCSG}
We may gauge the expressivity of simple sentences with the following context sensitive grammar.
\end{defn}
}

\marginnote{For verbs, adjectives, and modifiers, depicted unsaturated nouns as dotted and saturated with solid black lines, we have:
\[
\resizebox{\marginparwidth}{!}{\tikzfig{mushroom/simpleCFG}}
\]
}

\marginnote{
Adpositions require several helper-generators; we depict for example the beginning of the sequence of derivations that result from appending adpositions to an intransitive verb (the generators are implicit in the derivations):
\[
\resizebox{0.75\marginparwidth}{!}{\tikzfig{mushroom/simpleADP}}
\]
}

\marginnote{
\begin{proposition}\label{prop:simpsent}
Up to labels, the simple-sentence rules yield the same simple sentences as the CSG for simple sentences.
\begin{proof}
By graphical correspondence; viewing nodes on the pink surface as 1-cells, each rewrite rule yields a 2-cell. For example, for the \textcolor{green}{\texttt{IV}}-intro:
\[
\resizebox{0.5\marginparwidth}{!}{\tikzfig{mushroom/simpcorr}}
\]
\end{proof}
\end{proposition}
}

Simple sentences are sentences that only contain a single intransitive or transitive verb. Simple sentences will contain at least one noun, and may optionally contain adjectives, adverbs, and adpositions. The rules for generating simple sentences are as follows:

\[
\resizebox{\textwidth}{!}{\tikzfig{mushroom/simplesentences}}
\]

The $\texttt{N}_\uparrow$-intro rule introduces new unsaturated nouns from the end of a simple sentence. The \textcolor{green}{\texttt{IV}}-intro rule applies when there is precisely one unsaturated noun in the sentence, and the \textcolor{green}{\texttt{TV}}-intro rule applies when there are precisely two. Both verb-introduction rules saturate their respective nouns, which we depict with a black bulb. Adjectives may be introduced immediately preceding saturated nouns, and adverbs may be introduced immediately preceding any kind of verb. The position of adpositions in English is context-sensitive. To capture this, the $\textcolor{blue}{\texttt{ADP}}_{\textcolor{green}{\texttt{V}}}$-tendril rule allows an unsaturated adposition to appear immediately after a verb; a bulb may travel by homotopy to the right, seeking an unsaturated noun. Conversely, the bidirectional $\textcolor{blue}{\texttt{ADP}}_{\texttt{N}}$-tendril rule sends a mycelic tendril to the left, seeking a verb. The two pass-rules allow unsaturated adpositions to swap past saturated nouns and adjectives; note that by construction, neither verbs nor adverbs will appear in a simple sentence to the right of a verb, so unsaturated adpositions will move right until encountering an unsaturated noun. In case it doesn't, the tendril- and pass- rules are bidirectional and reversible.

\clearpage

\subsection{Complex sentences}

Now we consider two refinements; conjunctions, and verbs that take sentential complements. we may have two sentences joined by a conjunction, e.g. \texttt{Alice dances \underline{while} Bob drinks}. In the same vein, the unary case of a conjunction of simple sentences is an adverbial adposition, e.g. \texttt{\underline{Yesterday} Alice dance(d)}. Finally, we may have verbs that take a sentential complement rather than a noun phrase, e.g. \texttt{Alice \underline{sees} Bob dance}; these verbs require nouns, which we depict as wires spanning bubbles.

\begin{figure}[h!]\label{fig:sentbestiary}
\centering
\[
\resizebox{\textwidth}{!}{\tikzfig{mushroom/Sbestiary}}
\]
\caption{The dotted-blue wires do not contentfully interact with anything else; they will serve as visual aids for circuit-translation later, to indicate the contents of boxes. They do however indicate a diagrammatic strategy for extensions to accommodate noun phrases, to be explored later.}
\end{figure}

\marginnote{
\begin{defn}[Sentence structure]\label{dfn:sentCSG}
A sentence can be:
\begin{itemize}
\item a simple sentence, which...
\item ... may generate unsaturated nouns from the right.
\item a pair of sentences with a conjunction in between.
\item (if there is a single unsaturated noun) a sentence with a sentential-complement verb that scopes over a sentence.
\end{itemize}
As a CSG, these considerations are respectively depicted as:
\end{defn}
\[
\resizebox{\marginparwidth}{!}{\tikzfig{mushroom/complexsentenceCSG}}
\]
}

\marginnote{
\begin{proposition}\label{prop:compsent}
Up to labels, the rules so far yield the same sentences as the combined CSG of Definitions \ref{dfn:simpCSG} and \ref{dfn:sentCSG}.
\begin{proof}
Same correspondence as Proposition \ref{prop:simpsent}, ignoring the dotted-blue guards.
\end{proof}
\end{proposition}
}

\begin{figure}[h!]\label{fig:soberA}
\centering
\[
\resizebox{\textwidth}{!}{\tikzfig{mushroom/soberA}}
\]
\caption{
\begin{example}[\texttt{sober} $\alpha$ \texttt{sees drunk} $\beta$ \texttt{clumsily dance.}]
Now we can see our rewrites in action for sentences. As a matter of convention -- reflected in how the various pass- rules do not interact with labels -- we assume that labelling occurs after all of the words are saturated. We have still not introduced rules for labelling nouns: we delay their consideration until we have settled coreferential structure. For now they are labelled informally with greeks.
\end{example}
}
\end{figure}

\begin{figure}[h!]\label{fig:Alaughs}
\centering
\[
\resizebox{\textwidth}{!}{\tikzfig{mushroom/Alaughs}}
\]
\caption{
\begin{example}[$\alpha$ \texttt{laughs at} $\beta$]
Adpositions form by first sprouting and connecting tendrils under the surface. Because the tendril- and pass- rules are bidirectional, extraneous tendrils can always be retracted, and failed attempts for verbs to find an adpositional unsaturated noun argument can be undone. Though this seems computationally wasteful, it is commonplace in generative grammars to have the grammar overgenerate and later define the set of sentences by restriction, which is reasonable so long as computing the restriction is not computationally hard. In our case, observe that once a verb has been introduced and its argument nouns have been saturated, only the introduction of adpositions can saturate additionally introduced unsaturated nouns. Therefore we may define the finished sentences of the circuit-growing grammar to be those that e.g. contain no unsaturated nodes on the surface, which is a very plausible linear-time check by traversing the surface.
\end{example}
}
\end{figure}

\clearpage

\subsection{Text structure and noun-coreference}

\begin{figure}[h!]
\centering
\[
\tikzfig{mushroom/sintro}
\]
\caption{Only considering words, text is just a list of sentences. However, for our purposes, text additionally has \emph{coreferential structure}. Ideally, we would like to connect "the same noun" from distinct sentences as we would circuits.}
\end{figure}

\begin{figure}[h!]
\centering
\[
\tikzfig{mushroom/circuitplan}
\]
\caption{We choose the convention of connecting from left-to-right and from bottom-to-top, so that we might read circuits as we would text: the components corresponding to words will be arranged left-to-right and top-to-bottom. Connecting nouns across distinct sentences presents no issue, but a complication arises when connecting nouns within the same sentence as with reflexive pronouns e.g. \texttt{Alice likes herself}.}
\end{figure}

\begin{figure}[h!]\label{fig:reflcomp}
\centering
\[
\tikzfig{mushroom/reflcomplication}
\]
\caption{This would violate of the processivity condition of string diagrams for symmetric monoidal categories. Not all symmetric monoidal categories possess the appropriate structure to interpret such reflexive pronouns, but there exist interpretative options. From left to right in roughly decreasing stringency, compact closed categories are the most direct solution. More weakly, traced symmetric monoidal categories also suffice. If there are no traces, so long as the noun wire possesses a monoid and comonoid, a convolution works. If all else fails, one can just specify a new gate. We will define coreference structure to exclude such reflexive coreference and revisit the issue as an extension.}
\end{figure}

\newpage

Now we will deal with coreferential structure and noun-labels. Coreferenced unsaturated nouns will possess their own introduction rule, along with pass- rules to allow them to shift further down text.

\[
\resizebox{\textwidth}{!}{\tikzfig{mushroom/nounbestiary_newnew}}
\]

\marginnote{
We require a definition of coreference structure in order verify that our rewrite rules are complete with respect to them. One reasonable constraint on coreference in text is that coreference always goes backwards. One way to express this constraint mathematically is the following:
\begin{defn}[Coreference structure]
In a text with $K \in \mathbb{N}$ distinct nouns and their coreferents, the linear order of noun+referents in a text corresponds to a list of positive integers that:
\begin{enumerate}
\item Starts with 1
\item Contains every $k \in [1\cdots K]$
\item For all $k \in [1 \cdots K]$, the head of the list up to the first occurence of $k$ contains all $j < k$.
\end{enumerate}
\end{defn}
}

\marginnote{
\begin{proposition}
The noun-introduction and manipulation rules are expressively complete with respect to coreference structure.
\begin{proof}

\end{proof}
\end{proposition}
}

\newpage

Nouns require verbs in order to be saturated. From left-to-right; if there is precisely one unlabelled noun, we may introduce an unlabelled intransitive verb and saturate the noun so that it is now ready to grow a label; or if there are two unlabelled nouns, we may introduce an unlabelled transitive verb on the surface and saturate the two nouns that will be subject and object; and verbs may be labelled.
\[
\tikzfig{mushroom/simpbestiary}
\]

\subsection{Modifiers}

Modifiers are optional parts of sentences that modify (and hence depend on there being) nouns and verbs. We consider adjectives, adverbs, and adpositions. From left to right; we allow adjectives to sprout immediately before a saturated noun, and we allow adverbs to sprout immediately before any verb.

\[
\tikzfig{mushroom/adjadv}
\]

Adpositions modify verbs by tying in an additional noun argument; e.g. while \texttt{runs} is intransitive, \texttt{runs towards} behaves as a transitive verb. Some more advanced technology is required to place adpositions and their thematic nouns in the correct linear order on the surface. In the left column; an adposition tendril can sprout from a verb via an unsaturated adposition, seeking an unsaturated noun to the right; an unsaturated noun can sprout an tendril seeking a verb to connect to on the left. Both of these rewrites are bidirectional, as tendrils might attempt connection but fail, and so be retracted. In the centre, when an unsaturated adposition and its tendril find an unsaturated noun, they may connect, saturating the adposition so that it is ready to label. In the right column; an unsaturated adposition may move past a saturated noun in the same sentence, which allows multiple adpositions for the same verb; finally, a saturated adposition can be labelled.

\[
\tikzfig{mushroom/adpbestiary}
\]

\subsection{Rewriting to circuit-form}

\newthought{Resolving references}

\[
\tikzfig{mushroom/pronbestiary}
\]

\[
\tikzfig{mushroom/pronres}
\]

\newthought{Connecting circuits}

\newpage
\subsection{Putting it all together}

\begin{figure}[h!]
\centering
\[
\resizebox{\textwidth}{!}{\tikzfig{mushroom/bigex1}}
\]
\caption{Starting from the initial sentence bubble, we generate a new sentence, introduce some nouns and an SCV, and then we connect our references at the bottom.}
\end{figure}

\begin{figure}[h!]
\centering
\[
\resizebox{\textwidth}{!}{\tikzfig{mushroom/bigex2}}
\]
\caption{Then we introduce intransitive verbs to saturate nouns, and we may also sprout some modifying adjectives and adverbs.}
\end{figure}

\begin{figure}[h!]
\centering
\[
\resizebox{\textwidth}{!}{\tikzfig{mushroom/bigex3}}
\]
\caption{For the remaining unsaturated noun, we use the adposition introduction rules to sprout tendrils off of the other verb in the bubble, and connect.}
\end{figure}

\begin{figure}[h!]
\centering
\[
\resizebox{\textwidth}{!}{\tikzfig{mushroom/bigex4}}
\]
\caption{All of the non-noun words may now be labelled.}
\end{figure}

\begin{figure}[h!]
\centering
\[
\resizebox{\textwidth}{!}{\tikzfig{mushroom/bigex5}}
\]
\caption{To begin assigning nouns, observe that by compact closure of the bubble boundaries, we can deform the diagram to obtain suitable forms for our local rewrite rules for link-generation at the bottom.}
\end{figure}

\begin{figure}[h!]
\centering
\[
\resizebox{\textwidth}{!}{\tikzfig{mushroom/bigex6}}
\]
\caption{Now we can introduce our noun labels and linearise our link structure.}
\end{figure}

\begin{figure}[h!]
\centering
\[
\resizebox{\textwidth}{!}{\tikzfig{mushroom/bigex7}}
\]
\caption{Once the link structure is linearised, we can undo the deformation, and propagate links to the surface.}
\end{figure}

\begin{figure}[h!]
\centering
\[
\resizebox{\textwidth}{!}{\tikzfig{mushroom/bigex8}}
\]
\caption{The bubbles may be rearranged to respect circuit-form, such that the propagation of noun labels to the surface traces out the wires of the end circuit.}
\end{figure}

\clearpage
\newpage
\subsection{Extensions I: relative and reflexive pronouns}

\newthought{Subject relative pronouns}

\begin{example}

\end{example}

\newthought{Object relative pronouns}

\begin{example}

\end{example}

\newthought{Reflexive pronouns}

\begin{example}

\end{example}

\subsection{Extensions II: grammar equations}

\newthought{Attributive vs. predicative modifiers}

\begin{example}

\end{example}

\newthought{Copulas}

\begin{example}

\end{example}

\newthought{Possessive pronouns}

\begin{example}

\end{example}

\subsection{Extensions III: higher-order modifiers}

\newthought{Intensifiers}

\begin{example}

\end{example}

\newthought{Comparatives}

\begin{example}

\end{example}

\subsection{Equivalence to internal wirings}

\subsection{Text circuit theorem}

\subsection{Related work}