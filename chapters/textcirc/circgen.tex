\section{A generative grammar for text circuits}

\subsection{A circuit-growing grammar}

There are many different ways to write an $n$-categorical signature that generates circuits. Mostly as an illustration of expressive capacity, I will provide a signature where the terms "surface" and "deep" structure are taken literally as metaphors; the generative grammar will grow a line of words in (roughly) syntactic order, and like mushrooms on soil, the circuits will appear as the mycelium underneath words.\\

\newthought{Simplifications}: Propositions only, no determiners, only one tense, no morphological agreement between nouns and their verbs and referring pronouns, and we assume that adverbs, adverbial adjunctions, and adjectives stack indefinitely and without further order requirements; e.g. \texttt{Yesterday yesterday yesterday Alice happily secretly finds red big toy shiny car that he gives to Bob.} we consider grammatical enough. For now, we consider only the case where adjectives and adverbs appear before their respective noun or verb. Note that all of these limitations can principle be overcome by the techniques we developed in Section \ref{sec:ncat} for restricted tree-adjoining and links.

\newthought{How to read the following diagrams}: We work in a dimension where wires behave symmetric monoidally by homotopy, and the signature still works if interpreted in a compact closed setting. We start each derivation with a pink sentence bubble, which we depict for aesthetic purposes as horizontal, which amounts to picking slightly different axes by which to read diagrams. The insides of the sentence bubble will fill up with a circuit, and labelled words will appear on the top surface bubble like stylised mushrooms, to be read off left-to-right. We will only express the rewrite rules; the generators of lower dimension are implicit.

\[
\tikzfig{mushroom/howtoread}
\]

\subsection{Sentences}

Before the contents of a sentence are even decided, we may decide to (top row, left to right) get another sentence ready, introduce an adverbial adjunction (such as \texttt{yesterday}), or introduce a conjunction of two sentences (such as \texttt{because}). Introducing new words yields dots that carry information of the grammatical category of the word, and there are separate rewrites that allow dots to be labelled according to the lexicon. In the bottom, we have a rewrite that allows a sentence with one unlabelled noun to be the subject of a "sentential complement verb", abbreviated \texttt{SCV}: these are verbs that take sentences as objects rather than other nouns, and they are typically verbs of cognition, perception, and causation, such as \texttt{Alice \underline{suspects} Bob drinks}. The blue-dotted lines are just syntactic guardrails that correspond to the holes in boxes of circuits later.

\[
\tikzfig{mushroom/bestiary2}
\]

\subsection{Simple sentences}
Within each sentence bubble, each derivation starts life as a "simple sentence", which only involves nouns and a single verb, which is either an intransitive verb that takes a single noun argument, or a transitive verb that takes two. You just can't have a (propositional) sentence without at least a noun and a verb. Within each sentence, we may start introducing nouns. From left-to-right; we may introduce a new noun (which comes with tendrils that extrude outside the sentence bubble for later use to resolve pronominal reference); split a noun so that the same noun-label is used multiply; and label a noun \emph{if it is saturated -- depicted by a solid black circle} which includes a copy of the label for bookkeeping purposes when resolving references.
\[
\tikzfig{mushroom/nounbestiary}
\]
Nouns require verbs in order to be saturated. From left-to-right; if there is precisely one unlabelled noun, we may introduce an unlabelled intransitive verb and saturate the noun so that it is now ready to grow a label; or if there are two unlabelled nouns, we may introduce an unlabelled transitive verb on the surface and saturate the two nouns that will be subject and object; and verbs may be labelled.
\[
\tikzfig{mushroom/simpbestiary}
\]

\subsection{Modifiers}

Modifiers are optional parts of sentences that modify (and hence depend on there being) nouns and verbs. We consider adjectives, adverbs, and adpositions. From left to right; we allow adjectives to sprout immediately before a saturated noun, and we allow adverbs to sprout immediately before any verb.

\[
\tikzfig{mushroom/adjadv}
\]

Adpositions modify verbs by tying in an additional noun argument; e.g. while \texttt{runs} is intransitive, \texttt{runs towards} behaves as a transitive verb. Some more advanced technology is required to place adpositions and their thematic nouns in the correct linear order on the surface. In the left column; an adposition tendril can sprout from a verb via an unsaturated adposition, seeking an unsaturated noun to the right; an unsaturated noun can sprout an tendril seeking a verb to connect to on the left. Both of these rewrites are bidirectional, as tendrils might attempt connection but fail, and so be retracted. In the centre, when an unsaturated adposition and its tendril find an unsaturated noun, they may connect, saturating the adposition so that it is ready to label. In the right column; an unsaturated adposition may move past a saturated noun in the same sentence, which allows multiple adpositions for the same verb; finally, a saturated adposition can be labelled.

\[
\tikzfig{mushroom/adpbestiary}
\]

\subsection{Rewriting to circuit-form}

\newthought{Resolving references}

\[
\tikzfig{mushroom/pronbestiary}
\]

\[
\tikzfig{mushroom/pronres}
\]

\newthought{Connecting circuits}

\newpage
\subsection{Putting it all together}

\begin{figure}[h!]
\centering
\[
\resizebox{\textwidth}{!}{\tikzfig{mushroom/bigex1}}
\]
\caption{Starting from the initial sentence bubble, we generate a new sentence, introduce some nouns and an SCV, and then we connect our references at the bottom.}
\end{figure}

\begin{figure}[h!]
\centering
\[
\resizebox{\textwidth}{!}{\tikzfig{mushroom/bigex2}}
\]
\caption{Then we introduce intransitive verbs to saturate nouns, and we may also sprout some modifying adjectives and adverbs.}
\end{figure}

\begin{figure}[h!]
\centering
\[
\resizebox{\textwidth}{!}{\tikzfig{mushroom/bigex3}}
\]
\caption{For the remaining unsaturated noun, we use the adposition introduction rules to sprout tendrils off of the other verb in the bubble, and connect.}
\end{figure}

\begin{figure}[h!]
\centering
\[
\resizebox{\textwidth}{!}{\tikzfig{mushroom/bigex4}}
\]
\caption{All of the non-noun words may now be labelled.}
\end{figure}

\begin{figure}[h!]
\centering
\[
\resizebox{\textwidth}{!}{\tikzfig{mushroom/bigex5}}
\]
\caption{To begin assigning nouns, observe that by compact closure of the bubble boundaries, we can deform the diagram to obtain suitable forms for our local rewrite rules for link-generation at the bottom.}
\end{figure}

\begin{figure}[h!]
\centering
\[
\resizebox{\textwidth}{!}{\tikzfig{mushroom/bigex6}}
\]
\caption{Now we can introduce our noun labels and linearise our link structure.}
\end{figure}

\begin{figure}[h!]
\centering
\[
\resizebox{\textwidth}{!}{\tikzfig{mushroom/bigex7}}
\]
\caption{Once the link structure is linearised, we can undo the deformation, and propagate links to the surface.}
\end{figure}

\begin{figure}[h!]
\centering
\[
\resizebox{\textwidth}{!}{\tikzfig{mushroom/bigex8}}
\]
\caption{The bubbles may be rearranged to respect circuit-form, such that the propagation of noun labels to the surface traces out the wires of the end circuit.}
\end{figure}

\clearpage
\newpage
\subsection{Extensions I: relative and reflexive pronouns}

\newthought{Subject relative pronouns}

\begin{example}

\end{example}

\newthought{Object relative pronouns}

\begin{example}

\end{example}

\newthought{Reflexive pronouns}

\begin{example}

\end{example}

\subsection{Extensions II: grammar equations}

\newthought{Attributive vs. predicative modifiers}

\begin{example}

\end{example}

\newthought{Copulas}

\begin{example}

\end{example}

\newthought{Possessive pronouns}

\begin{example}

\end{example}

\subsection{Extensions III: higher-order modifiers}

\newthought{Intensifiers}

\begin{example}

\end{example}

\newthought{Comparatives}

\begin{example}

\end{example}

\subsection{Equivalence to internal wirings}

\subsection{Text circuit theorem}

\subsection{Related work}