\section{An everyday miracle of language}

\newthought{Speakers and listeners understand one another.} Even if we cautiously hedge this mundane statement to exclude pragmatics and context and only encompass small and boring fragments of factual language, as far as formal theories of syntax and semantics go, this is still a miracle. In this section I will first explain why this simple fact represents a gap in our understanding of language. Hopefully the urgency of filling the gap will be impressed, and then I will provide a mathematical structure that bridges this gap.

\subsection{Speaking grammars, listening grammars}

Here are some na\"{i}ve observations on the nature of speaking and listening. Let's suppose that a speaker, Preube, wants to communicate a thought to Fondo. Just as a running example that does not affect the point, let's say we can gloss the thought as carrying the relations:\\

(alice, bob, flowers, like, give)\\

Preube and Fondo cooperate to achieve a minor miracle; Preube encodes his thoughts -- a structure that doesn't look like a one-dimensional string of symbols -- into a one-dimensional string of symbols, and Fondo does the reverse, turning a one-dimensional string of symbols into \emph{a thought-structure like that of Preube's.}\marginnote{What I mean by this is just that both Preube and Fondo agree on the structure of entities and relations up to the words for those entities and relations. For example, Preube could ask Fondo comprehension questions such as \texttt{WHO GAVE WHAT? TO WHO?}, and if Fondo can always correctly answer -- e.g. \texttt{BOB GAVE FLOWERS. TO CLAIRE.} -- then both Preube and Fondo agree on the relational structure of the communicated thought to the extent permitted by language. It may still be that Preube and Fondo have radically different internal conceptions of what \texttt{FLOWERS} or \texttt{GIVING} or \texttt{BEETLES IN BOXES} are, but that is alright: we only care that the \emph{interacting structure} of the thought-relations are the same, not their specific representations.} The two can do this for infinitely many thoughts, and new thoughts neither have encountered before.\\

\[placeholder\]\\



We assume Preube and Fondo speak the same language, so both know how words in their language correspond to putative building blocks of thoughts, and how the order of words in sentences and special grammatical words affect the (de/re)construction procedures.

\[placeholder\]\\

The nature of the problem can be summarised as an asymmetry of information. The speaker knows the structure of a thought and has to make choices to turn that thought into text. The listener knows only the text, and must make choices to deduce the thought behind it. By this perspective, language is a shared and cooperative strategy to solve this (de/en)coding interaction. I will now outline the constraints imposed by this interaction, and then explain how context-free grammars and typelogical grammars partially model these constraints.

\newthought{Speakers choose.} The speaker Preube must supply decisions about phrasing a thought in the process of speaking it. At some point at the beginning of an utterance, Preube has a thought but has not yet decided how to say it, and finding a particular phrasing requires choices to be made, because there are many ways to express even simple relational thoughts. For example, the relational content of our running example might be expressed as:

\[\texttt{\underline{ALICE LIKES THE FLOWERS BOB GIVES CLAIRE.}}\]
\[\texttt{\underline{BOB GIVES CLAIRE FLOWERS. ALICE LIKES THE FLOWERS.}}\]
\[\texttt{\underline{THE FLOWERS TO CLAIRE THAT BOB GIVES ARE LIKED BY ALICE.}}\]

Whether those decisions are made by committee or coinflips, those decisions represent information that must be supplied to Preube in the process of speaking a thought.

\[placeholder\]\\

For this reason, we consider context-free-grammars (and more generally, other string-rewrite systems) to be \emph{Grammars of the speaker}. The start symbol $S$ is incrementally expanded and determined by rule-applications that are selected by the speaker. The important aspect here is that the speaker has an initial state of information $S$ that requires more information as input in order to arrive at the final sentence.

\newthought{Listeners deduce.} The listener Fondo has a text, and must supply decisions about how syntactic...

