\section{How do we communicate?}

\newthought{Speakers and listeners understand one another.} As far as formal theories of syntax and semantics go, this is a miracle. It remains so even if we cautiously hedge this mundane statement to exclude pragmatics and context and only encompass small and boring fragments of factual language.\\

In short: there is a distinction between \emph{grammars of the speaker} -- which produce sentences -- and \emph{grammars of the listener} -- which deduce sentences. Viewed as mathematical processes, the two kinds of grammars go in opposite directions; speaking grammars [e.g. string-rewrite systems] start with some structure, and require informational input [e.g. which rule comes next] to produce sentences; listening grammars [e.g. typelogical grammars] start with a sentence, and require informational input [e.g. grammatical typing and which proof rule to try next] to deduce a grammatical structure. Since we can understand each other, these two types of grammar must enjoy a systematic correspondence. The correspondence looks something like this:\\

\[placeholder\]

In this section I will elaborate on the above argument, and provide the first steps of a formal mathematical framework to present and organise the nature of this correspondence. This framework comes from the idea that speakers and listeners may use language as a vehicle to communicate thought; from this formal framework, we detect that the intermediate structure -- that which is being communicated -- leads us to the idea of text circuits.\\

The empirical observation that speakers and listeners can communicate is a trivial one; any account of language that does not take this interactive and procedural aspect into account is arguably lacking. Text circuit theory aims to provide the beginnings of an theory of language that accounts for not just this...

\subsection{Speaking grammars, listening grammars}

Here are some na\"{i}ve observations on the nature of speaking and listening. Let's suppose that a speaker, Preube, wants to communicate a thought to Fondo. Just as a running example that does not affect the point, let's say we can gloss the thought as carrying the relations:\\

(alice, bob, flowers, like, give)\\

Preube and Fondo cooperate to achieve a minor miracle; Preube encodes his thoughts -- a structure that doesn't look like a one-dimensional string of symbols -- into a one-dimensional string of symbols, and Fondo does the reverse, turning a one-dimensional string of symbols into \emph{a thought-structure like that of Preube's.}\marginnote{What I mean by this is just that both Preube and Fondo agree on the structure of entities and relations up to the words for those entities and relations. For example, Preube could ask Fondo comprehension questions such as \texttt{WHO GAVE WHAT? TO WHO?}, and if Fondo can always correctly answer -- e.g. \texttt{BOB GAVE FLOWERS. TO CLAIRE.} -- then both Preube and Fondo agree on the relational structure of the communicated thought to the extent permitted by language. It may still be that Preube and Fondo have radically different internal conceptions of what \texttt{FLOWERS} or \texttt{GIVING} or \texttt{BEETLES IN BOXES} are, but that is alright: we only care that the \emph{interacting structure} of the thought-relations are the same, not their specific representations.} The two can do this for infinitely many thoughts, and new thoughts neither have encountered before.\\

\[placeholder\]\\

We assume Preube and Fondo speak the same language, so both know how words in their language correspond to putative building blocks of thoughts, and how the order of words in sentences and special grammatical words affect the (de/re)construction procedures.

\[placeholder\]\\

The nature of the problem can be summarised as an asymmetry of information. The speaker knows the structure of a thought and has to make choices to turn that thought into text. The listener knows only the text, and must make choices to deduce the thought behind it. By this perspective, language is a shared and cooperative strategy to solve this (de/en)coding interaction. I will now outline the constraints imposed by this interaction, and then explain how context-free grammars and typelogical grammars partially model these constraints.

\newthought{Speakers choose.} The speaker Preube must supply decisions about phrasing a thought in the process of speaking it. At some point at the beginning of an utterance, Preube has a thought but has not yet decided how to say it. Finding a particular phrasing requires choices to be made, because there are many ways to express even simple relational thoughts. For example, the relational content of our running example might be expressed as 

\[\texttt{\underline{ALICE LIKES THE FLOWERS BOB GIVES CLAIRE.}}\]
\[\texttt{\underline{BOB GIVES CLAIRE FLOWERS. ALICE LIKES THE FLOWERS.}}\]
\[\texttt{\underline{THE FLOWERS TO CLAIRE THAT BOB GIVES ARE LIKED BY ALICE.}}\]

Whether those decisions are made by committee or coinflips, those decisions represent information that must be supplied to Preube in the process of speaking a thought.

\[placeholder\]\\

For this reason, we consider context-free-grammars (and more generally, other string-rewrite systems) to be \emph{Grammars of the speaker}. The start symbol $S$ is incrementally expanded and determined by rule-applications that are selected by the speaker. The important aspect here is that the speaker has an initial state of information $S$ that requires more information as input in order to arrive at the final sentence.

\newthought{Listeners deduce.} The listener Fondo must supply decisions about which words are grammatically related, and how. Like right-of-way in driving, sometimes these decisions are settled by convention, for example, subject-verb-object order in English. Sometimes sophisticated decisions need to be made that override or are orthogonal to conventions, and we give two examples.

\marginnote[-15cm]{
It is worth noting that in practice, neither grammar nor meaning strictly determines the other. Clearly there are cases where grammar supercedes: when Fondo hears \texttt{man bites dog}, despite his prior prejudices and associations about which animal is more likely to be biting, he knows that the \texttt{man} is doing the biting and the \texttt{dog} is getting bitten. Going the other way, there are many cases in which the meaning of a subphrase affects grammatical acceptability and structure.

\begin{example}[Exclamations: how meaning affects grammar]
The following examples from [Lakofflecture] illustrate how whether a phrase is an \emph{exclamation} affects what kinds of grammatical constructions are acceptable. By this argument, to know whether something is an exclamation in context is an aspect of meaning, so we have cases where meaning determines grammar. Observe first that the following three phrases are all grammatically acceptable and mean the same thing.
\[\texttt{\textcolor{blue}{nobody knows} how many beers Bob drinks}\]
\[\texttt{\textcolor{blue}{who knows} how many beers Bob drinks}\]
\[\texttt{\textcolor{blue}{God knows} how many beers Bob drinks}\]
The latter two are distinguished when \texttt{God knows} and \texttt{who knows} are exclamations. First, the modularity of grammar and meaning may not match when an exclamation is involved. For example, negating the blue text, we obtain:
\[\texttt{\textcolor{blue}{somebody knows} how many beers Bob drinks}\]
\[\texttt{\textcolor{blue}{who doesn't know} how many beers Bob drinks}\]
\[\texttt{\textcolor{purple}{God doesn't} know how many beers Bob drinks}\]
The first two are acceptable, but mean different things; the latter means to say that everyone knows how many beers Bob drinks, which is stronger than the former. The last sentence is awkward: unlike in the first two cases, the quantified variable in the (gloss) $\cdots \neg \exists x_{Person} \cdots$ of \texttt{God knows} is lost, and what is left is a literal reading $\cdots \neg \texttt{knows}(\texttt{God},\cdots) \cdots$.
Second, whether a sentence is grammatically acceptable may depend on whether an exclamation is involved. \texttt{\textcolor{blue}{God knows}} and \texttt{\textcolor{blue}{who knows}} can be shuffled into the sentence to behave as an intensifier as in:
\[\texttt{Bob drinks \textcolor{blue}{God knows} how many beers}\]
\[\texttt{Bob drinks \textcolor{blue}{who knows} how many beers}\]
But it is awkward to have:
\[\texttt{Bob drank \textcolor{purple}{nobody knows} how many beers}\]
And it is not acceptable to have:
\[\texttt{Bob drank \textcolor{red}{Alice knows} how many beers}\]
\end{example}
}

\begin{example}[Garden path sentences]
So-called "garden path" sentences illustrate that listeners have to make choices to resolve lexical ambiguities. One such garden-path sentence is \texttt{The old man the boat}, where typically readers take \texttt{The old man} as a noun-phrase and \texttt{the boat} as another noun-phrase. We can sketch how the readers might think with a (failed) pregroup grammar derivation:

\[
\AxiomC{$\texttt{the} : n \cdot n^{-1}$}
\AxiomC{$\texttt{old} : n \cdot n^{-1}$}
\BinaryInfC{$\texttt{the\textvisiblespace old}: n \cdot n^{-1}$}
\AxiomC{$\texttt{man} : n$}
\BinaryInfC{$\texttt{the\textvisiblespace old\textvisiblespace man}: n$}
\AxiomC{$\texttt{the} : n \cdot n^{-1}$}
\AxiomC{$\texttt{boat} : n$}
\BinaryInfC{$\texttt{the\textvisiblespace boat}: n$}
\BinaryInfC{\textcolor{red}{Not a sentence!}}
\DisplayProof
\]

So the reader has to backtrack, taking \texttt{The old} as a noun-phrase and \texttt{man} as the transitive verb. This yields a sentence as follows:

\[
\AxiomC{$\texttt{the} : n \cdot n^{-1}$}
\AxiomC{$\texttt{old} : n$}
\BinaryInfC{$\texttt{the\textvisiblespace old}: n$}
\AxiomC{$\texttt{man} : ^{-1}n \cdot s \cdot n^{-1}$}
\BinaryInfC{$\texttt{the\textvisiblespace old\textvisiblespace man}: s \cdot n^{-1}$}
\AxiomC{$\texttt{the} : n \cdot n^{-1}$}
\AxiomC{$\texttt{boat} : n$}
\BinaryInfC{$\texttt{the\textvisiblespace old}: n$}
\BinaryInfC{$\texttt{the\textvisiblespace old\textvisiblespace man\textvisiblespace the\textvisiblespace boat\textvisiblespace}: s$}
\DisplayProof
\]

Garden-path sentences illustrate that listeners must make choices about what grammatical roles to assign words. We make these kinds of contextual decisions all the time with lexically ambiguous words or languages with many homophones; garden-path sentences are special in that they trick the default choices badly enough that the mental effort for correction is noticeable.
\end{example}

\begin{example}[Ambiguous scoping]
Consider the following sentence:
\[\texttt{Everyone loves someone}\]
The sentence is secretly (at least) two, corresponding to two possible parses. The usual reading is (glossed) $\forall x \exists y : \texttt{loves}(x,y)$. The odd reading is $\exists y \forall x : \texttt{loves}(x,y)$: a Raymond situation where there is a single person loved by everyone. We can sketch this difference formally using a simple combinatory categorial grammar.

\[
\AxiomC{$\texttt{everyone} : (n \multimap s) \multimap s$}
\AxiomC{$\texttt{loves} : n \multimap s \ \rotatebox[origin=c]{180}{$\multimap$} \ n$}
\BinaryInfC{$\texttt{everyone\textvisiblespace loves} : s \ \rotatebox[origin=c]{180}{$\multimap$} \ n$}
\AxiomC{$\texttt{someone} : (s \ \rotatebox[origin=c]{180}{$\multimap$} \ n) \multimap s$}
\BinaryInfC{$\texttt{everyone\textvisiblespace loves\textvisiblespace someone} : s$}
\DisplayProof
\]

\[
\AxiomC{$\texttt{everyone} : (n \multimap s) \multimap s$}
\AxiomC{$\texttt{loves} : n \multimap s \ \rotatebox[origin=c]{180}{$\multimap$} \ n$}
\AxiomC{$\texttt{someone} : (s \ \rotatebox[origin=c]{180}{$\multimap$} \ n) \multimap s$}
\BinaryInfC{$\texttt{loves\textvisiblespace someone} : n \multimap s$}
\BinaryInfC{$\texttt{everyone\textvisiblespace loves\textvisiblespace someone} : s$}
\DisplayProof
\]

CCGs have functorial semantics in any cartesian closed category, such as one where morphisms are terms in the lambda calculus and composition is substitution []. So we might specify a semantics as follows:

\begin{align}
[\![ \texttt{everyone} ]\!] = \lambda (\lambda x . V(x)) . \forall x : V(x) \\
[\![ \texttt{loves} ]\!] = \lambda x \lambda y . \texttt{loves}(x,y) \\
[\![ \texttt{someone} ]\!] = \lambda (\lambda y . V(y)) . \exists y : V(y)
\end{align}

Now we can plug-in these interpretations to obtain the two different meanings. We decorate with corners just to visually distinguish which bits are partial first-order logic.

\[
\AxiomC{$\lambda (\lambda x . V(x)) . \ulcorner \forall x : V(x) \urcorner : (n \multimap s) \multimap s$}
\AxiomC{$\lambda x \lambda y . \ulcorner \texttt{loves}(x,y) \urcorner : n \multimap s \ \rotatebox[origin=c]{180}{$\multimap$} \ n$}
\BinaryInfC{$ \lambda y . \ulcorner \forall x : \texttt{loves}(x,y) \urcorner : s \ \rotatebox[origin=c]{180}{$\multimap$} \ n$}
\AxiomC{$\lambda (\lambda y . V(y)) . \ulcorner \exists y : V(y) \urcorner : (s \ \rotatebox[origin=c]{180}{$\multimap$} \ n) \multimap s$}
\BinaryInfC{$ \ulcorner \exists y \forall x : \texttt{loves}(x,y) \urcorner : s$}
\DisplayProof
\]

\[
\AxiomC{$\lambda (\lambda x . V(x)) . \ulcorner \forall x : V(x) \urcorner : (n \multimap s) \multimap s$}
\AxiomC{$\lambda x \lambda y . \ulcorner \texttt{loves}(x,y) \urcorner : n \multimap s \ \rotatebox[origin=c]{180}{$\multimap$} \ n$}
\AxiomC{$\lambda (\lambda y . V(y)) . \ulcorner \exists y : V(y) \urcorner : (s \ \rotatebox[origin=c]{180}{$\multimap$} \ n) \multimap s$}
\BinaryInfC{$\lambda x . \ulcorner \exists y : \texttt{loves}(x,y) \urcorner : n \multimap s$}
\BinaryInfC{$ \ulcorner \forall x \exists y : \texttt{loves}(x,y) \urcorner : s$}
\DisplayProof
\]

\end{example}

\newthought{The speaker's choices and the listener's deductions must be related.} Even if the pair are cooperating, so that the speaker tries to avoid ambiguity, 

\begin{fullwidth}

\subsection{A context free grammar to generate \texttt{Alice sees Bob quickly run to school}}

We can describe a context-free grammar with the same combinatorial rewriting data that specifies string diagrams. For this example, we take the following n-categorical signature. The generators subscripted $L$ (for \emph{label} or \emph{leaf}) correspond to terminals of the CFG, and represent a family of generators indexed by a lexicon for the language. The generators subscripted $i$ (for \emph{introducing a type}) correspond to rewrites of the CFG.

\[\scalebox{0.5}{\tikzfig{tree2gate/cfg/cfgsignature}}\]

Consider the sentence \texttt{Alice sees Bob quickly run to school}, which we take to be generated by the following context-free grammar derivation, read from left-to-right. We additionally depict the breakdown of the derivation in terms of rewrites of lower dimension from our signature.

\[\scalebox{0.5}{\tikzfig{tree2gate/cfg/bigcfgbreakdown}}\]

\[\scalebox{0.5}{\tikzfig{tree2gate/workedexample/bigcfg}}\]

\[\tikzfig{tree2gate/workedexample/biginter}\]
\subsection{Relating the generative grammar to a pregroup grammar via a discrete monoidal fibration}
We merge the monoidal functor-boxes and we slide the bottom edge down using the fibration.
\[\scalebox{0.5}{\tikzfig{tree2gate/workedexample/bigex_0}}
\quad = \quad
\scalebox{0.5}{\tikzfig{tree2gate/workedexample/bigex_1}}
\quad = \quad
\scalebox{0.5}{\tikzfig{tree2gate/workedexample/bigex_2}}\]
\texttt{quickly} could find itself modifying an intransitive (single noun) or transitive (two noun) verb. Suppose that it is the job of some process $\textcolor{orange}{\texttt{q'}}$ to handle intransitive verbs, and similarly $\textcolor{orange}{\texttt{q''}}$ to handle transitive ones. We use the functor for bookkeeping, by asking it to send both $\textcolor{orange}{\texttt{q'}}$ and $\textcolor{orange}{\texttt{q''}}$ to the dependent label $\textcolor{orange}{\bar{\texttt{q}}}$. Diagrammatically, this assignment is expressed by the following equations:
\[\tikzfig{tree2gate/workedexample/gather_quickly}\]
Since the functor is a monoidal discrete fibration, it introduces the appropriate choice of \texttt{quickly} when we pull the functor-box down, while leaving everything else in parallel alone.
\[\scalebox{0.5}{\tikzfig{tree2gate/workedexample/bigex_2}}
\quad = \quad
\scalebox{0.5}{\tikzfig{tree2gate/workedexample/bigex_3}}\]
Adpositions can apply for verbs of any noun-arity. We again use the fiber of the functor for bookkeeping by asking it to send all of the following partial pregroup diagrams to the adposition generator. We consider the pregroup typing of a verb of noun-arity $k \geq 1$ to be $^{-1} n \cdot s \cdot \underbrace{n^{-1} \cdots n^{-1}}_{(k-1)}$.
\[\tikzfig{tree2gate/workedexample/gather_adp}\]
When we pull down the functor-box, the discrete fibration introduces the appropriate choice of diagram from above, corresponding to the intransitive verb case.
\[\scalebox{0.5}{\tikzfig{tree2gate/workedexample/bigex_3}}
\quad = \quad
\scalebox{0.5}{\tikzfig{tree2gate/workedexample/bigex_4}}\]
Similarly to \texttt{quickly}, we suppose we have a family of processes for the word \texttt{to}, one for each noun-arity of verb.
\[\tikzfig{tree2gate/workedexample/gather_to}\]
Again the discrete fibration introduces the appropriate choice of \texttt{to} when we pull the functor box down.
\[\scalebox{0.5}{\tikzfig{tree2gate/workedexample/bigex_4}}
\quad = \quad
\scalebox{0.5}{\tikzfig{tree2gate/workedexample/bigex_5}}\]
Now we visually simplify the inside of the functor-box by applying yanking equations.
\[\scalebox{0.5}{\tikzfig{tree2gate/workedexample/bigex_5}}
\quad = \quad
\scalebox{0.5}{\tikzfig{tree2gate/workedexample/bigex_6}}\]
Similarly as before, we can pull the functor-box past the intransitive verb node. There is only one pregroup type $^{-1}n \cdot s$ that corresponds to the grammatical category $\textcolor{green}{\texttt{(I)VP}}$.
\[\scalebox{0.5}{\tikzfig{tree2gate/workedexample/bigex_6to7}}\]
Proceeding similarly, we can pull the functor-box past the sentential-complement-verb node. There are multiple possible pregroup types for $\textcolor{ForestGreen}{\texttt{SCV}}$, depending on how many noun-phrases are taken as arguments in addition to the sentence. For example, in $\texttt{Alice \textcolor{ForestGreen}{sees} \textcolor{cyan}{[sentence]}}$, $\textcolor{ForestGreen}{\texttt{sees}}$ returns a sentence after taking a noun to the left and a sentence to the right, so it has pregroup typing $^{-1}n \cdot s \cdot s^{-1}$. On the other hand, for something like $\texttt{Alice \textcolor{ForestGreen}{\texttt{tells}} Bob \textcolor{cyan}{[sentence]}}$, $\textcolor{ForestGreen}{\texttt{tells}}$ returns a sentence after taking a noun (the teller) to the left, a noun (the tellee) to the left, and a sentence (the story) to the left, so it has a pregroup typing $^{-1}n \cdot s \cdot n^{-1} \cdot s^{-1}$. These two instances of sentential-complement-verbs are introduced by different nodes. We can record both of these pregroup typings in the functor by asking for the following:
\[\tikzfig{tree2gate/workedexample/gather_scv}\]
Pulling down the functor box:
\[\scalebox{0.5}{\tikzfig{tree2gate/workedexample/bigex_7}}
\quad = \quad
\scalebox{0.5}{\tikzfig{tree2gate/workedexample/bigex_8}}\]
As before, we can ask the functor to send an appropriate partial pregroup diagram to the dependent label $\textcolor{ForestGreen}{\bar{\texttt{see}}}$.
\[\scalebox{0.5}{\tikzfig{tree2gate/workedexample/bigex_8to9}}\]
Now again we can visually simplify using the yanking equation and isotopies, which obtains a pregroup diagram.
\[\tikzfig{tree2gate/workedexample/bigex_a0}\]
The pregroup diagram corresponds to a particular pregroup proof of the syntactic correctness of the sentence \texttt{Alice sees Bob run quickly to school}.
\[
\AxiomC{$\texttt{A} : n$}
\AxiomC{$\textcolor{green}{\texttt{s}} : ^{-1}n \cdot s \cdot s^{-1}$}
\BinaryInfC{$\texttt{A\textvisiblespace \textcolor{green}{s}} : s \cdot s^{-1}$}
\AxiomC{$\texttt{B} : n$}
\AxiomC{$\textcolor{orange}{\texttt{q}} :  (^{-1}n \cdot s) \cdot ( ^{-1}n \cdot s)^{-1} $}
\AxiomC{$\textcolor{green}{\texttt{r}} : ^{-1}n \cdot s$}
\BinaryInfC{$\texttt{\textcolor{orange}{q}\textvisiblespace \textcolor{green}{r}} : ^{-1}n \cdot s$}
\AxiomC{$\textcolor{blue}{\texttt{t}} : ^{-1}( ^{-1}n \cdot s) \cdot ( ^{-1}n \cdot s) \cdot n^{-1}$}
\BinaryInfC{$\texttt{\textcolor{orange}{q}\textvisiblespace \textcolor{green}{r}\textvisiblespace \textcolor{blue}{t}} : (^{-1}n \cdot s) \cdot n^{-1}$}
\AxiomC{$\texttt{S} : n$}
\BinaryInfC{$\texttt{\textcolor{orange}{q}\textvisiblespace \textcolor{green}{r}\textvisiblespace \textcolor{blue}{t}\textvisiblespace S} : ^{-1}n \cdot s$}
\BinaryInfC{$\texttt{B\textvisiblespace \textcolor{orange}{q}\textvisiblespace \textcolor{green}{r}\textvisiblespace \textcolor{blue}{t}\textvisiblespace S} : s$}
\BinaryInfC{$\texttt{A\textvisiblespace \textcolor{green}{s}\textvisiblespace B\textvisiblespace \textcolor{orange}{q}\textvisiblespace \textcolor{green}{r}\textvisiblespace \textcolor{blue}{t}\textvisiblespace S} : s$}
\DisplayProof
\]

\begin{remark}
Technical addendums.\\
The above construction only requires the source category of the functor to be rigid autonomous. Since no braidings are required, the free autonomous completion [Antonificaton] of any monoidal category may be used.\\
To enforce the well-definedness of the functor $\mathbf{F}: \mathcal{P}\mathcal{G} \rightarrow \mathcal{G}$ on objects, we may consider the strictified "category of..." [ghicadiagrams]...
\end{remark}

\subsection{Discrete Monoidal Fibrations}

We introduce the concept of a discrete monoidal fibration: a mathematical bookkeeping tool that relates kinds of choices speakers and listerns make when generating and parsing text respectively. We proceed diagrammatically. The first concept is that of a \emph{monoidal functor box}.

\marginnote{
\begin{scholium}
Functor boxes are from [meillies].
Expressing the coherence conditions of monoidal functors using equations involving functor boxes as below is not new [meillies]
The idea of a functor being simultaneously monoidal and a fibration is not new [monoidalfibration].
What is new is minor: the express requirement that the lifts of the fibration satisfy interchange, which is in general not guaranteed by just having a functor be monoidal and a (even discrete) fibration [fosco].
\end{scholium}
}

Suppose we have a functor between monoidal categories $\mathbf{F}: \mathcal{C} \rightarrow \mathcal{D}$. Then we have the following diagrammatic representation of a morphism $\mathbf{F}A \overset{\mathbf{F}f}{\rightarrow} \mathbf{F}B$ in $\mathcal{D}$:

\[\tikzfig{tree2gate/mfunctorbox/mfbox-notation}\]

The use of a functor box is like a window from the target category $\mathcal{D}$ into the source category $\mathcal{C}$; when we know that a morphism in $\mathcal{D}$ is the image under $\mathbf{F}$ of some morphism in $\mathcal{C}$, the functor box notation is just a way of presenting all of that data at once. Since $\mathbf{F}$ is a functor, we must have that $\mathbf{F}f ; \mathbf{F}g = \mathbf{F}(f;g)$. Diagrammatically this equation is represented by freely splitting and merging functor boxes vertically.

\[\tikzfig{tree2gate/mfunctorbox/mfbox-seq}\]

Assume that $\mathbf{F}$ is strict monoidal; without loss of generality by the strictification theorem [], this lets us gloss over the associators and unitors. For $\mathbf{F}$ to be strict monoidal, it has to preserve monoidal units and tensor products on the nose: i.e. $\mathbf{F}I_\mathcal{C} = I_\mathcal{D}$ and $\mathbf{F}A \otimes_\mathcal{D} \mathbf{F}B = \mathbf{F}(A \otimes_\mathcal{C} B)$. Diagrammatically these structural constraints amount to the following equations:

\[\tikzfig{tree2gate/mfunctorbox/mfbox-fibration+structural}\]

What remains is the monoidality of $\mathbf{F}$, which is the requirement $\mathbf{F}f \otimes \mathbf{F}g = \mathbf{F}(f \otimes g)$. Diagrammatically, this equation is represented by freely splitting and merging functor boxes horizontally; analogously to how splitting vertically is the functor-boxes' way of respecting sequential composition, splitting horizontally is how they respect parallel composition.
\[\tikzfig{tree2gate/mfunctorbox/mfbox-tensor}\]

And for when we want $\mathbf{F}$ to be a (strict) symmetric monoidal functor, we are just asking that boxes and twists do not get stuck on one another.
\[\tikzfig{tree2gate/mfunctorbox/mfbox-twist}\]

\begin{remark}
To motivate fibrations, first observe that by the diagrammatic equations of monoidal categories and functor boxes we have so far, we can always "slide out" the contents of a functor box out of the bottom:
\[\tikzfig{tree2gate/mfunctorbox/mfbox-prefibex}\]
When can we do the reverse? That is, take a morphism in $\mathcal{D}$ and \emph{slide it into} a functor box? We know that in general this is not possible, because not all morphisms in $\mathcal{D}$ may be in the image of $\mathbf{F}$. So instead we ask "under what circumstances" can we do this for a functor $\mathbf{F}$? The answer is when $\mathbf{F}$ is a discrete fibration.
\end{remark}

\begin{defn}[Discrete opfibration]
$\mathbf{F}: \mathcal{C} \rightarrow \mathcal{D}$ is a \emph{discrete fibration} when:\\
for all morphisms $f: \mathbf{F}A \rightarrow B$ in $\mathcal{D}$ with domain in the image of $\mathbf{F}$...\\
there exists a unique object $\Phi^A_f$ and a unique morphism $\phi_f: A \rightarrow \Phi^A_f$ in $\mathcal{C}$...\\
such that $f = \mathbf{F}\phi_f$.\\

Diagrammatically, we can present all of the above as an equation reminiscent of sliding a morphism \emph{into} a functor box from below.
\[\tikzfig{tree2gate/mfunctorbox/mfbox-fibration}\]
\end{defn}

\begin{defn}[Monoidal discrete opfibration]
We consider $\mathbf{F}$ to be a \emph{(strict, symmetric) monoidal discrete opfibration} when it is a (strict, symmetric) monoidal functor, a discrete opfibration, and the following equations relating lifts to interchange hold:
\[\scalebox{0.75}{\tikzfig{tree2gate/mfunctorbox/mfbox-fibration+interchange}}\]
\end{defn}
\marginnote{
\begin{remark}
The diagrammatic motivation for the additional coherence equations is that -- if we view the lifts of opfibrations as sliding morphisms into functor boxes -- we do not want the order in which sliding occurs to affect the final result. In this way, lifts behave as `graphical primitives' in the same manner as interchange isotopies and symmetry twists.
\end{remark}
}

\subsection{Discrete monoidal fibrations for grammatical functions}

\subsection{Extended analysis: Tree Adjoining Grammars}

Here is a formal but unenlightening defintion of tree adjoining grammars, which we will convert to diagrams.

\begin{defn}[Tree Adjoining Grammar: Classic Computer Science style]
A \textbf{TAG} is a tuple $(\mathcal{N}, \mathcal{N}^\downarrow, \mathcal{N}^\star, \Sigma, \mathcal{I}, \mathcal{A}, \mathbf{s} \in \mathcal{N})$. These denote, respectively:
\begin{itemize}
	\item{ The \emph{non-terminals}:
		\begin{itemize}
			\item{A set of \emph{non-terminal symbols} $\mathcal{N}$ -- these stand in for grammatical types such as $\texttt{NP}$ and $\texttt{VP}$.}
			\item{A bijection $\downarrow: \mathcal{N} \rightarrow \mathcal{N}^\downarrow$ which acts as $\texttt{X} \mapsto \texttt{X}^\downarrow$. Nonterminals in $\mathcal{N}$ are sent to marked counterparts in $\mathcal{N}^\downarrow$, and the inverse sends marked nonterminals to their unmarked counterparts. These markings are \emph{substitution markers}, which are used to indicate when certain leaf nodes are valid targets for a substitution operation -- discussed later.}
			\item{A bijection $\star: \mathcal{N} \rightarrow \mathcal{N}^\star$ -- the same idea as above. This time to mark \emph{foot nodes} on auxiliary trees, which is structural information used by the adjoining operation -- discussed later.}
	\end{itemize}
	}
	\item{A set of \emph{terminal symbols} $\Sigma$ -- these stand in for the words of the natural language being modelled.}
	\item{The \emph{elementary trees}:
		\begin{itemize}
			\item{A set of \emph{initial trees} $\mathcal{I}$, which satisfy the following constraints:
			\begin{itemize}
				\item{The interior nodes of an initial tree must be labelled with nonterminals from $\mathcal{N}$}
				\item{The leaf nodes of an initial tree must be labelled from $\Sigma \cup \mathcal{N}^{\downarrow}$}
			\end{itemize}
			}
	\item{A set of \emph{auxiliary trees} $\mathcal{A}$, which satisfy the following constraints:
		\begin{itemize}
			\item{The interior nodes of an auxiliary tree must be labelled with nonterminals from $\mathcal{N}$}
			\item{Exactly one leaf node of an auxiliary tree must be labelled with a foot node $\texttt{X}^{\star} \in \mathcal{N}^{\star}$; moreover, this labelled foot node must be the marked counterpart of the root node label of the tree.}
			\item{All other leaf nodes of an auxiliary tree are labelled from $\Sigma \cup \mathcal{N}^{\downarrow}$}
		\end{itemize}
		}
	\end{itemize}
	}
\end{itemize}
There are two operations to build what are called \emph{derived trees} from elementary and derived trees. These operations are called \emph{substitution} and \emph{adjoining}.
\begin{itemize}
	\item{\emph{Substitution} replaces a substitution marked leaf node $\texttt{X}^\downarrow$ in a tree $\alpha$ with another tree $\alpha'$ that has $\texttt{X}$ as a root node.}
	\item{\emph{Adjoining} takes auxiliary tree $\beta$ with root and foot nodes $\texttt{X},\texttt{X}^\star$, and a derived tree $\gamma$ at an interior node $\texttt{X}$ of $\gamma$. Removing the $\texttt{X}$ node from $\gamma$ separates it into a parent tree with an $\texttt{X}$-shaped hole for one of its leaves, and possibly multiple child trees with \texttt{X}-shaped holes for roots. The result of adjoining is obtained by identifying the root of $\beta$ with the $\texttt{X}$-context of the parent, and making all the child trees children of $\beta$s foot node $\texttt{X}^\star$.}
\end{itemize}
\end{defn}

The essence of a tree-\emph{adjoining} grammar is as follows: whereas for a CFG one grows the tree by appending branches and leaves at the top of the tree (substitution), in a TAG one can also sprout subtrees from the middle of a branch (adjoining). Now we show that this gloss is more formal than it sounds, by the following steps. First we show that the 2-categorical data of a CFG can be transformed into 3-categorical data -- which we call \emph{Leaf-Ansatz} -- which presents a rewrite system that obtains the same sentences as the CFG, by a bijective correspondence between composition of 2-cells in the CFG and constructed 3-cells in the leaf-ansatz. These 3-cells in the leaf ansatz correspond precisely to the permitted \emph{substitutions} in a TAG. Then we show how to model \emph{adjoining} as 3-cells. Throughout we work with a running example, the CFG grammar introduced earlier.

\begin{construction}[Leaf-Ansatz of a CFG]
Given a signature $\mathfrak{G}$ for a CFG, we construct a new signature $\mathfrak{G}'$ which has the same 0- and 1-cells as $\mathfrak{G}$. See the dashed magenta arrows in the schematic below. For each 1-cell wire type $\texttt{X}$ of $\mathfrak{G}$, we introduce a \emph{leaf-ansatz} 2-cell $\texttt{X}^\downarrow$. For each leaf 2-cell $\texttt{X}_L$ in $\mathfrak{G}$, we introduce a renamed copy $\texttt{X}'_L$ in $\mathfrak{G}'$. Now see the solid magenta arrows in the schematic below. We construct a 3-cell in $\mathfrak{G}'$ for each 2-cell in $\mathfrak{G}$, which has the effect of systematically replacing open output wires in $\mathfrak{G}$ with leaf-ansatzes in $\mathfrak{G}'$.
\[\scalebox{0.5}{\tikzfig{tree2gate/tag/CFGtoTAGpattern}}\]
\end{construction}

\begin{example}
The leaf-ansatz construction just makes formal the following observation: there are multiple equivalent ways of modelling terminal symbols in a rewrite system considered string-diagrammatically. One way (which we have already done) is to treat non-terminals as wires and terminals as effects, so that the presence of an open wire available for composition visually indicates non-terminality. Another (which is the leaf-ansatz construction) treats all symbols in a rewrite system as leaves, where bookkeeping the distinction between (non-)terminals occurs in the signature. So for a sentence like \texttt{Bob drinks}, we have the following derivations that match step for step in the two ways we have considered.
\[\scalebox{0.5}{\tikzfig{tree2gate/tag/leafansatzintuition}}\]
\end{example}

\begin{proposition}[Leaf-ansatzes of CFGs are precisely TAGs with only initial trees and substitution]\label{prop:cfgastag1}
\begin{proof}
By construction. Consider a CFG given by 2-categorical signature $\mathfrak{G}$, with leaf-ansatz signature $\mathfrak{G}'$. The types $\texttt{X}$ of $\mathfrak{G}$ become substitution marked symbols $\texttt{X}^{\downarrow}$ in $\mathfrak{G}'$. The trees $\texttt{X}_i$ in $\mathfrak{G}$ become initial trees $\texttt{X}^0$ in $\mathfrak{G}'$. The 3-cells $\texttt{X}_s$ of $\mathfrak{G}'$ are precisely substitution operations corresponding to appending the 2-cells $\texttt{X}_i$ of $\mathfrak{G}$.
\end{proof}
\end{proposition}

\begin{example}[Leaf-ansatz signature of \texttt{Alice sees Bob quickly run to school} CFG]
\[\scalebox{0.5}{\tikzfig{tree2gate/tag/CFGasTAGsign}}\]
\end{example}

\begin{example}[Adjoining is sprouting subtrees in the middle of branches]
One way we might obtain the sentence \texttt{Bob runs to school} is to start from the simpler sentence \texttt{Bob runs}, and then refine the verb \texttt{runs} into \texttt{runs to school}. This refinement on part of an already completed sentence is not permitted in CFGs, since terminals can no longer be modified. The adjoining operation of TAGs gets around this constraint by permitting rewrites in the middle of trees, as follows:
\[\scalebox{0.5}{\tikzfig{tree2gate/tag/3compareintuition}}\]
\end{example}

\begin{example}[TAG signature of \texttt{Alice sees Bob quickly run to school}]
The highlighted 2-cells are auxiliary trees that replace CFG 2-cells for verbs with sentential complement, adverbs, and adpositions. The highlighted 3-cells are the tree adjoining operations of the auxiliary trees.
\[\scalebox{0.5}{\tikzfig{tree2gate/tag/tagsignature}}\]
\end{example}

The construction yields as a corollary an alternate proof of Theorem [Joshi 6.1.1...]...

\begin{corollary}
For every context-free grammar $\mathfrak{G}$ there exists a tree-adjoining grammar $\mathfrak{G}'$ such that $\mathfrak{G}$ and $\mathfrak{G}'$ are strongly equivalent -- both formalisms generate the same set of strings (weak equivalence) and the same abstract syntactic structures (in this case, trees) behind the strings (strong equivalence).
\begin{proof}
Proposition \ref{prop:cfgastag1} provides one direction of both equivalences. For the other direction, we have to show that each auxiliary tree (a 2-cell) and its adjoining operation (a 3-cell) in $\mathfrak{G}'$ corresponds to a single 2-cell tree of some CFG signature $\mathfrak{G}$, which we demonstrate by construction. See the example above; the highlighted 3-cells of $\mathfrak{G}'$ are obtained systematically from the auxiliary 2-cells as follows: the root and foot nodes $\texttt{X},\texttt{X}^\star$ indicate which wire-type to take as the identity in the left of the 3-cell, and the right of the 3-cell is obtained by replacing all non-$\texttt{X}$ open wires $\texttt{Y}$ with their leaf-ansatzes $\texttt{Y}^\downarrow$. This establishes a correspondence between any 2-cells of $\mathfrak{G}$ considered as auxiliary trees in $\mathfrak{G}'$.
\end{proof}
\end{corollary}

\[\scalebox{0.1}{\tikzfig{tree2gate/workedexample/bigexaltogether}}\]

\end{fullwidth}