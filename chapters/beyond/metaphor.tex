\section{Modelling metaphor}

I will take a \emph{metaphor} to be text where systematic language for one kind of concept is used to structure meanings for another kind of concept where literal meanings cannot apply. This may subsume some cases of what would otherwise be called \emph{similes} or \emph{analogies}.\\

\newthought{The idea:} First, we observe that we can model certain kinds of analogies between conceptual spaces by considering structure-preserving maps between them. For example, Planck's law gives a partial continuous function from part of the positive reals measuring Kelvin to wavelengths of light, and the restriction of this mapping to the visible spectrum gives the so-called "colour temperature" framework used by (image-professionals?).

Second, we observe that we can use simple natural language to describe conceptual spaces, instead of geometric or topological models. Back to the example of colour temperature, instead of precise values in Kelvin, we may instead speak of landmark regions that represent both temperature and colour such as "candle", "incandescent", "daylight", which obey both temperature-relations (e.g. candle is a lower temperature than incandescent) and colour-relations (daylight is bluer than incandescent).

Third, we observe that we can describe complex conceptual schemes using simple natural language, and formally specify what it means for one conceptual scheme to structure another by describing structure-preserving maps between the text circuits of the descriptions. This will allow us to reason about and calculate with complex metaphors, such as "Time is Money".

\newthought{The motivation:}

\subsection{Orders, Temperature, Colour, Mood}

\subsection{Complex conceptual structure}

\newthought{Time is Money}\\
\texttt{"Do you have time to look at this?"}\\
\texttt{"Do you think this is worth the time?"}\\
\texttt{"What a waste of time."}

I will work through an example of partially using the concept of Money to structure that of Time. Part of the concept of Money is that it can be \emph{exchanged} for something. The concept of exchange can be glossed approximately as the following text, with variable noun-entries capitalised.

\texttt{AGENT has THEME.}\\
\texttt{AGENT wants OBJECT.}\\
\texttt{AGENT gives THEME to PATIENT in-order-to get OBJECT.}

\[\tikzfig{metaphor/timeISmoney}\]

The 

\subsection{}