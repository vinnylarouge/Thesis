\subsection{Two attitudes towards figurative language, and their synthesis:}\\

\emph{The Martian} would stumble on the feasibility of possessing time, let alone the possibility of exchanging time for something advantageously. How does one literally waste time? What are the truth conditions of such figures of speech? What in the empirical world can be measured to ascertain those truth conditions? How might we explicitly structure conceptual frames to capture the figurative language? GOFAI died in martian valleys such as these.\\

\emph{Searle's tenant} would use figurative language completely fluidly and context-appropriately, but while the occasional idiom or proverb is a feather in the speaker's cap, it's a different ballgame if the speaker can't call a spade a spade.\\

This section is about reconciling the structured and compositional approach of the martian with the purely associative babble of the tenant. Consider again the previous paragraph:\\

\texttt{\emph{Searle's tenant} would use figurative language completely fluidly, always but only with context in mind. The occasional idiom or proverb is a feather in the speaker's cap, but it's a different ballgame if the speaker can't call a spade a spade.}\\

What would that look like in terms acceptable for the martian? Maybe something like this:\\

\texttt{"\emph{Searle's tenant}" designates a kind of language user adept in the use of figurative language, with connotations of nonsystematic associative behaviour from Searle's eponymous \emph{Chinese Room} thought experiment. The tenant's use of figurative language may always be contextually appropriate without implying complex mental states, as figurative language can be employed as if reflexively to contextual stimuli. While occasional use of figurative or idiomative language is a signal of language competence and intelligence broadly construed, there is strong indication of the oppposite if the speaker is incapable of re-expressing figurative language in other, meaningfully equivalent terms.}