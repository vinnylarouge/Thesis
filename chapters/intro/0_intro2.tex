This thesis is about representing natural language text using string diagrams.

(looks like this)

We all know a thing or two about natural language, so let's start with string diagrams. String diagrams are a mathematically formal syntax for representing complex, composite systems. I say \emph{mathematically} formal to emphasise that string diagrams are not merely heuristic tools backed by a handbook of standards decided by committee: they are unambiguous mathematical objects that you can bet your life on. [Cites]

(1 + 1 = 2)

Just as crustaceans independently converge to crab-like shapes by what is called \emph{carcinisation}, formal notation for formal theories of "real world" problem domains undergo "string diagrammatisation". Why should that be so? Our best formal theories of the "real world" treat complexity as the outcome of composing simple interacting parts; perhaps nature really works that way, or we cannot help but express ourselves using composition.

When one has many different processes sending information to each other via channels, it becomes tricky to keep track of all the connections using one-dimensional syntax; if there are $N$ processes, there may be on the order of $\mathcal{O}(N^2)$ connections, which quickly becomes unmanageable to write down in a line, prompting the development of indices in notation. In time, probably by doodling a helpful line during calculation to match indices, connected indices become wires, and string diagrams are born.

